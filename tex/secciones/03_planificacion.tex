\chapter{\textbf{Planificación}}
La planificación representa una fase crucial en cualquier proyecto de desarrollo, especialmente en uno que requiera de soluciones software. Este capítulo resume el proceso organizativo y estratégico adoptado para llevar a buen término el presente trabajo, tras haber descrito la motivación del mismo en el anterior.

\section{Temporización}
La temporización define el marco temporal en el cual se desarrollarán las distintas actividades y tareas del proyecto. En esta sección, se detallarán las fases y etapas del proyecto, junto con los tiempos asignados a cada una, ofreciendo así una visión clara del recorrido temporal propuesto para alcanzar los objetivos planteados.

\subsection{Primera fase: análisis del problema}
Desde el mes de Junio hasta mediados del mes de Julio se realizó un estudio exhaustivo del problema, recopilando toda la información posible a base de testimonios de personas reales y el estudio de la teoría que concierne al marco central del proyecto. Acto seguido, se definieron las Historias de Usuario y los Milestones del proyecto. Esta fase corresponde al desarrollo del Milestone 0 (~\ref{subsection:PMV0} ).

\subsection{Segunda fase: estudio y diseño del lenguaje Lamport}
Desde mediados del mes de Julio a comienzos de Agosto se realizó el estudio del pseudocódigo de la asignatura y la sintetización de la gramática del lenguaje Lamport, desde sus tokens hasta sus reglas sintácticas. Esta fase también corresponde al desarrollo del Milestone 0 (~\ref{subsection:PMV0} ).

\subsection{Tercera fase: implementación de los analizadores del lenguaje}
Desde comienzos de Agosto hasta comienzos de Septiembre se trabajó en la implementación de los tres analizadores del lenguaje: léxico, sintáctico y semántico. El objetivo era disponer de una estructura de datos computacional extraída del código escrito en Lamport, para posteriormente trabajar con él en fases posteriores. Esta fase corresponde al desarrollo del milestone 1 (~\ref{subsection:PMV1} ), milestone 2 (~\ref{subsection:PMV2} ) y milestone 3 (~\ref{subsection:PMV3} ).

\subsection{Cuarta fase: ejecución del lenguaje Lamport}
Desde el mes de Septiembre hasta Octubre se trabajó en la implementación del generador de código intermedio y de la Máquina Virtual que ejecuta dichas instrucciones. En este punto, ya se tiene un lenguaje totalmente funcional. Esta fase corresponde al desarrollo del milestone 4 (~\ref{subsection:PMV4} ) y milestone 5 (~\ref{subsection:PMV5} ).

\subsection{Quinta fase: desarrollo teórico}
Durante el mes de octubre y noviembre se recuperan algunos aspectos estudiados en las primeras fases, y se trabaja en el desarrollo teórico tanto en la base matemática como en la base informática. Esta fase corresponde al desarrollo del Milestone 6 (~\ref{subsection:PMV6} ).



\section{Seguimiento del desarrollo}
Puesto que el desarrollo de el proyecto se ha realizado en un proyecto de GitHub, se puede realizar el seguimiento completo del mismo accediendo al repositorio:

\begin{center}
    \href{https://github.com/danielsp13/lamport}{github::danielsp13/lamport}
\end{center}