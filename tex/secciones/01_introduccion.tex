\chapter{\textbf{Introducción}}

\section{Motivación}
En el mundo actual, marcado por un avance tecnológico acelerado y una creciente digitalización, la importancia de los sistemas concurrentes y distribuidos se ha vuelto más evidente que nunca. Estos sistemas son fundamentales para el funcionamiento eficiente y óptimo de numerosas aplicaciones y servicios que forman la columna vertebral de la sociedad moderna. Desde el procesamiento de datos a gran escala hasta la computación en la nube, pasando por las redes de telecomunicaciones y la proliferación de la Internet de las Cosas (IoT), los sistemas concurrentes y distribuidos permiten el manejo y procesamiento simultáneo de enormes cantidades de información, asegurando la escalabilidad y la disponibilidad de los servicios.

En este entorno, la habilidad para diseñar, implementar y mantener sistemas que puedan manejar múltiples tareas de manera simultánea y distribuida es más que una habilidad técnica; es una necesidad vital. Con la integración creciente de la inteligencia artificial en diversos sectores, la eficiencia en la gestión de procesos concurrentes y la distribución efectiva de las cargas de trabajo no son solo deseables, sino esenciales para el avance y la innovación tecnológica. Además, la garantía de fiabilidad y seguridad en estos sistemas se convierte en un reto crítico, dado que cualquier fallo puede tener repercusiones significativas en un amplio espectro de actividades humanas.

En el grado de Ingeniería Informática de la Universidad de Granada, se dedica una asignatura completa al estudio de estos sistemas, denominada \textit{Sistemas Concurrentes y Distribuidos (SCD)}, donde se exploran tanto sus aspectos teóricos como prácticos. Esta asignatura no solo proporciona a los estudiantes una comprensión profunda de los fundamentos de los sistemas concurrentes y distribuidos, sino que también los introduce en el rigor matemático necesario para la verificación de sus propiedades utilizando un sistema lógico formal. Esta área de estudio es crucial, ya que asegura que los sistemas diseñados sean fiables, seguros y eficientes. Al enfrentarse a la complejidad inherente de estos sistemas, los alumnos aprenden a aplicar técnicas avanzadas de modelado y análisis.

La contrapartida es que su comprensión y aplicación práctica puede ser desafiante, especialmente para quienes se acercan por primera vez a la materia, y es por ello por lo que es crucial disponer de suficientes recursos y herramientas de apoyo que permitan alcanzar un aprendizaje íntegro y efectivo en este campo dinámico y en constante evolución.


\section{Objetivos del trabajo}
El propósito central de este proyecto es el desarrollo de un lenguaje de programación intuitivo y accesible , acompañado de un compilador específico, para facilitar el desarrollo y simulación de sistemas concurrentes. Este enfoque práctico busca proporcionar una alternativa a los lenguajes de programación de alto nivel convencionales, como C++, simplificando así el proceso de aprendizaje y enseñanza desde una perspectiva más accesible y transparente. Adicionalmente, este trabajo se adentra en un estudio sobre los fundamentos matemáticos subyacentes, centrándose en la Lógica Temporal de Acciones propuesta por Leslie Lamport. Este estudio es crucial para la especificación y verificación de los sistemas concurrentes, proporcionando una base teórica sólida que complementa y enriquece la comprensión práctica. Además, se explorarán los fundamentos detrás del procesamiento de lenguajes, esenciales para el diseño efectivo del lenguaje de programación propuesto. La combinación de estos enfoques prácticos y teóricos no solo busca lograr un entendimiento integral de los sistemas concurrentes y distribuidos, sino también fomentar una sinergia entre la teoría matemática y la aplicación práctica en el campo de la informática. A modo de resumen, los objetivos a cumplimentar son:

\begin{itemize}
    \item Desarrollar un marco teórico que abarque los conceptos clave de los sistemas concurrentes y distribuidos, proporcionando una comprensión clara y detallada de su funcionamiento, principios y desafíos. Incluir un estudio de los fundamentos matemáticos de las lógicas formales que sustentan las bases de la Lógica Temporal de Acciones, haciendo énfasis en su aplicación para la especificación y verificación de sistemas concurrentes. Se abordarán los principios fundamentales de la lógica y del tiempo, explorando cómo se integran y aplican en el contexto de la computación concurrente y distribuida.
    \item Llevar a cabo el desarrollo de un proyecto de software desde sus inicios utilizando metodologías de desarrollo ágil, poniendo el foco siempre en los clientes, que son las personas interesadas en el producto y quienes verdaderamente guían a un equipo hacia su progreso.
    \item Realizar un análisis y descripción detallada del lenguaje de programación a desarrollar, denominado \textit{Lamport}, tomando como punto de partida la sintaxis de los ejemplos de pseudocódigo de la asignatura \textit{Sistemas Concurrentes y Distribuidos} del grado de Ingeniería Informática de la Universidad de Granada.
    \item Implementar un compilador del código descrito, teniendo en cuenta las fases de análisis y optimización. Posteriormente, modelar una máquina virtual capaz de interpretar directamente el código generado, mostrando la ejecución resultante de las instrucciones al usuario, además de otorgarle transparencia en los sucesivos eventos que han ido ocurriendo por dentro, con el objetivo de garantizar un aprendizaje integral.
\end{itemize}

\section{Contenido del documento}\label{section:documentInfo}
Este documento está estructurado en 5 partes principales, cada una enfocada en aspectos distintos y fundamentales del proyecto. A continuación, se detalla el contenido de cada parte:

\begin{itemize}
    \item \textbf{Parte 1: Introducción y Motivación - El Origen y Propósito del Proyecto}. Esta sección explora la motivación detrás del proyecto, el problema que busca resolver y la metodología aplicada. Se detalla la planificación y evolución del proyecto, incluyendo un enlace al repositorio de GitHub para seguimiento práctico. Comprende los capítulos 1, 2 y 3.
    \item \textbf{Parte 2: Estado del Arte - El Contexto y la Relevancia Actual}. Se presenta un contexto histórico sobre la evolución de los sistemas concurrentes y distribuidos, destacando la contribución única del proyecto a esta área. Comprende el capítulo 4.
    \item \textbf{Parte 3: Fundamentos Teóricos y Matemáticos - La Base Conceptual del Proyecto}. Se realiza un análisis teórico completo de los sistemas concurrentes y distribuidos, abarcando desde la teoría matemática de sistemas lógicos formales, hasta los aspectos necesarios para la implementación práctica. Comprende los capítulos 5, 6 y 7.
    \item \textbf{Parte 4: Diseño e Implementación del Lenguaje de Programación Lamport - La Construcción de una Herramienta Innovadora}. Se tratan los aspectos técnicos del lenguaje Lamport, desde su diseño hasta la compilación, resaltando la importancia de los ejemplos prácticos para demostrar su eficacia. Comprende los capítulos 8, 9, 10, 11.
    \item \textbf{Parte 5: Conclusiones - Reflexiones y Perspectivas Futuras}. Se ofrecen reflexiones finales sobre los temas tratados y se sugieren direcciones futuras para la herramienta desarrollada, resumiendo los logros y aprendizajes clave del proyecto. Comprende el capítulo 12.

    
\end{itemize}