\chapter{\textbf{Introducción}}

\section{Motivación}
El mundo y la tecnología avanzan a pasos agigantados. El lapso de tiempo entre un hito importante en la historia de la informática y otro inmediatamente anterior es cada vez más corto. Hace años quedaron atrás los gigantes armarios llenos de circuitería que procesaban bandas magnéticas con cálculos muy específicos. Ahora, en un simple grano de arroz podrían existir múltiples unidades de cálculo. Es una evolución normal, consecuencia de las necesidades de este siglo, que también crecen. En un mundo globalizado y universal que te permite viajar hasta las mismas antípodas, la cooperación entre países, y seres humanos, es de vital importancia. Los programas, con sus singulares algoritmos y propósitos, también se vieron obligados a cooperar. Y es que precisamente los programas dejaron de ser \textit{singulares}, para ser \textit{plurales}. Ahora son como aquellos hermanos que comparten todo, datos, memoria, procesador, pero no necesariamente siempre van a poder hacerlo al mismo tiempo. Esto es lo que se conoce como un sistema concurrente, donde múltiples unidades ejecutan instrucciones lógicamente al mismo tiempo. Cuando se dio luz a este concepto, a mediados del siglo pasado, el mundo dejó de ser el mismo. La velocidad de procesamiento de datos frutos de la investigación científica aumentó considerablemente al poder desarrollar más tareas en menos tiempo, el desarrollo de nuevas máquinas por parte de los gigantes tecnológicos de aquella época dio un giro importante, haciendo más accesible la informática al olvidarse de su carácter elitista, bajando hacia lo más mundano y cotidiano, no sólo visto a través de los ordenadores de sobremesa y portátiles, sino también la forma de interacción humana, generando un alter ego digital. 

En la actualidad, internet reina con soberanía. Los videojuegos, los servicios de \textit{streaming} bajo demanda, y la inquietud constante del ser humano de aumentar su desarrollo como persona en múltiples campos como la medicina, seguridad, economía, defensa, justicia, e igualdad, hacen que la concurrencia siga siendo un concepto no sólo importante, sino necesario. Esta necesidad de concurrencia se extiende aún más en el ámbito de los sistemas distribuidos. La creciente demanda de aplicaciones y servicios que operan en tiempo real y a gran escala, como las redes sociales, el comercio electrónico y las plataformas de análisis de datos, subraya la importancia de sistemas que no solo procesen tareas en paralelo, sino que también compartan y sincronicen datos a través de múltiples nodos y geografías.

El desafío de diseñar y mantener estos sistemas complejos es enorme. No solo deben ser eficientes y rápidos, sino también robustos y seguros. La concurrencia y la distribución presentan desafíos únicos, especialmente en la gestión de recursos compartidos, la prevención de condiciones de carrera, y la garantía de la coherencia de los datos. Además, la creciente preocupación por la privacidad y seguridad de los datos en aplicaciones distribuidas agrega una capa adicional de complejidad al diseño de estos sistemas.

Este trabajo introduce al lector en las nociones básicas de los sistemas concurrentes, incorporando además un formalismo matemático para la verificación de las propiedades de estos sistemas. A continuación, se presenta el elemento central del proyecto: la creación de un lenguaje de programación y su correspondiente intérprete. El objetivo es diseñar y simular programas que requieran la creación y sincronización de múltiples procesos, eliminando la necesidad de dominar un lenguaje de programación de alto nivel convencional o de utilizar bibliotecas externas complejas. Esto se logra mediante una sintaxis sencilla y eficaz. La implementación de una máquina virtual encargada de llevar a cabo dicha simulación asegura total transparencia para el usuario en todas las fases de la traducción y en la ejecución posterior del programa.

De esta manera, se pone a disposición de cualquier persona interesada en aprender o enseñar los conceptos de los sistemas concurrentes y distribuidos una herramienta compacta, poderosa y altamente expresiva.

\section{Objetivos del trabajo}
Con lo anteriormente mencionado, los objetivos a cubrir en este proyecto son los siguientes:

\begin{itemize}
    \item Establecer un marco teórico en materia de conceptos de sistemas concurrentes, así como mecanismos de verificación formal de sistemas a través de la Lógica Temporal de Acciones (TLA).
    \item Llevar a cabo el desarrollo de un proyecto de software desde sus inicios utilizando metodologías de desarrollo ágil, poniendo el foco siempre en los clientes, que son las personas interesadas en el producto y quienes verdaderamente guían a un equipo hacia su progreso.
    \item Realizar un análisis y descripción detallada del lenguaje de programación Lamport, tomando como punto de partida la sintaxis de los ejemplos de pseudocódigo de la asignatura ``Sistemas Concurrentes y Distribuidos'' del curso de Ingeniería Informática en la Universidad de Granada.
    \item Desarrollar un traductor del código descrito, teniendo en cuenta las fases de análisis y optimización. Posteriormente, modelar una máquina virtual capaz de interpretar directamente el código generado por el traductor, mostrando la ejecución resultante de las instrucciones al usuario, además de otorgarle transparencia en los sucesivos eventos que han ido ocurriendo por dentro, con el objetivo de garantizar un aprendizaje integral.
\end{itemize}