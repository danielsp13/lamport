%%%%%%%%%%%%%%%%%%%%%%%%%%%%%%%%%%%%%%%%%
% Short Sectioned Assignment LaTeX Template Version 1.0 (5/5/12)
% This template has been downloaded from: http://www.LaTeXTemplates.com
% Original author:  Frits Wenneker (http://www.howtotex.com)
% License: CC BY-NC-SA 3.0 (http://creativecommons.org/licenses/by-nc-sa/3.0/)
%%%%%%%%%%%%%%%%%%%%%%%%%%%%%%%%%%%%%%%%%

% \documentclass[paper=a4, fontsize=11pt]{scrartcl} % A4 paper and 11pt font size
\documentclass[11pt, a4paper]{book}
\usepackage[T1]{fontenc} % Use 8-bit encoding that has 256 glyphs
\usepackage[utf8]{inputenc}
%\usepackage{mathptmx} % Use the Adobe Utopia font for the document - comment this line to return to the LaTeX default
\usepackage{listings} % para insertar código con formato similar al editor
\usepackage{color}  % si quieres usar colores en tu código
\usepackage[spanish, es-tabla]{babel} % Selecciona el español para palabras introducidas automáticamente, p.ej. "septiembre" en la fecha y especifica que se use la palabra Tabla en vez de Cuadro
\usepackage{url} % ,href} %para incluir URLs e hipervínculos dentro del texto (aunque hay que instalar href)
\usepackage{graphics,graphicx, float} %para incluir imágenes y colocarlas
\usepackage[gen]{eurosym} %para incluir el símbolo del euro
\usepackage{cite} %para incluir citas del archivo <nombre>.bib
\usepackage{enumerate}
\usepackage{enumitem}
\usepackage{hyperref}
\usepackage{graphicx}
\usepackage{tabularx}
\usepackage{booktabs}
\usepackage{textcomp}
\usepackage{adjustbox}
\usepackage{geometry}
\usepackage{amsmath}
\usepackage{amsthm} % Añadir este paquete
\usepackage{amsfonts}
\usepackage{amssymb}
\usepackage{subfigure}
\usepackage{textcomp}
\usepackage{longtable}
\usepackage{mwe} % Paquete necesario para la columna tipo 'm'
\usepackage{mathrsfs}
\usepackage{upquote}
\usepackage{mdframed}


\lstdefinestyle{lamportStyle}{
    language=Pascal,          % Usar configuración base de Pascal
    basicstyle=\ttfamily\small,
    numbers=left,
    numberstyle=\tiny,
    stepnumber=1,
    numbersep=5pt,
    tabsize=4,
    extendedchars=true,
    breaklines=true,
    keywordstyle=\color{blue},
    stringstyle=\color{orange}, % Cadena de caracteres en naranja
    commentstyle=\color{gray},
    frame=b,
    morekeywords={program, procedure, function, process, integer, real, char, boolean, string, begin, end, cobegin, coend, return},
    captionpos=b             % Posiciona el caption debajo del bloque de código   
}


\lstnewenvironment{BNFCode}{%
\lstset{
  language={}, % Vacío, no basarse en ningún lenguaje específico
  basicstyle=\ttfamily\color{black}, % Fuente de estilo básico en color negro
  morestring=[b]", % Define las cadenas de texto entre comillas dobles
  stringstyle=\color{red}, % Color rojo para cadenas de texto
  commentstyle=\color{orange}, % Color naranja para comentarios
  morecomment=[l]{\#}, % Define cómo se indican los comentarios
  keepspaces=true, % Mantiene los espacios
  breaklines=true, % Permite romper las líneas
  breakindent=1.5cm, % Sangría después de romper una línea
  breakatwhitespace=false, % Permite romper en cualquier espacio, no solo en espacios en blanco
  columns=fullflexible,
}
}{}

\lstdefinestyle{customflex}{
    language=C, % Flex se basa en C, así que esto debería resaltar la mayoría de la sintaxis correctamente
    basicstyle=\ttfamily\small,
    commentstyle=\color{gray},
    keywordstyle=\color{blue},
    numbers=left,
    breaklines=true,
    frame=single,
    captionpos=b
    \lstset{upquote=true}
}

\lstdefinestyle{myInlineCode}{
    basicstyle=\ttfamily,
    breaklines=true,  % Opcional: permite que el código se divida entre líneas
    keepspaces=true   % Opcional: conserva los espacios en el código
}


\usepackage[table,xcdraw]{xcolor}
\hypersetup{
	colorlinks=true,	% false: boxed links; true: colored links
	linkcolor=black,	% color of internal links
	urlcolor=cyan		% color of external links
}
%\renewcommand{\familydefault}{\sfdefault}
\usepackage{fancyhdr} % Custom headers and footers
\pagestyle{fancyplain} % Makes all pages in the document conform to the custom headers and footers
\fancyhead[L]{} % Empty left header
\fancyhead[C]{} % Empty center header
\fancyhead[R]{Daniel Pérez Ruiz} % My name
\fancyfoot[L]{} % Empty left footer
\fancyfoot[C]{} % Empty center footer
\fancyfoot[R]{\thepage} % Page numbering for right footer
%\renewcommand{\headrulewidth}{0pt} % Remove header underlines
\renewcommand{\footrulewidth}{0pt} % Remove footer underlines
\setlength{\headheight}{13.6pt} % Customize the height of the header

\usepackage{titlesec, blindtext, color}
\definecolor{gray75}{gray}{0.75}
\newcommand{\hsp}{\hspace{20pt}}
\titleformat{\chapter}[hang]{\Huge\bfseries}{\thechapter\hsp\textcolor{gray75}{|}\hsp}{0pt}{\Huge\bfseries}
\setcounter{secnumdepth}{4}
\usepackage[Lenny]{fncychap}


\newcolumntype{M}[1]{>{\centering\arraybackslash}m{#1}}

\renewcommand{\lstlistingname}{Programa}
\newcommand{\code}[1]{\lstinline[style=myInlineCode]!#1!}

\newtheorem{teorema}{Teorema}[section]
\newtheorem{definicion}[teorema]{Definición}
\newtheorem{corolario}[teorema]{Corolario}
\newtheorem{proposicion}[teorema]{Proposición}
\newtheorem{observacion}[teorema]{Observación}

\newcommand{\squareDiamond}{%
\tikz [x=1.2ex,y=1.2ex,line width=.1ex] \draw (0,0) -- (1,1) -- (0,2) -- (-1,1) -- cycle;}

\begin{document}

	% Plantilla portada UGR
	\begin{titlepage}
\newlength{\centeroffset}
\setlength{\centeroffset}{-0.5\oddsidemargin}
\addtolength{\centeroffset}{0.5\evensidemargin}
\thispagestyle{empty}

\noindent\hspace*{\centeroffset}\begin{minipage}{\textwidth}

\centering
\includegraphics[width=0.9\textwidth]{logos/logo_ugr.jpg}\\[1.4cm]

\textsc{ \Large TRABAJO FIN DE GRADO\\[0.2cm]}
\small \textsc{DOBLE GRADO EN INGENIERÍA INFORMÁTICA Y MATEMÁTICAS}\\[1cm]

{\huge\bfseries Simulador de Sistemas Concurrentes y Distribuidos. Lógica Temporal de Acciones (TLA) \\}
\noindent\rule[-1ex]{\textwidth}{3pt}\\[3.5ex]
\end{minipage}

\vspace{0.5cm}
\noindent\hspace*{\centeroffset}
\begin{minipage}{\textwidth}
\centering

\textbf{Autor}\\ {Daniel Pérez Ruiz}\\[2.5ex]
\textbf{Director}\\ {Carlos Ureña Almagro}\\[1.2cm]

\begin{figure}[H]
  \centering
  \includegraphics[width=.3\textwidth]{logos/etsiit_logo.png}\hspace{0.2\textwidth}
  \includegraphics[width=.17\textwidth]{logos/ciencias-logo.png}
\end{figure}
%\includegraphics[width=0.3\textwidth]{logos/etsiit_logo.png}\\[0.1cm]
\textsc{Escuela Técnica Superior de Ingenierías Informática y de Telecomunicación}\\[0.1cm]
\textsc{Y}\\
\textsc{Facultad de Ciencias}\\
\textsc{---}\\
Granada, Noviembre de 2023
\end{minipage}
\end{titlepage}


	% Plantilla prefacio UGR
	\thispagestyle{empty}

\begin{center}
{\large\bfseries Simulador de Sistemas Concurrentes y Distribuidos. Lógica Temporal de Acciones }\\
\end{center}
\begin{center}
Daniel Pérez Ruiz\\
\end{center}

%\vspace{0.7cm}

\vspace{0.5cm}
\noindent\textbf{Palabras clave}: \textit{software libre, ágil, concurrencia, sistema, programa, lógica, tiempo, instrucción, lenguaje, compilador, máquina virtual}
\vspace{0.7cm}

\noindent\textbf{Resumen}\\
Este trabajo se centra en el desarrollo de un lenguaje de programación y un compilador en tiempo de ejecución para el mismo, con el objetivo de facilitar el aprendizaje y la enseñanza de los conceptos y el diseño de sistemas concurrentes y distribuidos. Utilizando metodologías de desarrollo ágil y poniendo el foco siempre en los posibles usuarios del proyecto, se consigue una mejora continua y una respuesta eficaz a los desafíos emergentes durante el desarrollo. El lenguaje, creado con una sintaxis simplificada y clara, evita la necesidad de bibliotecas externas o dependencias adicionales, diferenciándolo de lenguajes de programación de alto nivel convencionales como C++.

En el desarrollo de este software, se ha hecho uso de avanzadas herramientas de análisis sintáctico y léxico, lo que ha permitido una implementación más precisa y eficiente del compilador. Estas herramientas han sido fundamentales para construir un sistema robusto que proporciona al usuario información detallada como el Árbol de Sintaxis Abstracta (AST), la Representación Intermedia (IR), y la traza de ejecución de instrucciones dentro de la Máquina Virtual implementada.

Como complemento al desarrollo del compilador y del lenguaje, se ha incluido una exploración de los fundamentos matemáticos detrás de verificación de sistemas concurrentes a través de sistemas lógicos formales, concretamente la Lógica Temporal de Acciones definida por Leslie Lamport con tal fin. Esto ofrece una visión adicional que enriquece el proceso de aprendizaje. La comprensión de la verificación es esencial para garantizar la fiabilidad y eficacia de los sistemas desarrollados, y su inclusión apoya la formación integral de los usuarios en la teoría y práctica de estos sistemas.

La creciente complejidad de los sistemas informáticos modernos y la omnipresencia de aplicaciones y servicios basados en la nube subrayan la importancia del entendimiento de los sistemas concurrentes y distribuidos. En un mundo donde la eficiencia y la capacidad de manejar múltiples tareas simultáneamente son cruciales, la programación concurrente se convierte en una habilidad indispensable. Sin embargo, estos sistemas a menudo presentan desafíos significativos debido a su naturaleza compleja. Proporcionar un medio para una comprensión clara y práctica de estos sistemas, como lo hace la herramienta desarrollada, es vital para la formación de futuros profesionales y el impulso de soluciones tecnológicas avanzadas.

\cleardoublepage

\begin{center}
	{\large\bfseries Concurrent and Distributed Systems Simulator. Temporal Logic of Actions}\\
\end{center}
\begin{center}
	Daniel Pérez Ruiz\\
\end{center}
\vspace{0.5cm}
\noindent\textbf{Keywords}: \textit{free software, agile, concurrency, system, program, logic, time, instruction, language, compiler, virtual machine}
\vspace{0.7cm}

\noindent\textbf{Abstract}\\
This work focuses on the development of a programming language and a runtime compiler for it, aiming to facilitate the learning and teaching of concepts and design of concurrent and distributed systems. By using agile development methodologies and always focusing on potential users of the project, continuous improvement and an effective response to emerging challenges during development are achieved. The language, created with simplified and clear syntax, avoids the need for external libraries or additional dependencies, distinguishing it from conventional high-level programming languages like C++.

In the development of this software, advanced syntactic and lexical analysis tools have been used, enabling a more precise and efficient implementation of the compiler. These tools have been fundamental in building a robust system that provides users with detailed information such as the Abstract Syntax Tree (AST), Intermediate Representation (IR), and the execution trace of instructions within the implemented Virtual Machine.

As a complement to the development of the compiler and language, an exploration of the mathematical foundations behind the verification of concurrent systems through formal logical systems, specifically the Temporal Logic of Actions defined by Leslie Lamport for this purpose, has been included. This offers an additional perspective that enriches the learning process. Understanding verification is essential to ensure the reliability and effectiveness of the developed systems, and its inclusion supports the comprehensive training of users in the theory and practice of these systems.

The increasing complexity of modern computer systems and the ubiquity of cloud-based applications and services underscore the importance of understanding concurrent and distributed systems. In a world where efficiency and the ability to handle multiple tasks simultaneously are crucial, concurrent programming becomes an indispensable skill. However, these systems often present significant challenges due to their complex nature. Providing a means for a clear and practical understanding of these systems, as the developed tool does, is vital for the training of future professionals and the advancement of sophisticated technological solutions.



\cleardoublepage

\thispagestyle{empty}

\noindent\rule[-1ex]{\textwidth}{2pt}\\[4.5ex]

D. \textbf{Tutora/e(s)}, Profesor(a) del ...

\vspace{0.5cm}

\textbf{Informo:}

\vspace{0.5cm}

Que el presente trabajo, titulado \textit{\textbf{Simulador de Sistemas Concurrentes y Distribuidos. Lógica Temporal de Acciones}},
ha sido realizado bajo mi supervisión por \textbf{Daniel Pérez Ruiz}, y autorizo la defensa de dicho trabajo ante el tribunal
que corresponda.

\vspace{0.5cm}

Y para que conste, expiden y firman el presente informe en Granada a Noviembre de 2023.

\vspace{1cm}

\textbf{El/la director(a)/es: }

\vspace{5cm}

\noindent \textbf{(nombre completo tutor/a/es)}

\chapter*{Agradecimientos}

Desde que era muy pequeño mi madre siempre me ha dicho que, si ves una mariposa blanca, eso es señal de suerte. Aunque no me caracterizo por ser una persona supersticiosa, siempre he convivido con esa inocencia cada vez que tenía ante mis ojos una de ellas, aleteando, al son de la libertad, iluminando en mí una sonrisa.

\vspace{0.2cm}

A lo largo de mi vida he tenido muchas ocasiones donde sentía que me habían abandonado. Y la realidad, es que siempre me han acompañado, aunque muchas veces no las viera. Ahí se encontraban, danzando al compás de todas las melodías y letras, de todas aquellas canciones que en tiempos pasados resonaron en mi cabeza.

\vspace{0.2cm} 

A mi madre, a mi hermana Irene, a mi padre. A mis amigos de toda una vida Adrián, Alberto, Elías, Juanmi, Juan, Rafa Aguilar y Rafa Barrales. A mis amigos de esta nueva vida Lucía, Martín, Mery, Jaime, Jose, Joseja y Pablo. No hay suficientes palabras en este mundo para describir lo inmensos que son. A mi amiga Candela, por haber por fin coincidido en este camino tras muchas historias y canciones, y que prometo recorrerlo bailando una vez más. A mi amiga Elena, por haber sido una de las voces que me guió y me dio luz cuando no la había. A mi amiga Coral, por haber venido casi al final con su deslumbrante forma de ser, brindándome un nuevo y prometedor comienzo. A mis dos grandes mentores: mi tutor, Carlos, por ser mi fuente de inspiración desde que coincidí con él; y JJ, la persona que me enseñó a cómo encontrarme si me perdía en el camino.

\vspace{0.2cm}
En este trabajo dejo atrás una muy buena parte de mi vida, donde cumplí el sueño de la infancia que me llevó hasta este mismo momento, el de descubrir y aprender todo lo \textit{immenso} que es el mundo de la informática, así como también lo es la vida misma. Esta parte de mí dejará de existir para siempre, pero ahora es mi turno de volar, al igual que todas las mariposas blancas que siempre estuvieron para mí.

\newpage
\vspace*{\fill}

\large
\noindent
$P := \textit{``Estás dispuesto a arriesgar.''}$
\newline
$Q := \textit{``Puedes crecer.''}$
\newline
$R := \textit{``Puedes dar lo mejor de ti.''}$
\newline
$S := \textit{``Puedes ser feliz.''}$
\newline
\newline

\noindent
\textit{Si} $\neg P \implies \neg Q$;
\newline
\textit{Si} $\neg Q \implies \neg R$;
\newline
\textit{Si} $\neg R \implies \neg S$;
\newline
\textit{Y si} $\neg S$, ¿qué te queda?
\newline



\vspace*{\fill}

	% Índice de contenidos
	\newpage
	\tableofcontents

	% Índice de imágenes y tablas
	\newpage
	\listoffigures

	% Si hay suficientes se incluirá dicho índice
	\listoftables 
	\newpage

	% Introducción 
	\chapter{\textbf{Introducción}}

\section{Motivación}
El mundo y la tecnología avanzan a pasos agigantados. El lapso de tiempo entre un hito importante en la historia de la informática y otro inmediatamente anterior es cada vez más corto. Hace años quedaron atrás los gigantes armarios llenos de circuitería que procesaban bandas magnéticas con cálculos muy específicos. Ahora, en un simple grano de arroz podrían existir múltiples unidades de cálculo. Es una evolución normal, consecuencia de las necesidades de este siglo, que también crecen. En un mundo globalizado y universal que te permite viajar hasta las mismas antípodas, la cooperación entre países, y seres humanos, es de vital importancia. Los programas, con sus singulares algoritmos y propósitos, también se vieron obligados a cooperar. Y es que precisamente los programas dejaron de ser \textit{singulares}, para ser \textit{plurales}. Ahora son como aquellos hermanos que comparten todo, datos, memoria, procesador, pero no necesariamente siempre van a poder hacerlo al mismo tiempo. Esto es lo que se conoce como un sistema concurrente, donde múltiples unidades ejecutan instrucciones lógicamente al mismo tiempo. Cuando se dio luz a este concepto, a mediados del siglo pasado, el mundo dejó de ser el mismo. La velocidad de procesamiento de datos frutos de la investigación científica aumentó considerablemente al poder desarrollar más tareas en menos tiempo, el desarrollo de nuevas máquinas por parte de los gigantes tecnológicos de aquella época dio un giro importante, haciendo más accesible la informática al olvidarse de su carácter elitista, bajando hacia lo más mundano y cotidiano, no sólo visto a través de los ordenadores de sobremesa y portátiles, sino también la forma de interacción humana, generando un alter ego digital. 

En la actualidad, internet reina con soberanía. Los videojuegos, los servicios de \textit{streaming} bajo demanda, y la inquietud constante del ser humano de aumentar su desarrollo como persona en múltiples campos como la medicina, seguridad, economía, defensa, justicia, e igualdad, hacen que la concurrencia siga siendo un concepto no sólo importante, sino necesario. Esta necesidad de concurrencia se extiende aún más en el ámbito de los sistemas distribuidos. La creciente demanda de aplicaciones y servicios que operan en tiempo real y a gran escala, como las redes sociales, el comercio electrónico y las plataformas de análisis de datos, subraya la importancia de sistemas que no solo procesen tareas en paralelo, sino que también compartan y sincronicen datos a través de múltiples nodos y geografías.

El desafío de diseñar y mantener estos sistemas complejos es enorme. No solo deben ser eficientes y rápidos, sino también robustos y seguros. La concurrencia y la distribución presentan desafíos únicos, especialmente en la gestión de recursos compartidos, la prevención de condiciones de carrera, y la garantía de la coherencia de los datos. Además, la creciente preocupación por la privacidad y seguridad de los datos en aplicaciones distribuidas agrega una capa adicional de complejidad al diseño de estos sistemas.

Este trabajo introduce al lector en las nociones básicas de los sistemas concurrentes, incorporando además un formalismo matemático para la verificación de las propiedades de estos sistemas. A continuación, se presenta el elemento central del proyecto: la creación de un lenguaje de programación y su correspondiente intérprete. El objetivo es diseñar y simular programas que requieran la creación y sincronización de múltiples procesos, eliminando la necesidad de dominar un lenguaje de programación de alto nivel convencional o de utilizar bibliotecas externas complejas. Esto se logra mediante una sintaxis sencilla y eficaz. La implementación de una máquina virtual encargada de llevar a cabo dicha simulación asegura total transparencia para el usuario en todas las fases de la traducción y en la ejecución posterior del programa.

De esta manera, se pone a disposición de cualquier persona interesada en aprender o enseñar los conceptos de los sistemas concurrentes y distribuidos una herramienta compacta, poderosa y altamente expresiva.

\section{Objetivos del trabajo}
Con lo anteriormente mencionado, los objetivos a cubrir en este proyecto son los siguientes:

\begin{itemize}
    \item Establecer un marco teórico en materia de conceptos de sistemas concurrentes, así como mecanismos de verificación formal de sistemas a través de la Lógica Temporal de Acciones (TLA).
    \item Llevar a cabo el desarrollo de un proyecto de software desde sus inicios utilizando metodologías de desarrollo ágil, poniendo el foco siempre en los clientes, que son las personas interesadas en el producto y quienes verdaderamente guían a un equipo hacia su progreso.
    \item Realizar un análisis y descripción detallada del lenguaje de programación Lamport, tomando como punto de partida la sintaxis de los ejemplos de pseudocódigo de la asignatura ``Sistemas Concurrentes y Distribuidos'' del curso de Ingeniería Informática en la Universidad de Granada.
    \item Desarrollar un traductor del código descrito, teniendo en cuenta las fases de análisis y optimización. Posteriormente, modelar una máquina virtual capaz de interpretar directamente el código generado por el traductor, mostrando la ejecución resultante de las instrucciones al usuario, además de otorgarle transparencia en los sucesivos eventos que han ido ocurriendo por dentro, con el objetivo de garantizar un aprendizaje integral.
\end{itemize}

	% Descripción del problema y hasta donde se llega
	\chapter{\textbf{Descripción del problema}}\label{chapter:problema}

\textit{"La inteligencia consiste no sólo en el conocimiento, sino también en la destreza de aplicarlos en la práctica"}, sostenía el célebre filósofo Aristóteles hace más de dos milenios. Aunque han pasado muchos años desde entonces, su reflexión mantiene un profundo significado que podría ser clave en el día a día para aspirar a ser mejores versiones de nosotros mismos. Este pensamiento es el hilo conductor del presente proyecto. Todo el trabajo realizado aquí busca responder a las dos siguientes preguntas: ¿cómo se puede optimizar, a nivel teórico y práctico, el estudio e implementación de los conceptos de diseño y desarrollo de programas concurrentes? ¿Qué métodos existen para verificar formalmente las propiedades de dichos programas? Aunque existen innumerables maneras de abordar estos desafíos planteados, sólo el uso de una metodología clara, segura y correcta resulta verdaderamente adecuado.

\section{Metodología: Desarrollo Ágil}
La siguiente frase de Siegbert Tarrasch, quien fue uno de los mejores jugadores de ajedrez de todos los tiempos, también es digna de mención: \textit{"La belleza de un movimiento no se refleja sólo en su apariencia, sino en el pensamiento detrás de él"}. No basta con tener acciones o hechos; es esencial que detrás de ellos exista una \textbf{idea}, un \textbf{problema} o simplemente una \textbf{pregunta} que sirva de base para construir paso a paso todos los objetivos que se deseen alcanzar. Del mismo modo que en el ajedrez cada movimiento es estratégico y sigue una lógica o un plan, en un proyecto que precise de la ingeniería informática cada paso dado debe estar fundamentado y orientado hacia la resolución del problema central.

En un proyecto de ingeniería informática, es esencial identificar claramente el problema que se desea resolver y la razón subyacente. Una vez definido, es importante garantizar que el proyecto en cuestión realmente aborde dicho problema \cite{jj-agile-objetivos}. La estrategia más efectiva consiste en dividir el problema en segmentos más manejables, a los que podemos referirnos como \textit{objetivos}. La recurrente mención de la palabra \textbf{"problema"} subraya su importancia central en la metodología, pues no se espera otra cosa que \textit{solucionar dicho problema}.

El desarrollo ágil surgió tras la redacción y firma del \textit{Manifiesto por el Desarrollo Ágil de Software} \cite{agile-manifest} por diecisiete expertos en programación. Con el término \textit{ágil} no se alude únicamente a una metodología para el desarrollo de proyectos que precisan de rapidez y flexibilidad, sino también una filosofía que implica una forma distinta de trabajar y de organizarse a la que predominaba anteriormente, denominada \textit{metodología en cascada}. Así pues, por \textit{ágil} se entiende una mentalidad que se aplica a todo el ciclo de vida del desarrollo de software, centrada en el cliente y en la mejora continua de productos mínimamente viables cada vez más complejos \cite{jj-agile-manifesto}.

Todo proyecto debe nacer de una motivación inicial, respondiendo a interrogantes como \textit{``por qué''}, \textit{``para qué''}, \textit{``para quién''} y \textit{``cómo''}. Si bien todas estas cuestiones son importantes, la penúltima destaca particularmente porque el éxito radica en satisfacer los \textit{deseos} y \textit{necesidades} de un grupo específico de clientes o usuarios unidos por una característica común: un \textit{problema}. 
Resolverlo implica practicar la \textbf{empatía} con ellos, esforzándose por comprender profundamente sus necesidades y determinar cómo satisfacerlas de manera óptima, aportando \textbf{valor} con los recursos disponibles. Las entrevistas personales o el seguimiento de las tendencias actuales pueden proporcionar la perspectiva adecuada.

De ahí surgen las \textbf{Historias de Usuario}, que proporcionan una explicación informal desde el punto de vista del usuario final y siempre situadas dentro del dominio del problema, de una funcionalidad del software que principalmente tendrá que ver con la lógica de negocio del proyecto.\cite{jj-design-thinking}. Una vez definidas, lo que queda es especificar los productos que se entregarán a los clientes, descritos a través de una secuencia de \textbf{hitos} o \textbf{milestones}. La esencia del desarrollo ágil radica en efectuar mejoras iterativas sobre el producto, contando siempre con la aprobación del usuario. En consecuencia, el avance del proyecto no es lineal, sino un proceso incremental.

En resumen, el desarrollo ágil supuso un gran cambio en el paradigma de la organización y planificación de proyectos de ingeniería informática, colocando al usuario y sus necesidades en el centro del proceso, guiando cada paso a través de la empatía y una comprensión profunda del problema. Con herramientas como las Historias de Usuario y la propia naturaleza de la metodología basada en un proceso iterativo e incremental, se busca proporcionar soluciones más adaptadas y flexibles de la ingeniería moderna, puesto que las necesidades de los clientes pueden ir variando con el tiempo. Es un enfoque que valora la colaboración y la resiliencia, con el objetivo de siempre aportar valor, abordando así los actuales y futuros desafíos del mundo tecnológico.

\section{Clientes}
Tras haber explorado la esencia del desarrollo ágil y su prioridad hacia el usuario y sus deseos, ahora se profundizará en el concepto de \textbf{cliente}. Los problemas y expectativas de los clientes o usuarios son lo que impulsan las decisiones y acciones del equipo de desarrollo. 

Para identificar y comprender los distintos usuarios del compilador de código, se empleó una metodología centrada en las personas. Se llevaron a cabo entrevistas individuales a dos grupos de personas \textit{reales} \footnote{Aunque no es estrictamente necesario utilizar personas reales para el análisis de un problema en el desarrollo de un proyecto, hacerlo añade una dimensión \textit{humana} que, en mi experiencia con otros proyectos, aumenta considerablemente las probabilidades de éxito.} vinculadas a la asignatura de \textit{Sistemas Concurrentes y Distribuidos}: estudiantes y profesorado. A ambos grupos se les formuló una serie de preguntas, que se presentan a continuación.

\noindent
Las preguntas planteadas al alumnado de la asignatura fueron:
\begin{itemize}
    \item \textit{¿Te resulta complicado entender los conceptos o fundamentos teóricos de la asignatura?}
    \item \textit{¿Eres capaz de resolver un problema que involucre concurrencia sin programarlo explícitamente?}
    \item \textit{¿Notas una diferencia significativa entre la teoría y práctica de la asignatura?}
    \item \textit{¿Crees que programar en el pseudocódigo específico de la asignatura te facilitaría su aprendizaje?}
\end{itemize}

\noindent
Las preguntas planteadas al profesorado fueron:
\begin{itemize}
    \item \textit{¿Consideras que la asignatura es difícil para los estudiantes?}
    \item \textit{¿Ves una diferencia significativa entre la teoría y la práctica de la asignatura?}
    \item \textit{¿Piensas que programar en el pseudocódigo específico de la asignatura ayudaría a los alumnos durante el curso? ¿Facilitaría tu labor docente al explicar los conceptos?}
\end{itemize}

Las dos últimas preguntas de cada grupo son similares, buscando comprobar la concordancia entre ambos puntos de vista. Las respuestas de los estudiantes indican que, aunque la asignatura posee una complejidad relativa en comparación con otras del grado, los conceptos en sí no son difíciles de asimilar. La verdadera barrera surge al aplicar estos conceptos en el diseño de sistemas concurrentes o al tratar de resolver problemas sin recurrir a un lenguaje de alto nivel, como C++. Los estudiantes generalmente consideran que se desenvuelven mejor en la práctica que en la teoría. Esta preferencia puede deberse a su familiaridad con el enfoque práctico adoptado en los primeros años del grado, con este lenguaje en particular. Por tanto, la respuesta a la última pregunta suele ser un \textit{sí} rotundo.

El profesorado, por su parte, no ve a su asignatura como particularmente complicada y no percibe una gran diferencia entre la teoría y práctica. Sin embargo, muestran empatía hacia los estudiantes, entendiendo que ellos puedan sentir una mayor complejidad. Coinciden en que disponer de un lenguaje de programación con la sintaxis propuesta en la asignatura podría facilitar una enseñanza más didáctica y accesible.

Con esta información no sólo obtenemos la motivación mencionada en la sección anterior, sino que también podemos identificar a los dos usuarios potenciales del compilador a desarrollar. A continuación, se describirá detalladamente el perfil de cada tipo de usuario.

\subsection{Tipo de usuario 1: Estudiante}
Se proporciona una descripción detallada de un perfil de usuario de tipo estudiante:

\begin{description}
    \item[Nombre:] Luis Martínez
    \item[Características Demográficas:] \hfill
        \begin{itemize}
            \item Edad: 19 años.
            \item Estudiante universitario de ingeniería informática, actualmente cursando la asignatura de \textit{Sistemas Concurrentes y Distribuidos}.
        \end{itemize}
    \item[Necesidades y Objetivos:] Luis aspira a aprobar la asignatura de \textit{Sistemas Concurrentes y Distribuidos} para avanzar en sus estudios. Además, busca comprender conceptos que sean importantes en futuras asignaturas relacionadas con su grado.
    
    \item[Habilidades Técnicas:] Posee habilidades de programación de nivel principiante a intermedio. Es probable que esta sea su primera experiencia con la implementación de programas no secuenciales.
    
    \item[Escenarios de Uso Comunes:] Luis utiliza el compilador para validar ejercicios específicos de la clase sobre sincronización de hebras o procesos.
    
    \item[Limitaciones:] Aunque Luis se siente cómodo programando en lenguajes de alto nivel como C++, enfrenta dificultades al comprender la sintaxis del pseudocódigo. Esto puede dificultar su capacidad para traducir rápidamente los conceptos teóricos en implementaciones prácticas utilizando pseudocódigo.
    
    \item[Expectativas:] Espera poder disponer de un lenguaje de pseudocódigo funcional, pudiendo añadir comentarios explicativos a cada línea de código para facilitar su comprensión. Además, desea que las ejecuciones de los programas se visualicen de forma clara, permitiéndole seguir el proceso paso a paso para consolidar su entendimiento de los conceptos teóricos.
\end{description}

\subsection{Tipo de usuario 2: Profesor}
Se proporciona una descripción detallada de un posible perfil de usuario de tipo profesor:

\begin{description}
    \item[Nombre:] Laura Ruiz
    \item[Características Demográficas:] \hfill
        \begin{itemize}
            \item Edad: 42 años.
            \item Profesora titular de la asignatura \textit{Sistemas Concurrentes y Distribuidos} en la facultad de Ingeniería Informática.
            \item Más de 10 años de experiencia docente en el campo de la informática.
        \end{itemize}
    \item[Necesidades y Objetivos:] Desea que sus estudiantes comprendan a fondo los conceptos y aplicaciones de los sistemas concurrentes y distribuidos. Busca herramientas y métodos que puedan hacer que la enseñanza sea más interactiva y efectiva.
    
    \item[Habilidades Técnicas:] Amplios conocimientos en programación, sistemas concurrentes, y pedagogía. Familiarizada con varios lenguajes de programación, incluido C++.
    
    \item[Escenarios de Uso Comunes:] Utilizar el compilador para demostrar ejemplos en clase, proponer ejercicios prácticos a los estudiantes y evaluar soluciones propuestas por ellos. Puede usarlo también para simular escenarios concurrentes y mostrar visualmente a los estudiantes cómo funcionan.
    
    \item[Limitaciones:] Prefiere que la herramienta tenga una interfaz amigable y sea intuitiva, ya que no desea invertir mucho tiempo en aprender a usarla. 
    
    \item[Expectativas:] Espera que el compilador permita explicaciones paso a paso y que pueda integrarse fácilmente con otros recursos didácticos. Le interesa que el pseudocódigo esté alineado con el contenido teórico de la asignatura, facilitando la transición entre teoría y práctica.
\end{description}

\newpage

\section{Historias de Usuario}

A través de las \textbf{Historias de Usuario}, se busca no solo definir qué es lo que el software debe hacer, sino también por qué es relevante hacerlo y cuál es el valor que se ofrece al usuario final. A continuación, se presentarán las Historias de Usuario identificadas para el desarrollo del compilador de código y cómo estas sientan las bases para los \textit{milestones} y Productos Mínimamente Viables del proyecto.

\begin{itemize}
    \item \textbf{Historia de usuario 1 (HU1):} Como estudiante de la asignatura, Luis quiere disponer de un compilador para escribir y ejecutar programas en el lenguaje de pseudocódigo propuesto en las transparencias, con el objetivo de practicar y verificar el funcionamiento de las soluciones a los ejercicios planteados.
    \item \textbf{Historia de usuario 2 (HU2):} Como profesora de la asignatura, Laura quiere contar con un compilador del lenguaje de las transparencias para poder explicar de forma más práctica los conceptos teóricos sin requerir un lenguaje de programación estándar como puede ser C++ o Java, además de poder revisar y analizar el código fuente desarrollado eventualmente por sus alumnos, y proporcionarles retroalimentación personalizada a los mismos.
\end{itemize}

\section{Milestones del proyecto: Productos Mínimamente Viables (PMV)}

Siguiendo la línea de las \textit{Historias de Usuario} y la importancia de establecer una comunicación efectiva con el usuario final, es el momento de definir productos entregables concretos que materialicen estas historias en el transcurso del desarrollo, que comúnmente se denomina \textbf{Productos Mínimamente Viables (PMV)}. Están diseñados no solo como representaciones tangibles del progreso, sino también como puntos de revisión donde se puede evaluar y adaptar el proyecto basándose en la retroalimentación aportada o por el propio cliente o por otro equipo de desarrollo. Los \textbf{milestones} o \textbf{hitos} sirven para secuenciar y organizar estos PMV en el ámbito del desarrollo ágil. A continuación, se enumeran los \textit{milestones} establecidos para este proyecto y cómo cada uno aporta al objetivo final del compilador de código.

\subsection{Milestone 0: Infraestructura del proyecto y definición del lenguaje.}
El objetivo de este milestone es definir adecuadamente el problema que sustenta el proyecto, realizar las configuraciones y elegir las herramientas más óptimas para su desarrollo, y finalmente, realizar una primera descripción del lenguaje. En resumen, se realizan estas tareas:
\newpage
\begin{itemize}
    \item \textbf{Definición del problema y organización del proyecto:}
    \begin{itemize}
        \item Realizar entrevistas personales a los posibles clientes interesados con las preguntas destinadas a la detección del problema.
        \item Formular el problema de interés relacionado con la asignatura en detalle utilizando la información recopilada de las entrevistas.
        \item Perfilar los usuarios y sus Historias de Usuario.
        \item Describir el resto de Milestones que siguen a este.
    \end{itemize}
    
    \item \textbf{Estudio y definición de la gramática del lenguaje de pseudocódigo:}
    \begin{itemize}
        \item Realizar un análisis de las componentes del lenguaje de pseudocódigo propuesto en la asignatura de \textit{Sistemas Concurrentes y Distribuidos} a partir de ejemplos que aparezcan en sus apuntes.
        \item Realizar una descripción de la sintaxis del lenguaje, utilizando la notación de Backus-Naur (BNF en inglés).
    \end{itemize}
    
    \item \textbf{Infraestructura del proyecto:}
    \begin{itemize}
        \item Elegir el lenguaje de programación utilizado para la implementación del compilador, y seleccionar un compilador adecuado para los ficheros fuente desarollados.
        \item Elegir un Sistema de Control de Versiones (VCS en inglés) para llevar un registro histórico de los cambios y facilitar la colaboración y configurarlo.
        \item Elegir una licencia adecuada para el proyecto, considerando las necesidades y derechos de los usuarios y desarrolladores.
        \item Elegir un Gestor de Dependencias para gestionar adecuadamente todas las dependencias del resto de herramientas del proyecto. Esto incluye la instalación, desinstalación y control de las versiones de estas dependencias.
        \item Elegir un Gestor de Tareas para la descripción y automatización de tareas repetitivas en el proyecto, como la compilación o la ejecución de tests.
        \item Elegir una herramienta de linting para asegurar un código limpio, legible y que sigue las mejores prácticas del lenguaje elegido.
        \item Elegir una herramienta de ejecución de tests, preferiblemente en paralelo para reducir el tiempo de obtención de resultados.
        \item Configurar tareas de instalación, desinstalación y comprobación de versiones de dependencias.
        \item Configurar tareas de comprobación de sintaxis de ficheros fuente para asegurar la calidad y coherencia del código.
    \end{itemize}
\end{itemize}

\subsection{Milestone 1: Implementación del Analizador Léxico.}
El objetivo de este milestone es construir y validar la fase inicial del proceso de interpretación: la conversión de una cadena de entrada en una serie de tokens identificables y procesables por las etapas posteriores del compilador.

\begin{itemize}
    \item \textbf{Desarrollo del Analizador Léxico:}
    \begin{itemize}
        \item Elegir una herramienta generadora de analizadores léxicos apropiada para el lenguaje de programación seleccionado y que cumpla con las necesidades del proyecto.
        \item Definir los tokens del lenguaje, incluyendo palabras reservadas, identificadores, operadores y otros símbolos, así como los patrones de reconocimiento asociados a cada uno.
        \item Implementar el analizador léxico utilizando la herramienta seleccionada, garantizando que pueda reconocer y clasificar adecuadamente cada token definido.
        \item Definir una tarea de generación/compilación del analizador léxico que facilite su construcción y posterior integración con el resto del sistema.
    \end{itemize}

    \item \textbf{Infraestructura del proyecto:}
    \begin{itemize}
        \item Definir tarea de limpieza de código objeto generado y de ficheros compilados.
    \end{itemize}

\end{itemize}

\subsection{Milestone 2: Implementación del Analizador Sintáctico.}
El objetivo de este milestone es diseñar e implementar el analizador sintáctico que será responsable de verificar la correcta estructura del código fuente escrito en el lenguaje de pseudocódigo, de acuerdo a la gramática previamente definida.

\begin{itemize}
    \item \textbf{Definición del lenguaje:}
    \begin{itemize}
        \item Definir operadores del lenguaje, su ariedad, asociatividad y su orden de precedencia.
    \end{itemize}
    
    \item \textbf{Desarrollo del Analizador Sintáctico:}
    \begin{itemize}
        \item Elegir una herramienta generadora de analizadores sintácticos apropiada.
        \item Validar y ajustar la gramática del lenguaje.
        \item Implementar el analizador sintáctico.
        \item Definir estrategia de recuperación ante errores sintácticos.
        \item Definir una tarea de generación/compilación del analizador sintáctico.
    \end{itemize}
    
    \item \textbf{Desarrollo del Árbol Sintáctico Abstracto (AST, en inglés):}
    \begin{itemize}
        \item Implementar estructuras y funciones para el AST.
        \item Definir tarea de compilación de API de gestión de AST.
        \item Integrar AST con el Analizador Sintáctico.
    \end{itemize}

    \item \textbf{Pruebas:}
    \begin{itemize}
        \item Realizar pruebas de integración del analizador sintáctico.
    \end{itemize}

    \item \textbf{Errores:}
    \begin{itemize}
        \item Implementar estructuras y funciones para gestión de errores sintácticos.
        \item Definir tarea de compilación de API de gestión de errores sintácticos.
        \item Integrar API de errores sintácticos con el Analizador Sintáctico.
    \end{itemize}

    \item \textbf{Desarrollo del compilador:}
    \begin{itemize}
        \item Implementar el programa principal (compilador).
        \item Habilitar lectura de ficheros de texto plano. Procesar los ficheros correctos escritos en el lenguaje de pseudocódigo y gestionar los errores en caso de argumentos incorrectos.
        \item Habilitar análisis sintáctico.
        \item Imprimir el AST generado si el análisis sintáctico fue superado con éxito o, en caso contrario, imprimir los errores sintácticos detectados para que el usuario pueda corregirlos.
        \item Definir tarea de compilación de compilador.
    \end{itemize}
\end{itemize}

\subsection{Milestone 3: Implementación del Analizador Semántico.}
El objetivo de este milestone es diseñar e implementar el analizador semántico que se encargará de verificar que el código fuente escrito en el lenguaje de pseudocódigo no solo tiene una estructura correcta (como verificó el analizador sintáctico) sino que también tiene un significado lógico y coherente, basándose en la semántica del lenguaje definido.

\begin{itemize}
    \item \textbf{Definición del lenguaje:}
    \begin{itemize}
        \item Descripción semántica del lenguaje de pseudocódigo.
    \end{itemize}
    
    \item \textbf{Desarrollo del Analizador Semántico:}
    \begin{itemize}
        \item Implementación de Tabla de Símbolos para gestionar la información sobre identificadores y sus propiedades en el código.
        \item Implementación del algoritmo de resolución de nombres para garantizar que cada identificador utilizado haya sido declarado y se use de manera coherente.
        \item Implementación del algoritmo de comprobación de tipos para verificar la correcta asignación y operación entre variables de diferentes tipos.
        \item Definir una tarea de compilación de la API del Analizador Semántico que facilite su integración con otras partes del proyecto.
        \item Integrar el analizador semántico con el resto de componentes del compilador, como el léxico y el sintáctico.
    \end{itemize}

    \item \textbf{Pruebas:}
    \begin{itemize}
        \item Realizar pruebas de integración del analizador semántico para garantizar su correcto funcionamiento ante diferentes situaciones y tipos de código.
    \end{itemize}

    \item \textbf{Errores:}
    \begin{itemize}
        \item Implementar estructuras y funciones para la gestión y reporte de errores semánticos.
        \item Definir una tarea de compilación de la API de gestión de errores semánticos.
        \item Integrar la API de errores semánticos con el Analizador Semántico para ofrecer retroalimentación específica al usuario.
    \end{itemize}

    \item \textbf{Desarrollo del compilador:}
    \begin{itemize}
        \item Habilitar la función de análisis semántico dentro del flujo del compilador.
        \item Mostrar un mensaje indicando si el análisis semántico fue superado con éxito o, en caso contrario, imprimir los errores semánticos detectados para que el usuario pueda corregirlos.
    \end{itemize}
\end{itemize}

\subsection{Milestone 4: Análisis e implementación de la fase de generación de código intermedio.}
El objetivo de este milestone es diseñar e implementar la fase de generación de código intermedio, que traduce el árbol de análisis sintáctico producido por las fases anteriores en una representación intermedia que es más cercana a código máquina,  que será interpretada por el simulador que se implemente en próximas fases.

\begin{itemize}
    \item \textbf{Definición del lenguaje:}
    \begin{itemize}
        \item Elección de una representación intermedia adecuada para el lenguaje desarrollado.
        \item Definición y descripción de las instrucciones de representación intermedia, especificando sus operandos y su uso.
    \end{itemize}
    
    \item \textbf{Desarrollo del Generador de Código Intermedio:}
    \begin{itemize}
        \item Definición de tablas encargadas de mapear variables, etiquetas y literales mediante direcciones.
        \item Definición e implementación del controlador encargado de la generación de código intermedio.
        \item Definición e implementación del controlador encargado de la optimización de código intermedio.
        \item Definir una tarea de compilación de la API del generador de código intermedio que facilite su integración con otras partes del proyecto.
        \item Integrar el generador de código intermedio con el resto de componentes del compilador.
    \end{itemize}

    \item \textbf{Desarrollo de manejador de registros de eventos:}
    \begin{itemize}
        \item Implementación del controlador encargado de imprimir en ficheros los resultados de las fases de análisis del compilador: análisis sintáctico, semántico, código intermedio, etc. El objetivo es ofrecer más transparencia al usuario y un mejor acceso al proceso de interpretación.
        \item Realizar registros de eventos de Analizador Sintáctico. En caso de análisis correcto, imprimir AST en un fichero. En otro caso, imprimir los errores sintácticos en un fichero.
        \item Realizar registros de eventos del Analizador Semántico. En caso de análisis fallido, imprimir los errores semánticos en un fichero.
        \item Realizar registros de eventos de la generación de código intermedio. Imprimir la secuencia de instrucciones de la representación intermedia generada.
    \end{itemize}

    \item \textbf{Pruebas:}
    \begin{itemize}
        \item Realizar pruebas de integración del módulo de generación de código intermedio.
    \end{itemize}

    \item \textbf{Desarrollo del compilador:}
    \begin{itemize}
        \item Habilitar la función de generación de código intermedio dentro del flujo del compilador.
        \item Habilitar la función de generación de ficheros de registro de eventos en las fases de análisis del compilador.
    \end{itemize}
\end{itemize}

\subsection{Milestone 5: Análisis e implementación del compilador de código intermedio o máquina virtual}
El objetivo de este milestone es diseñar e implementar un compilador para el código intermedio generado en las fases anteriores, actuando como una máquina virtual. Esta máquina virtual debe ser capaz de ejecutar la representación intermedia, proporcionando un entorno controlado que imita las operaciones de una máquina real, pero de manera abstracta e independiente de la arquitectura específica del hardware.

\begin{itemize}
    \item \textbf{Desarrollo de la Máquina Virtual:}
    \begin{itemize}
        \item Implementación de esquema de traducción de direcciones virtuales a direcciones físicas.
        \item Implementación de abstracción de bloque de memoria.
        \item Implementación de memoria de máquina virtual.
        \item Implementación de vector de registros de CPU.
        \item Implementación de CPU de máquina virtual.
        \item Implementación de SO de máquina virtual.
        \item Tratamiento de excepciones: ZeroDivision, IndexOutOfBounds.
    \end{itemize}
    \item \textbf{Pruebas:}
    \begin{itemize}
        \item Realizar pruebas de integración del módulo de máquina virtual.
    \end{itemize}
\end{itemize}

\subsection{Milestone 6: Estudio de la verificación formal de sistemas concurrentes. Lógica Temporal de Acciones}
El objetivo de este milestone es profundizar en el entendimiento y aplicación de técnicas de verificación formal específicas para sistemas concurrentes. Se busca estudiar cómo se pueden modelar, analizar y verificar sistemas que tienen múltiples entidades ejecutándose simultáneamente y que interactúan entre sí. Un enfoque particular será la Lógica Temporal de Acciones (LTA), que proporciona herramientas y métodos para razonar sobre el comportamiento de sistemas a lo largo del tiempo y en presencia de acciones concurrentes.

\section{Herramientas utilizadas}
Una vez se ha descrito el problema adecuadamente y se ha explicado cómo se solucionará y por qué, sólo queda definir \textit{con qué}. En esta sección se desglosan todas las herramientas utilizadas para el desarrollo de este proyecto:

\subsection{GitHub}

GitHub es una plataforma de alojamiento de código fuente basada en la herramienta de control de versiones Git, que es uno de los mayores repositorios de código del mundo, con millones de proyectos alojados en ella.



Más allá de ser simplemente un servicio de alojamiento, GitHub proporciona una serie de herramientas y funcionalidades que facilitan el trabajo colaborativo y la gestión de proyectos. Algunas de estas características incluyen la revisión de código, seguimiento de problemas \textbf{issues}, \textbf{integración continua}, o división del trabajo en \textbf{ramas}. Estas tres últimas características se usarán para los siguientes fines:

\begin{itemize}
    \item \textbf{Issues}: Se utilizarán para destacar problemas surgidos de la comprensión de las Historias de Usuario, o simplemente, tareas dentro del repositorio. Los issues pueden ser categorizados, etiquetados y asignados a miembros específicos del proyecto, permitiendo un seguimiento organizado de todas las tareas pendientes.
    
    \item \textbf{Integración continua}: Es una práctica de desarrollo que consiste en integrar automáticamente el código de distintas contribuciones en un proyecto. Con la integración continua, cada vez que se realiza un cambio en el código, este se prueba automáticamente, lo que permite detectar errores o incompatibilidades de forma temprana. GitHub ofrece herramientas y servicios, como GitHub Actions, que facilitan la implementación de la integración continua en los proyectos alojados en su plataforma.
    
    \item \textbf{Ramas (Pull Requests)}: Las ramas son versiones paralelas de un repositorio que permiten trabajar en diferentes características o pruebas sin afectar el código principal. Cuando un desarrollador quiere proponer cambios realizados en una rama al código principal (generalmente la rama ``master'' o ``main''), crea un Pull Request (PR). Este PR es una solicitud para revisar y eventualmente incorporar esos cambios en el código principal. Otros colaboradores pueden revisar, comentar y sugerir modificaciones en un PR antes de que sea fusionado con el código base.
\end{itemize}

\subsection{Lenguajes de programación: C y C++}
Puesto que el problema descrito se resuelve \textit{desarrollando código}, es natural decidir antes de empezar en qué lenguaje se va a dar solución a las historias de usuario mencionadas.


Los lenguajes C y C++ son elecciones prominentes cuando se busca eficiencia y rendimiento en un sistema. Estos lenguajes ofrecen un control cercano al hardware, permitiendo optimizaciones a nivel de memoria y ejecución. Además, la naturaleza compilada de sendos lenguajes asegura que el código se ejecuta directamente en la máquina anfitriona sin la necesidad de un intérprete intermedio, garantizando tiempos de respuesta rápidos. El compilador, al requerir análisis y ejecución eficiente del código fuente, se beneficia significativamente de estas características. Adicionalmente, la extensa biblioteca estándar y la amplia disponibilidad de bibliotecas de terceros en ambos lenguajes facilitan la implementación de funcionalidades complejas. Por estas razones, la decisión de utilizar C y C++ para definir el compilador garantiza un balance óptimo entre rendimiento y flexibilidad.

\subsection{Gestor de tareas y dependencias: Make}
Make es una herramienta de construcción automatizada que permite a los desarrolladores definir tareas y las dependencias entre ellas. Se utiliza ampliamente en programación para automatizar la compilación, pruebas, y otras tareas relacionadas con el ciclo de vida del software. Un archivo denominado `Makefile` contiene un conjunto de directivas y reglas que especifican cómo derivar los archivos objetivo a partir de archivos fuente. Al ejecutar la orden `make`, la herramienta lee el archivo `Makefile`, evalúa las dependencias y ejecuta las reglas necesarias en el orden adecuado.


\noindent
En este proyecto se usará para las siguientes tareas:
\begin{itemize}
    \item Gestión de las dependencias del proyecto. El Makefile debe contener reglas que permitan instalar, desinstalar y comprobar la versión de las dependencias del compilador.
    \item Compilación del compilador. Debe contener las reglas para generar código objeto de todos los módulos implementados y compilarlos en un único ejecutable binario final.
\end{itemize}

\subsection{Valgrind}
Valgrind es una herramienta de programación para la detección de errores en memoria y análisis de rendimiento. Permite a los desarrolladores identificar problemas relacionados con la gestión de memoria, como fugas de memoria y acceso a punteros no válidos, entre otros problemas comunes en C y C++. Al ejecutar programas bajo el control de Valgrind, se pueden detectar estos problemas en tiempo real, lo que facilita la identificación y corrección de errores en las etapas tempranas del desarrollo.

\subsection{Contenedores virtuales: Docker}
Docker es una plataforma de que permite a los desarrolladores empaquetar aplicaciones y sus dependencias en contenedores. Estos contenedores pueden ser ejecutados de manera consistente en cualquier entorno que tenga Docker instalado, independientemente de las diferencias en ese entorno con respecto al entorno original donde se desarrolló la aplicación. En generación de contenedores, a diferencia de la virtualización tradicional, no se crea una máquina virtual completa para cada aplicación, sino que comparte el mismo núcleo del sistema operativo y aisla la aplicación en un contenedor. Esto hace que Docker sea más ligero, más rápido y más eficiente en términos de recursos que las máquinas virtuales tradicionales.


En este proyecto se utilizará para aislar el compilador del lenguaje en un entorno donde ya disponga de todas las dependencias necesarias para funcionar, aprovechando todas las ventajas anteriormente mencionadas.



	% Estado del arte
	\chapter{\textbf{Estado del arte}}
La \textit{programación concurrente} sienta sus bases en la década de los 1960 con la aparición de los sistemas operativos multiprogramación, que pretendían resolver uno de los problemas más críticos que se daban en aquella época: el uso eficiente de los recursos hardware. Anteriormente, la CPU ejecutaba sólo un programa a la vez, quedando inactiva si se producían operaciones de entrada/salida subutilizando sus recursos, produciendo como resultado un rendimiento deficiente. Con la multiprogramación, varios programas podían residir simultáneamente en la memoria principal, permitiendo a la CPU ejecutar otro programa diferente mientras se gestionaba una interrupción de E/S. Esta gestión se facilitó gracias a la invención de los ``canales'', controladores de dispositivos que operaban de forma independiente. Esta revolucionaria técnica tuvo un gran impacto en algunos sistemas informáticos que se desarrollaban por aquel entonces, como es el caso de los ``mainframes'' \cite{TecnologiaInformaticaMainframe}.


La programación concurrente fue inicialmente motivo de preocupación de los diseñadores de sistemas operativos. Al final de esta década de 1960, los diseñadores de hardware desarrollaron máquinas multiprocesador, que supusieron todo un reto para la implementación de nuevos sistemas operativos adaptados a estos recursos, pero también definió una nueva oportunidad para que los desarrolladores de aplicaciones pudieran emerger y posicionarse en el mercado laboral como una profesión con grandes expectativas a futuro.



El primer gran reto de esta nueva técnica de programación fue resolver lo que se denomina como: \textit{el problema de la sección crítica} \cite{GeeksForGeeksCriticalSection}. Este desafío central se refiere al desarrollo de algoritmos efectivos para la sincronización de procesos concurrentes que requieren acceder a un recurso compartido. La importancia de resolver este problema radica en la necesidad de evitar conflictos y garantizar la coherencia de los datos cuando múltiples procesos intentan leer o modificar el mismo recurso simultáneamente. La correcta gestión de la sección crítica es crucial para asegurar que los sistemas concurrentes funcionen de manera fiable y eficiente. Para entender mejor y modelar este reto, se plantearon problemas teóricos como \textit{la cena de los filósofos} \cite{brosgol1996dining}, \textit{lectores y escritores} \cite{nithyasrikannathalreaders}, \textit{el barbero durmiente} \cite{SariSleepingBarber} o el \textit{problema de los fumadores} \cite{MTUSmokerProblem}.


Estos escenarios anteriormente mencionados han sido ampliamente estudiados, generando miles de artículos que proponen soluciones, mejoras, debates y nuevas implementaciones de primitivas de sincronización como los semáforos o los monitores, que simplificaban la tarea del programador. Paralelamente, los lenguajes de programación de alto nivel estaban proliferando y evolucionando. Un ejemplo notable es Simula \cite{Sklenar1997OOPSimula}, desarrollado en la década de 1960 por Ole-Johan Dahl y Kristen Nygaard. Considerado uno de los primeros lenguajes orientados a objetos, Simula también introdujo conceptos fundamentales que influirían en la simulación de concurrencia, aunque su principal aporte fue establecer las bases para la programación orientada a objetos, un paradigma que tendría un impacto significativo en el desarrollo futuro de lenguajes y técnicas de programación concurrente.


Al final de la década de 1970 y principios de 1980, el surgimiento de redes de ordenadores como ARPANET, que facilitó la computación en áreas amplias, y el desarrollo de tecnologías como Ethernet para redes locales, marcó el comienzo de una nueva era en la informática. Estos avances no solo transformaron la forma en que las computadoras se conectaban y comunicaban entre sí, sino que también ampliaron significativamente el campo de la programación concurrente. Mientras la programación concurrente se enfoca en gestionar múltiples procesos dentro de un mismo sistema, las redes de computadoras introdujeron el desafío de coordinar procesos que se ejecutan en diferentes máquinas físicas. Esto dio origen a lo que se conoce como \textit{programación distribuida}, un enfoque que extendía los principios de la programación concurrente a sistemas distribuidos cuya esencia reside en que los procesos interactúan entre ellos a través del envío de mensajes en vez de escribir y leer variables compartidas.


A medida que la programación concurrente y distribuida ganaban relevancia en la década de 1980 y 1990, su integración en los lenguajes de programación de alto nivel comenzó a materializarse de manera significativa. Un hito notable en este desarrollo fue el lenguaje de programación Ada \cite{burns1998concurrency}, introducido en 1983. Diseñado originalmente para satisfacer las exigentes necesidades de los proyectos de defensa y sistemas en tiempo real, Ada se destacó por su soporte nativo para la programación concurrente. Su enfoque en la seguridad, la fiabilidad y la gestión sofisticada de tareas concurrentes lo convirtió en una herramienta fundamental para aplicaciones críticas. Posteriormente, en la década de 1990, Java emergió como un lenguaje de programación influyente, llevando la programación concurrente a un público más amplio. Con su API de concurrencia, Java simplificó la gestión de hilos y procesos concurrentes, haciendo que la programación de este tipo fuera más accesible y manejable para los desarrolladores. Esta API proporcionó un conjunto robusto de herramientas y estructuras para el manejo eficiente de tareas concurrentes y sistemas distribuidos, reflejando la creciente demanda de aplicaciones que operaban en entornos de red y web. La evolución de la programación concurrente en Ada y Java no solo demuestra cómo se han adaptado los lenguajes de programación a desafíos técnicos complejos, sino también cómo han evolucionado para satisfacer las necesidades de aplicaciones modernas en diversos entornos de computación. Pero no solo los lenguajes en sí, sino también las herramientas para crearlos, evolucionaron significativamente durante este periodo. El desarrollo de herramientas avanzadas para generar compiladores facilitó la creación de lenguajes de programación más sofisticados, especialmente aquellos diseñados para la programación concurrente y distribuida. Estas herramientas permitieron a los diseñadores de lenguajes experimentar con nuevas construcciones y paradigmas, facilitando el desarrollo de lenguajes que pudieran manejar de manera más eficiente y segura las complejidades inherentes a la concurrencia y la distribución. 


Sin embargo, cada vez que se implementaban mejoras en la programación concurrente y distribuida, con nuevas ideas y hardware, la complejidad de los programas aumentaba exponencialmente. Esto llevó a un incremento similar en la dificultad de verificar su funcionamiento correcto. Poco después del origen de este modelo de programación, surgió la necesidad de introducir formalismos capaces de demostrar rigurosamente que un programa concurrente posee ciertas propiedades. En este contexto, las herramientas matemáticas formales, especialmente la lógica temporal, se convirtieron en indispensables. La lógica temporal, que se ocupa de los aspectos temporales del razonamiento en la lógica matemática, permitía especificar y razonar sobre el comportamiento de los programas a lo largo del tiempo, asegurando propiedades como la seguridad (nunca ocurrirá nada malo) y la vivacidad (realmente sucede algo bueno) en sistemas concurrentes.


Un desarrollo particularmente significativo en el campo de la verificación de sistemas fue la Lógica Temporal de Acciones (TLA) de Leslie Lamport, cuya vida y carrera han sido tan influyentes como sus logros académicos \footnote{Lamport fue también el desarrollador inicial de \LaTeX, el sistema de composición de documentos utilizado para redactar este informe. A lo largo de su carrera, ha sido galardonado con numerosos premios, incluyendo el prestigioso Premio Turing, por sus contribuciones a la informática teórica. Sus trabajos, incluyendo la TLA, han influenciado tanto la teoría como la práctica en el diseño y verificación de sistemas distribuidos y concurrentes.}. Nacido en 1941 en Nueva York, Lamport demostró un interés temprano en las matemáticas y la ciencia, lo que eventualmente lo llevó a obtener un Ph.D. en Matemáticas. Su transición hacia la computación fue motivada por un interés creciente en los problemas de sistemas distribuidos y la sincronización de procesos, un campo en el que se convertiría en un líder mundial.


Lamport, una figura destacada en la computación teórica, ha sido pionero en proponer metodologías rigurosas para el diseño y análisis de algoritmos en entornos complejos. La metodología de la TLA, que permite describir el comportamiento de un sistema en términos de estados y transiciones entre estos, facilita la verificación de propiedades críticas de sistemas complejos. Su visión era proporcionar una herramienta que no solo facilitara la especificación precisa de estos sistemas, sino que también permitiera su análisis y verificación de manera sistemática y confiable.


En la actualidad, la programación concurrente y distribuida se ha consolidado como un pilar fundamental en el panorama tecnológico. En el ámbito de la computación ubicua, donde múltiples dispositivos y sensores interactúan en tiempo real, es crucial mantener la precisión y la sincronización para ofrecer experiencias de usuario cohesivas y confiables. En el campo de la inteligencia artificial, la capacidad de manejar operaciones de procesamiento paralelo y distribuido resulta esencial para analizar grandes conjuntos de datos y ejecutar complejos algoritmos de aprendizaje automático. La corrección y eficiencia de los procesos concurrentes son vitales en áreas como el reconocimiento de patrones y el procesamiento de lenguaje natural. En el procesamiento multimedia, esta programación facilita la manipulación eficiente de contenidos de alta definición en tiempo real, siendo la precisión en la sincronización y la integridad de los datos fundamentales en aplicaciones que van desde la edición de vídeo hasta la transmisión en vivo. Estos ejemplos ilustran la omnipresencia y el impacto crucial de la programación concurrente y distribuida en la tecnología moderna, resaltando la necesidad de métodos rigurosos para su verificación fiable, lo que a su vez impulsa la innovación tecnológica.

        % Planificación
        \input{secciones/04_planificacion}

	% Análisis del problema (TLA)
	\chapter{\textbf{Verificación formal de sistemas concurrentes: Lógica Temporal de Acciones}}\label{chapter:tla}
La estrecha colaboración entre la informática y las matemáticas se evidencia especialmente en el ámbito de la verificación formal de sistemas concurrentes, cuya misión es asegurar que los sistemas que requieren de múltiples procesos ejecutándose simultáneamente sean correctos y fiables. La \textit{corrección}, en el contexto de los algoritmos concurrentes, implica la satisfacción de las propiedades deseadas del programa. Este capítulo se centra en la Lógica Temporal de Acciones (TLA), una creación de Leslie Lamport y que tiene como objetivo especificar y verificar las propiedades de estos sistemas mediante fórmulas.

La motivación detrás de la definición de TLA por parte de Lamport radica en su convicción de que utilizar un razonamiento riguroso es la única manera de prevenir errores graves en algoritmos concurrentes. Esta percepción lleva al planteamiento de dos preguntas: ¿Por qué optar por una lógica en lugar de un lenguaje de programación convencional? ¿No sería más sencillo trabajar directamente con programas en lugar de fórmulas lógicas? La respuesta a ambas es no. ``\textit{La lógica es la formalización de las matemáticas de toda la vida, y las matemáticas de toda la vida son más simples que los programas}'', según comenta Lamport, no sin razón. Un lenguaje de programación puede usar términos matemáticos como \textit{función}, pero los constructos que representa no son tan simples como sus correspondientes conceptos matemáticos. Las funciones matemáticas son simples, mientras que en la mayoría de lenguajes de programación las funciones envuelven conceptos adicionales y complejos como \textit{expresión de retorno}, \textit{convención de llamada}, etc.

TLA combina dos lógicas: la lógica de acciones, que se centra en las transiciones de estados en un sistema, y la lógica temporal estándar, que permite razonar sobre el comportamiento del sistema a lo largo del tiempo. Esta integración permite a TLA capturar con precisión tanto la dinámica de los estados individuales como la evolución global de un sistema concurrente.

\section{Conceptos básicos de sistemas concurrentes}\label{section:concurrentprop}
Antes de profundizar en el análisis y la aplicación de la Lógica Temporal de Acciones, resulta fundamental entender los conceptos básicos de la programación concurrente. Esta sección se dedica a presentarlos, estableciendo así una base sólida para apreciar la relevancia y la efectividad de TLA en la verificación de programas.

En primer lugar, debe quedar claro lo que es un programa secuencial. Un \textbf{programa secuencial} consta de un conjunto de declaraciones de datos más un conjunto de instrucciones sobre dichos datos que se ejecutan en secuencia. Ahora, se puede definir un \textbf{programa concurrente} como el conjunto de programas secuenciales ordinarios que se pueden ejecutar \textit{lógicamente} en paralelo, y cada programa secuencial es ejecutado por un \textbf{proceso}. Finalmente, la \textbf{programación concurrente} es el conjunto de notaciones y técnicas de programación utilizas para expresar paralelismo potencial y resolver problemas de sincronización y comunicación.

\subsection{Modelos de arquitecturas de programación concurrente}\label{subsec:concurrentarch}
De la definición anterior de programa concurrente, cabe destacar la importancia de decir que los programas pueden ejecutarse \textit{lógicamente} en paralelo. Esto quiere decir que la concurrencia en un sistema se puede gestionar de diversas maneras, dando lugar a un paralelismo real o ilusorio. Es por ello que se se disponen de tres modelos diferentes de concurrencia basados en la arquitectura hardware:

\begin{itemize}
    \item \textbf{Concurrencia en sistemas monoprocesador}: Puesto que sólo hay una CPU disponible, se trabaja sobre un sistema operativo multiprogramación, gestionando cómo múltiples procesos se reparten ciclos de procesador. Los mecanismos de sincronización y comunicación se hace mediante variables compartidas.
    \item \textbf{Concurrencia en sistemas multiprocesador de memoria compartida}: Los procesadores pueden compartir o no físicamente la misma memoria, pero sí disponen de un espacio de direcciones compartido. La interacción de los procesos en este modelo también se realiza mediante variables compartidas.
    \item \textbf{Concurrencia en sistemas distribuidos}: Aquí también hay múltiples procesadores. No existe una memoria común, cada procesador tiene su espacio de direcciones privado. La interacción se realiza transfiriendo datos entre procesos a través de una red de interconexión (paso de mensajes).
\end{itemize}

\subsection{Instrucciones atómicas y entrelazamiento}\label{subsec:concurrentatomic}
Una instrucción (o sentencia) de un proceso en un programa concurrente es \textbf{atómica} si siempre se ejecuta de principio a fin sin verse afectada durante su ejecución por las instrucciones de otros procesos del programa que también estén en ejecución. En otras palabras, no se verá afectada cuando el \textit{funcionamiento} de dicha instrucción \textit{no dependa nunca} de cómo estén ejecutando otras instrucciones, entendiéndose por \textit{funcionamiento de instrucción} como el efecto en el estado de ejecución del programa justo cuando ésta acaba. Como ejemplo de instrucciones atómicas se pueden considerar muchas de las instrucciones máquina de un procesador como es el caso de las instrucciones de carga (\code{LOAD}) y almacenamiento (\code{STORE}) que operan entre los registros de CPU y las celdas de memoria. Mientras tanto, un ejemplo de instrucción no atómica puede ser \code{x = x+1}. Aquí, se ejecutan 3 instrucciones diferentes: primero se carga el valor de la variable \code{x} en un registro; segundo se incrementa el valor del registro en \code{1}, y finalmente, el resultado de esa operación se almacena en la celda de memoria de la variable \code{x}. El valor de la variable \code{x} justo cuando termine la ejecución de la instrucción no atómica dependerá de que haya o no haya otras sentencias ejecutándose a la vez y que intenten escribir simultáneamente en dicha variable. Cuando eso sucede, se dice que existe \textit{indeterminación}, pues no se puede predecir el estado final del proceso a partir de su estado inicial.

Tras comprender las instrucciones atómicas y la indeterminación que surge con las no atómicas, es crucial abordar el entrelazamiento en programas concurrentes. El \textbf{entrelazamiento} se refiere a la manera en que las instrucciones de diferentes procesos se intercalan o ``entrelazan'' durante la ejecución. En un sistema concurrente, múltiples procesos avanzan aparentemente en paralelo, pero en realidad, sus instrucciones pueden ser ejecutadas de forma intercalada por el procesador. Esto significa que la secuencia exacta de ejecución de instrucciones entre procesos concurrentes puede variar de una ejecución a otra, lo que introduce un nivel significativo de no determinismo en el sistema. Por ejemplo, si dos procesos intentan modificar una misma variable compartida (como podría ser el caso en el ejemplo anterior con la variable \code{x}) sin mecanismos adecuados de sincronización, el resultado final puede depender del orden específico en que se ejecuten sus instrucciones respectivas. Este comportamiento impredecible, inherente al entrelazamiento, es una fuente principal de condiciones de carrera y otros problemas relacionados con la sincronización en sistemas concurrentes. Por lo tanto, entender y manejar el entrelazamiento es fundamental para asegurar la corrección y la estabilidad de estos sistemas, preparando el camino para explorar cómo las técnicas de sincronización y las herramientas como TLA que se verán más adelante pueden ayudar en este proceso.

\subsection{Independencia del entorno de ejecución: Hipótesis del progreso finito}\label{subsec:concurrentprogfinit}
El modelo basado en el estudio de todas las posibles secuencias de los procesos de un programa concurrente constituye una \textbf{abstracción}, donde se consideran sólo las características relevantes que determinan el resultado final del programa, ignorando otras como el estado de la memoria asignado a cada proceso, los registros particulares a los que accederá cada uno, el costo de cambios de contexto entre procesos que hace el sistema operativo, la política de planificación y las diferencias de velocidad entre entornos multiprocesador y monoprocesador. Además, el entrelazamiento de instrucciones atómicas preserva la consistencia de los resultados, pues en caso contrario, sería imposible poder razonar sobre las propiedades de corrección de los programas concurrentes que se hablarán más adelante. Es por ello que, para esa corrección de los programas concurrentes, se utiliza la hipótesis siguiente.

\subsubsection{Hipótesis del progreso finito}\label{subsubsec:concurrentprogfinithip}
El enunciado de la hipótesis es: \textit{No se puede hacer ninguna suposición acerca de las velocidades absolutas o relativas de ejecución de los procesos, salvo que es mayor que cero. Un programa concurrente se entiende sólo con base en sus componentes (procesos) y sus interacciones, sin tener en cuenta el entorno de ejecución.} Si se hicieran suposiciones que dependiesen del tiempo o de la velocidad de ejecución de los procesos, sería difícil detectar y corregir fallos y además, la corrección dependería de la configuración de ejecución, que puede cambiar.

\subsection{Exclusión mutua y sincronización}\label{subsec:concurrentexclusion}
No todas las secuencias de entrelazamiento de las instrucciones de los procesos que pueden producirse en un programa concurrente son posibles en la realidad, pues los procesos no suelen ejecutarse de una forma totalmente independiente, sino que colaboran entre ellos. Se denomina \textbf{condición de sincronización} a la restricción en el orden en que se pueden entremezclar las instrucciones que generan los procesos de un programa. Cuando se impone una condición de sincronización, uno o varios procesos deben esperar a que se cumpla una determinada condición global que depende de varios procesos. Un ejemplo de condición de sincronización sencillo puede ser el de observar el valor de una variable global compartida entre varios procesos.

\subsubsection{Sección crítica y exclusión mutua}\label{subsubsec:concurrentsc}
Al conjunto de secuencias comunes de instrucciones consecutivas que aparecen en varios procesos de un programa concurrente se denomina \textbf{sección crítica} (SC). Además, se dice que ocurre \textbf{exclusión mutua} (EM) cuando los procesos sólo funcionan correctamente si, en cada instante de tiempo, hay como mucho uno de ellos ejecutando cualquier instrucción de la sección crítica.

\subsection{Corrección y propiedades de los sistemas concurrentes}\label{subsec:concurrentproperties}
Esta subsección es imprescindible para el propósito de este capítulo, pues el objetivo es ofrecer un razonamiento preciso y eficaz sobre las propiedades que cumple un programa concurrente, utilizando para ello la Lógica Temporal de Acciones. Se entiende por \textbf{propiedad} a un atributo de un programa concurrente que es cierto para todas las posibles secuencias de entrelazamiento. Hay dos principales tipos de propiedades:

\begin{enumerate}[label=P\arabic*]
    \item \textbf{Propiedad de seguridad (safety)}: ``\textit{Nunca pasará nada malo}''. Son condiciones que deben cumplirse \textit{siempre}. Ejemplos de propiedades de seguridad son:
    \begin{enumerate}[label=P1.\arabic*]
        \item \textit{Exclusión Mutua (Mutual Exclusion)}: Dos procesos nunca entrelazan ciertas subsecuencias de operaciones.
        \item \textit{Ausencia de Interbloqueo (Deadlock-freedom)}: Nunca ocurrirá que los procesos se encuentren esperando algo que nunca sucederá.
    \end{enumerate}
    \item \textbf{Propiedad de vivacidad (liveness)}: ``\textit{Realmente sucede algo bueno}''. Son condiciones que deben cumplirse \textit{eventualmente}. Ejemplos de propiedades de vivacidad son:
    \begin{enumerate}[label=P2.\arabic*]
        \item \textit{Ausencia de inanición (starvation-freedom)}: Un proceso o grupo de procesos no puede ser indefinidamente pospuesto. En algún momento, podrá avanzar.
        \item \textit{Equidad (fairness)}: Tipo particular de propiedad de vivacidad. Un proceso que desee progresar debe hacerlo con justicia relativa con respecto a los demás.
    \end{enumerate}
\end{enumerate}

\section{La Lógica Temporal de Acciones (TLA)}\label{section:TLA}
Tras establecer los fundamentos de la programación concurrente, sus características inherentes y las propiedades que puede poseer un programa concurrente, se aborda la sección central de este capítulo: el estudio de la Lógica Temporal de Acciones (TLA) \cite{lamport1994temporal}. Como se mencionó anteriormente, TLA se origina de la combinación de dos lógicas fundamentales: la lógica de estados y la lógica temporal estándar. En su estructura, TLA se fundamenta en la lógica de predicados ordinaria, ampliada con dos tipos de variables: las \textit{variables rígidas}, comúnmente referidas como constantes, y las \textit{variables flexibles}, correspondientes a las variables de programa en las especificaciones. TLA incorpora operadores clásicos de la lógica (como $\land$, $\lor$, $\implies$),  incluyendo ahora nuevos como los cuantificadores $\forall$ y $\exists$. También el operador $'$ (prima), que funciona como el operador de estado siguiente (next) en la lógica temporal, utilizado para describir transiciones entre estados. Además, se incluye el operador ``$\Box$'' (siempre), esencial para especificar propiedades temporales y reflejar la naturaleza evolutiva y continua de los sistemas que TLA busca modelar. Se introducirán más operadores que resultan de la combinación de los operadores primitivos.

\subsection{Lógica de acciones}\label{subsection:LActions}
En la manipulación de datos por algoritmos, se realiza frecuentemente la asignación de valores a variables. Se define $Val$ como un conjunto que comprende \textit{valores} diversos, incluyendo números enteros, números reales, cadenas de caracteres y conjuntos tales como los números naturales, representados por $Nat$ \footnote{Se define $Nat$ como $\mathbb{N} \cup {0}$, incluyendo el cero.}. Adicionalmente, $Var$ se identifica como el conjunto de todos los nombres posibles para las variables, ejemplificados por $x$ o $nombre$. En otro aspecto, la lógica se caracteriza por un conjunto de reglas para la manipulación de fórmulas. Sin embargo, para la comprensión del significado de dichas fórmulas y sus procesos de manipulación, es imprescindible la presencia de una \textit{semántica}. Se utiliza la notación $[[F]]$ para indicar el significado semántico de cada elemento sintáctico $F$ en la lógica.\footnote{La notación $[[\cdot]]$ se interpreta como una función entre conjuntos. A lo largo del documento, se procederá a definir formalmente cada elemento lógico, utilizando de manera consistente el mismo término para todas las funciones semánticas.}

En esta lógica, la semántica se define a través del concepto de estados. Se entiende por \textbf{estado} a una función específica que asigna valores a las variables, es decir, una función que conecta el conjunto de variables con su respectiva colección de valores:

\begin{align*}
s : Var &\to Val \\
x &\mapsto s(x)
\end{align*}

El conjunto que incluye todos los estados posibles se representa como $St$. El término $s[[x]]$ se utiliza para referirse a $s(x)$, interpretando el significado $[[x]]$ de la variable $x$ como la función que relaciona los estados con sus valores, aplicando una notación de posfijo para la función. De manera formal, esto se define como:

\begin{align*}
[[\cdot]] : Var &\to (St \to Val) \\
[[x]] &\mapsto (s \mapsto s(x))
\end{align*}

Después de establecer una comprensión básica de los elementos más sencillos, es apropiado introducir las \textbf{expresiones} lógicas, que se forman utilizando operadores y otros elementos como variables, constantes y expresiones del tipo $o(e_1,\ldots,e_n)$, en las cuales $o$ representa un operador y cada $e_i$ es una expresión. Un ejemplo es $+(x,y)$, aunque convendrá escribirlo como $x+y$. Cada operador $o$ tiene un significado semántico asignado, representado por $[[o]]$. Así, el significado de una expresión compuesta como $+(e_1,e_2)$ se establece de manera inductiva. En un estado $s$, esta expresión se interpreta como $[[+]](s[[e_1]],s[[e_2]])$, aplicando el significado semántico del operador a los valores de las subexpresiones en ese estado.

Es importante mencionar que bajo los fundamentos de la teoría de conjuntos, es posible representar todos los operadores requeridos en esta lógica con un conjunto limitado de operadores primitivos, como $\land$ (y), $\neg$ (no), $\in$ (pertenencia) y $\epsilon$ (elección de Hilbert). Esta aproximación facilita la formulación y entendimiento de las expresiones. Habitualmente, se emplea una notación simplificada que no diferencia entre un operador y su interpretación semántica. Por ejemplo, en vez de detallar la expresión semántica completa $s[[x+y]]$, se abrevia a $s[[x]] + s[[y]]$, lo que simplifica la lectura y análisis de las expresiones en este marco.

Se identifican tres categorías de expresiones. Las $\textbf{funciones de estado}$ son expresiones como $x + y^2 - 7$. Se define $StFunc$ como el conjunto de estas funciones de estado. El significado $[[f]]$ de una función $f$ es una aplicación desde el conjunto $St$ de estados hacia la colección $Val$ de valores:

\begin{align*}
[[\cdot]] : StFunc &\to (St \to Val) \\
[[f]] &\mapsto (s \mapsto s[[f]] \triangleq f(\forall 'v' : s[[v]] / v)) \hspace{0.1cm}\textsuperscript{\dag}
\end{align*}

\footnotetext{\dag \hspace{0.1cm} $\triangleq$ significa ``igual por definición''.}


donde $f(\forall 'v' : s[[v]]/v)$ indica el valor resultante de $f$ al sustituir $s[[v]]$ (el valor de la variable) por $v$, para cada variable $v$. En el caso mencionado, $[[x + y^2 - 7]]$ otorga a un estado $s$ el valor $s[[x]] + s[[y]]^2 -7$, donde $2$ y $7$ son constantes simbólicas que también representan los valores que simbolizan. No se realiza distinción entre símbolos constantes y sus valores representativos.

Se considera que una variable $x$ actúa igualmente como una función de estado. Por lo tanto, la definición de $[[f]]$ para una función de estado $f$ se aplica también a la definición de $[[x]]$ para una variable $x$.

El segundo tipo de expresión se conoce como \textbf{predicado de estado} (a menudo referido simplemente como \textbf{predicado}), y se construye de manera similar a las funciones de estado. Ejemplos de predicados incluyen $x^2 = y -3$ y $x \in Nat$. El conjunto que engloba a todos los predicados de estado se designa como $StPred$. La distinción principal entre las funciones de estado y los predicados radica en su significado. El significado $[[P]]$ de un predicado $P$ se define como una función que asocia el conjunto de estados $St$ con el conjunto de valores de verdad, que se representa por $\left\lbrace \top, \bot\right\rbrace$ (verdadero y falso, respectivamente):

\begin{align*}
[[\cdot]] : StPred &\to (St \to \left\lbrace \top, \bot \right\rbrace) \\
[[P]] &\mapsto (s \mapsto s[[P]])
\end{align*}

La última categoría de expresión en esta lógica es fundamental y se define como una \textbf{acción}. Consiste en una expresión compuesta por variables, variables alteradas (indicadas con prima) y símbolos constantes. Expresiones tales como $x' + 1 = y$ y $x-1 \not \in z'$ son ejemplos de acciones, donde $x,y,z$ son variables. Dichas acciones establecen una conexión entre estados previos y estados posteriores; las variables sin prima denotan el estado previo, mientras que las variables con prima aluden al estado posterior. Por ejemplo, en $z = y' - 1$, se interpreta que el valor de $z$ en el estado anterior es inferior al valor de $y$ en el estado posterior.

Formalmente, el significado $[[\mathcal{A}]]$ de una acción $\mathcal{A}$ se entiende como una relación entre dos estados, es decir, una función que asigna un valor de verdad $s[[\mathcal{A}]]t$ a una pareja de estados $\langle s,t \rangle$. Al conjunto de estas acciones se le denomina $Act$. Por lo tanto, el valor de verdad $s[[\mathcal{A}]]t$ se determina considerando a $s$ como el \textit{estado anterior} y a $t$ como el \textit{estado subsiguiente}. Esto se logra en $\mathcal{A}$ sustituyendo cada variable $v$ sin prima por su valor en $s$ ($s[[v]]$) y cada variable con prima $v'$ por su valor en $t$ ($t[[v]]$).

\begin{align*}
[[\cdot]] : Act &\to (St \times St \to \left\lbrace \top, \bot \right\rbrace) \\
[[\mathcal{A}]] &\mapsto ((s,t) \mapsto s[[\mathcal{A}]]t \triangleq \mathcal{A}(\forall 'v' : s[[v]] / v, t[[v]] / v'))
\end{align*}

Con el ejemplo anterior $z = y' - 1$, quedaría entonces $[[z = y' - 1 ]] = s[[z]] = t[[y]] - 1$. Al par de estados $\langle s,t\rangle$ se le nombra como \textbf{$\mathcal{A}$ paso} si, y sólo si, $s[[\mathcal{A}]]t$ se evalúa como verdadero ($\top$). Ahora, $\mathcal{A}$ se dice que es una \textbf{acción válida}, escrito como $\models \mathcal{A}$, si y sólo si todo paso es un $\mathcal{A}$ paso. Formalmente:

\begin{align}
\models \mathcal{A} \triangleq \forall s,t \in St : s[[\mathcal{A}]]t
\end{align}

Con esta nueva definición de acción, y recordando la definición de predicado, se puede ver a $P$ como una acción que no contiene variables con prima. Un par de estados $\langle s,t \rangle$ es un \textbf{$P$ paso sí} y sólo sí $s$ satisface $P$, o en otras palabras, $s[[P]]t = s[[P]]$ se evalúa como verdadero. Para cualquier función de estado o predicado $F$, se define ahora $F'$ como la expresión obtenida de la sustitución de cada variable $v$ de $F$ por la variable con prima $v'$:

\begin{align}
F' \triangleq F(\forall 'v' : v' / v)
\end{align}

En particular, si $P$ es un predicado, entonces $P$ es una acción, y $s[[P']]t = t[[P]]$ para todo par de estados $\langle s,t \rangle$. También las funciones de estado pueden llevar prima, y se denomina \textit{Unchanged} $f$ como:

\begin{align}
\textit{Unchanged} \hspace{0.2cm} f \triangleq f = f'
\end{align}

es decir, \textit{Unchanged} $f$ es una acción y se dice que es un \textit{paso tartamudo}, donde $f$ no cambia. Se puede considerar también combinar acciones con funciones de estado. Dada una acción $\mathcal{A}$ y una función de estado $f$ se define también $[\mathcal{A}]_f$ como:

\begin{align}
[\mathcal{A}]_f \triangleq \mathcal{A} \lor \textit{Unchanged} \hspace{0.2cm} f \triangleq \mathcal{A} \lor (f = f')
\end{align}

y se dice que o es un $\mathcal{A}$ paso o un paso tartamudo (con respecto a $f$). Por otro lado también se encuentra su forma dual $\langle \mathcal{A} \rangle_f$:

\begin{align}
\langle\mathcal{A}\rangle_f \triangleq \mathcal{A} \land \neg(\textit{Unchanged} \hspace{0.2cm} f) \triangleq \mathcal{A} \land \neg(f = f')
\end{align}

\noindent
que indica que es un $\mathcal{A}$ paso que necesariamente cambia $f$.

En el caso de cualquier acción $\mathcal{A}$, se establece el predicado \textit{Enabled} $\mathcal{A}$ que resulta verdadero para un estado específico únicamente si es factible ejecutar un paso de $\mathcal{A}$ comenzando desde ese estado. Desde el punto de vista sintáctico, este nuevo predicado puede describirse de la siguiente manera. Suponiendo que $v_1,\ldots,v_n$ son variables flexibles\footnote{Dentro del contexto de la lógica modal y otros sistemas lógicos, es importante diferenciar entre las variables rígidas y las flexibles. Las variables rígidas mantienen su valor constante en diferentes contextos o mundos posibles, presentando una naturaleza invariable y estable, independientemente del estado o situación. En contraste, las variables flexibles tienen valores que varían según el contexto o mundo posible específico. Estas variables no poseen un valor fijo, lo que les permite adaptarse a diversas situaciones o estados dentro de un sistema lógico. Esta distinción es esencial para comprender la interpretación y asignación de valores a las variables en distintos contextos lógicos y mundos posibles.} que aparecen en $\mathcal{A}$, entonces:

\begin{align}
\textit{Enabled} \hspace{0.2cm} \mathcal{A} \triangleq \exists c_1,\ldots,c_n : \mathcal{A}(c_1 / v_1', \ldots c_n / v_n')
\end{align}

donde $\mathcal{A}(c_1 / v_1', \ldots c_n / v_n')$ denota la fórmula obtenida de la sustitución de nuevas variables rígidas \footnote{Ver nota inmediatamente anterior.} $c_i$ por todas las ocurrencias de $v_i'$ en $\mathcal{A}$. Su semántica se define de esta forma para cualquier estado $s$:

\begin{align}
s[[\textit{Enabled} \hspace{0.2cm} \mathcal{A}]] \triangleq \exists t \in St : s[[\mathcal{A}]]t
\end{align}

Hasta ahora, es fundamental resaltar cómo esta lógica de acciones se relaciona con los sistemas concurrentes. En TLA, una instrucción atómica (como se menciona en~\ref{subsec:concurrentatomic}) se representa por una acción $\mathcal{A}$, y un par $(s,t)$ constituye un paso $\mathcal{A}$ si y solo si la ejecución de la instrucción en el estado $s$ resulta en el estado $t$. Adicionalmente, si una acción $\mathcal{A}$ simboliza una instrucción atómica, entonces el predicado \textit{Enabled} $\mathcal{A}$ es verdadero para todos los estados donde la instrucción es ejecutable.

\subsection{Lógica Temporal}\label{subsection:LTemporal}
Una \textbf{fórmula temporal} se compone de fórmulas básicas combinadas con operadores clásicos ($\land$, $\neg$), y el operador unario $\Box$, que se interpreta como \textit{siempre}. Vale la pena señalar que otros operadores que se mencionarán más adelante se definen usando $\land$ y $\neg$. Un ejemplo de una fórmula temporal es $F \land \Box(\neg G)$, donde $F,G$ son fórmulas básicas. El conjunto total de fórmulas, incluyendo las temporales y las básicas, se denomina $Form$.

La semántica de la lógica temporal se centra en los \textit{comportamientos}, donde un \textbf{comportamiento} es una serie infinita de estados. Este enfoque es útil para hacer una comparación con lo que sucede en la ejecución de un algoritmo por una máquina, cubriendo también casos de secuencias finitas de estados que marcan el fin de un programa\footnote{Aun así, considerar secuencias infinitas de estados es adecuado y suficiente.}. El conjunto de todos los comportamientos se representa como $St^\infty$. Para entender el significado de una fórmula temporal, es necesario analizar las fórmulas elementales que contiene, o sea, basta con definir la semántica de $[[F \land G]]$, $[[\neg F]]$ y $[[\Box F]]$ en términos de $[[F]]$ y $[[G]]$.

La interpretación de una fórmula temporal implica hacer una afirmación acerca de los comportamientos. De forma formal, el significado $[[F]]$ de una fórmula $F$ es el valor de verdad que esta fórmula asigna a un comportamiento $\sigma$:

\begin{align*}
[[\cdot]] : Form &\to (St^\infty \to \left\lbrace \top, \bot \right\rbrace) \\
[[F]] &\mapsto (\sigma \mapsto \sigma[[F]])
\end{align*}

Se dice además que $\sigma$ satisface $F$ si y sólo si $\sigma[[F]]$ se evalúa como verdadera. Ahora, ya se pueden definir también los significados de $[[F \land G]]$ y $[[\neg F]]$:

\begin{gather*}
\sigma [[F \land G]] \triangleq \sigma[[F]] \land \sigma[[G]] \\
\sigma [[\neg F]] \triangleq \neg \sigma[[F]]
\end{gather*}

Por tanto, un comportamiento $\sigma$ satisface $F \land G$ si satisface a ambas fórmulas, y satisface a $\neg F$ si y sólo si no satisface a $F$. Ahora, se pueden derivar fórmulas similares para otros operadores:

\begin{gather*}
F \implies G \equiv \neg (F \land \neg G) \\
F \lor G \equiv \neg (\neg F \land \neg G)
\end{gather*}

\noindent
y su respectivo significado semántico como:

\begin{align*}
\sigma[[F \implies G]] &\equiv \sigma[[\neg (F \land \neg G)]] \\
&\triangleq \neg\sigma [[F \land \neg G]] \\
&\triangleq \neg(\sigma[[F]] \land \neg\sigma[[G]]) \\
&\triangleq \sigma[[F]] \implies \sigma[[G]]
\end{align*}

\begin{align*}
\sigma[[F \lor G]] &\equiv \sigma[[\neg (\neg F \land \neg G)]] \\
&\triangleq \neg \sigma[[\neg F \land \neg G]] \\
&\triangleq \neg (\neg\sigma[[F]] \land \neg\sigma[[G]]) \\
&\triangleq \sigma[[F]] \lor \sigma[[G]]
\end{align*}

Para definir el significado de $[[\Box F]]$, que es la única que faltaba, se hará en términos de $[[F]]$. Sea $\sigma = \langle s_0, s_1, \ldots \rangle$ el comportamiento cuyos estados están numerados de forma natural (el primer estado es $s_0$, el siguiente $s_1$, y así sucesivamente). Entonces:

\begin{align}
\sigma[[\Box F]] \triangleq \forall n \in Nat : \langle s_n, s_{n+1}, s_{n+2}, \ldots \rangle[[F]]
\end{align}

Esta interpretación sigue los preceptos clásicos de la lógica temporal lineal. Se puede comprender el comportamiento $\sigma$ como la evolución del universo, donde $s_n$ indica el estado del universo en el instante de tiempo $n$. $\sigma[[\Box F]]$ indica por tanto, que $F$ es verdadera en todos los tiempos durante el comportamiento $\sigma$, o dicho de otra manera, $\Box F$ asegura que $F$ es verdad \textit{siempre}.

\subsection{Fórmulas temporales. Validez}\label{subsection:LTForms}
A continuación se presentan algunas de las fórmulas temporales que se utilizarán de aquí en adelante, teniendo en cuenta todo el marco teórico visto en la sección anterior. Estas fórmulas tienen su importancia en la verificación de programas concurrentes, donde ya se están empezando a ver dichos programas como objetos matemáticos, en términos de \textit{acciones} y \textit{comportamientos}.

\subsubsection{Eventualidad}\label{subsubsection:LTFormsEventually}
\noindent
Para cualquier fórmula $F$, se define $\Diamond F$ como:

\begin{align}
\Diamond F \triangleq \neg \Box \neg F
\end{align}

Esta fórmula asegura que no siempre $F$ es falso, o en otras palabras, $F$ es \textit{eventualmente} verdad. El significado semántico de esta fórmula se puede expresar teniendo en cuenta las relaciones entre los cuantificadores universales, pues $\neg\forall\neg \equiv \exists$, por lo que:

\begin{align}
\sigma[[\Diamond F]] \equiv \exists n \in Nat : \langle s_n, s_{n+1}, s_{n+2}, \ldots \rangle[[F]]
\end{align}

para cualquier comportamiento $\sigma = \langle s_0, s_1, s_2, \ldots \rangle$. La fórmula $\Diamond F$ asegura que $F$ es verdad en algún punto del comportamiento.

\subsubsection{Infinitamente a menudo}
Para cualquier fórmula $F$, y combinando los operadores siempre ($\Box$) y eventualmente ($\Diamond$), se define la fórmula:

\begin{align}
\Box \Diamond F
\end{align}

Esta fórmula se lee como que \textit{$F$ es cierta infinitamente a menudo}, y será cierta para un comportamiento $\sigma = \langle s_0, s_1, \ldots \rangle \in St^\infty$ si y sólo si $\Diamond F$ es cierta en todos los tiempos $n$ durante ese comportamiento, y $\Diamond F$ será cierta si y sólo si existe un tiempo $m$ mayor o igual a $n$ donde $F$ es cierta. Formalmente:

\begin{align}
\sigma [[\Box \Diamond F]] \equiv \forall n \in Nat : \exists m \in Nat : \langle s_{n+m}, s_{n+m+1},\ldots \rangle [[F]]
\end{align}

\subsubsection{Eventualmente siempre}
Para cualquier fórmula $F$ y de forma similar a la anterior fórmula temporal, se define la fórmula:

\begin{align}
\Diamond \Box F
\end{align}

Esta fórmula se lee como que \textit{eventualmente, $F$ es cierta siempre}, y será cierta para un comportamiento $\sigma = \langle s_0, s_1, \ldots \rangle$ si y sólo si existe algún tiempo $m$ tal $F$ es verdad a partir de ese momento. Formalmente:

\begin{align}
\sigma [[\Diamond \Box F]] \equiv \exists m \in Nat : \forall n \geq m : \langle s_{n}, s_{n+1}, \ldots \rangle [[F]]
\end{align}

\subsubsection{Conduce a}
\noindent
Para cualquier par de fórmulas $F,G$ se define:

\begin{align}
F \leadsto G \triangleq \Box(F \implies \Diamond G)
\end{align}

El operador ($\leadsto$) se lee como \textit{conduce a}, y la fórmula asegura que siempre se da el caso de que si F es verdadero, entonces G es verdadero ahora o en algún momento posterior. Si $F \leadsto G$ y $G \leadsto H$, entonces $F \leadsto H$.

\subsubsection{Justicia (fairness)}
En la sección referenciada como ~\ref{subsec:concurrentproperties}, se abordaron las propiedades que puede tener un programa concurrente, incluyendo su interpretación. TLA facilita el análisis de si un programa cumple con la propiedad de justicia, una categoría específica dentro de las propiedades de vivacidad. Empleando términos previamente mencionados, la \textit{justicia} puede entenderse como la garantía de que una instrucción en particular será ejecutada por el programa en algún momento, siempre que sea factible. Se identifican dos variantes de justicia: la \textit{justicia débil} y la \textit{justicia fuerte}.

En el caso de la justicia débil, se establece que una instrucción específica del programa será ejecutada en algún momento o se volverá eventualmente imposible de ejecutar. En cuanto a la justicia fuerte, se sostiene que la instrucción será ejecutada eventualmente o que su ejecución no será infinitamente a menudo posible. Al afirmar que estas condiciones se cumplen en todo momento, las expresiones para estas propiedades (definidas aún de manera informal) serían:

\begin{align*}
\textit{justicia débil}: \hspace{0.3cm} (\Box\Diamond \text{ejecutado}) \lor (\Box\Diamond \text{imposible}) \\
\textit{justicia fuerte}: \hspace{0.3cm} (\Box\Diamond \text{ejecutado}) \lor (\Diamond\Box \text{imposible}) \\
\end{align*}

queda sólo por determinar el concepto de ``ejecutado'' e ``imposible''. ``Ejecutar'' una instrucción significa tomar un $\langle \mathcal{A}\rangle_f$ (ver definición 5.5) paso, para alguna acción $\mathcal{A}$ y una función de estado $f$, y es posible tomarlo  si y sólo si \textit{Enabled} $\langle \mathcal{A}\rangle_f$ es cierta. Por lo tanto, \textit{Enabled} $\langle A \rangle_f$ aserta que es posible ejecutar la instrucción representada por la acción $\langle \mathcal{A}\rangle_f$, designando por ``imposible'' entones $\neg$\textit{Enabled} $\langle \mathcal{A}\rangle_f$. Ahora sí se puede hacer una definición formal de las fórmulas de justicia débil (WF) y fuerte (SF):

\begin{align}
\text{WF}_f(\mathcal{A}) \triangleq (\Box\Diamond \langle \mathcal{A}\rangle_f) \lor (\Box\Diamond \neg \textit{Enabled} \hspace{0.1cm} \langle \mathcal{A}\rangle_f) 
\end{align}

\begin{align}
\text{SF}_f(\mathcal{A}) \triangleq (\Box\Diamond \langle \mathcal{A}\rangle_f) \lor (\Diamond\Box \neg \textit{Enabled} \hspace{0.1cm} \langle \mathcal{A}\rangle_f) 
\end{align}

Una consecuencia directa de esta definición, y utilizando la siguiente tautología de la lógica modal \footnote{Se recuerda que la lógica temporal, es una extensión de la lógica modal, por lo que muchas de sus conectivas, reglas, y tautologías, siguen manteniéndose.} $\Diamond\Box F \implies \Box\Diamond F$ para cualquier fórmula $F$, es que:

\begin{align}
\text{SF}_f(\mathcal{A}) \implies \text{WF}_f(\mathcal{A})
\end{align}

\subsubsection{Validez de fórmulas temporales}
Tras haber mostrado algunas de las fórmulas temporales más útiles, hay que precisar la definición de fórmula válida. Una fórmula temporal es \textbf{válida}, escrito como $\models F$ si y sólo si es satisfecha para todos los comportamientos posibles. Esto se traduce en:

\begin{align}
\models F \triangleq \forall \sigma \in St^\infty : \sigma[[F]]
\end{align}

\subsubsection{Fórmulas temporales y programas concurrentes}
Los programas concurrentes y sus propiedades serán representadas como fórmulas temporales, tal y como se esperaba en el propósito marcado de ofrecer un formalismo matemático para la corrección. Puesto que un comportamiento modela una ejecución de un programa concurrente, se dirá que $\models F \implies G$ cuando el algoritmo o programa $F$ satisface la propiedad $G$, $\forall \sigma \in St^\infty$.

\subsection{Características adicionales de la TLA}
Hasta el momento, se han visto todos los objetos que conforman la Lógica Temporal de Acciones: los estados, las expresiones y las fórmulas temporales. Con todas estas herramientas, el siguiente objetivo es realizar una descripción de las reglas que la conforman, además de hacer alguna que otra consideración adicional.

En \ref{subsection:LActions} se definió que, si $\mathcal{A}$ es una acción, $\langle s,t \rangle$ es un $\mathcal{A}$ paso si y sólo si $s[[\mathcal{A}]]t$ es verdadera, donde $s[[\mathcal{A}]]t$ es el significado semántico que se le otorgó a $[[\mathcal{A}]]$. Ahora, tomando $\sigma = \langle s_0, s_1, \ldots \rangle \in St^\infty$, se puede extender el significado semántico de $[[\mathcal{A}]]$ como sigue: 

\begin{align}
\sigma[[\mathcal{A}]] = \langle s_0, s_1, s_2, \ldots \rangle[[\mathcal{A}]] \triangleq s_0[[\mathcal{A}]]s_1
\end{align}

Por otra parte, también se introdujo lo que se denominó \textit{pasos tartamudos} en funciones de estado, esto es cuando $f = f'$, siendo $f$ función de estado. Dada cualquier acción $\mathcal{A}$ y una función de estado $f$ se introdujo la notación $[\mathcal{A}]_f$ (ver definición 5.4) que representaba que o es un $\mathcal{A}$ paso o un paso tartamudo con respecto a $f$. Se procede a incluir notación particular sobre esta última definición como sigue:

\begin{align}
[\mathcal{A}]_{\langle x,y \rangle} &\equiv \mathcal{A} \lor (\langle x,y \rangle ' = \langle x,y \rangle) \\ \nonumber
&\equiv \mathcal{A} \lor ((x' = x) \land (y' = y))
\end{align}

donde $x,y \in Var$ y $\langle x,y \rangle$ es una tupla ordenada. Esta nueva notación permite indicar que los estados involucrados o son un $\mathcal{A}$ paso o son un paso tartamudo, dejando las variables $x,y$ invariantes.

Ahora sí, se entiende por \textbf{TLA} a la lógica temporal cuyas fórmulas elementales son predicados y fórmulas de la forma $\Box[\mathcal{A}]_f$, donde $\mathcal{A}$ es una acción y $f$ es una función de estado.

\subsubsection{Invarianza ante tartamudez}\label{subsubsection:LTAsttutering}
Este concepto, se refiere a la propiedad de una fórmula de mantener su verdad o falsedad incluso cuando se introducen repeticiones de estados (tartamudeos) que no cambian el estado observable del sistema. En otras palabras, una fórmula invariante frente a tartamudeo no se ve afectada por los cambios irrelevantes en la secuencia de estados.

Se denomina $\natural \sigma$ como el comportamiento obtenido de $\sigma = \langle s_0, s_1, \ldots \rangle \in St^\infty$ eliminando los pasos tartamudos. La definición precisa es:

\[
\natural(\sigma) = 
\begin{cases} 
\langle s_0, s_0, s_0, \ldots \rangle & \text{si } \forall n \in \mathbb{N},\ s_n = s_0 \\
\natural(\langle s_{1}, s_2, s_3, \ldots \rangle) & \text{si } s_1 = s_0 \\
\langle s_0 \rangle \circ \natural(\langle s_{1}, s_2, s_3, \ldots \rangle) & \text{en otro caso}
\end{cases}
\]

donde $\circ$ denota ``composición'' de comportamientos. A continuación, se enuncia uno de los resultados más importantes de la Lógica Temporal de Acciones:
\begin{proposicion}
    Las fórmulas de la Lógica Temporal de Acciones son invariantes ante tartamudez. Formalmente: $\natural \sigma = \natural \tau$ implica que $\sigma[[F]] = \tau[[F]]$ para toda fórmula $F$ de la TLA y para cualquier $\sigma,\tau \in St^\infty$.
\end{proposicion}

\subsubsection{Cuantificación sobre variables flexibles}
En esta sección se aborda la definición de la cuantificación existencial $\exists x: F$ en el contexto de variables flexibles y fórmulas temporales. La veracidad de $F$ no depende específicamente de los valores actuales de $x$, sino de la existencia de ciertos valores que $x$ puede tomar, haciendo que $F$ sea verdadera. Este enfoque permite una comprensión más profunda de la interacción entre variables cuyos valores son susceptibles de cambio y las fórmulas que se desarrollan en un marco temporal.

Para definir el significado de $\exists x : F$ formalmente, es necesario incluir definiciones auxiliares. Para cualquier variable $x \in Var$ y estados $s,t \in St$, se define $s =_x t$ y significa que $s$ y $t$ assignan los mismos valores a todas las variables distintas de $x$. Esto es:

\begin{align}
s =_x t \triangleq \forall 'v' \neq 'x' : s[[v]] = t[[v]]
\end{align}

y se extiende esta definición a comportamientos $\sigma,\tau \in St^\infty$, con $\sigma = \langle s_0, s_1, \ldots \rangle$ y $\tau = \langle t_0, t_1, \ldots \rangle$ ,de una manera natural:

\begin{align}
\sigma =_x \tau \triangleq \forall n \in Nat : s_n =_x t_n
\end{align}

\noindent
Finalmente, se puede definir el significado de $\exists x : F$ como:

\begin{align}
\sigma[[\exists x : F]] \triangleq \exists \rho, \tau \in St^\infty : (\natural \sigma = \natural \rho) \land (\rho =_x \tau) \land \tau[[F]]
\end{align}

donde $\natural$ es la operación para eliminar pasos tartamudos definida en ~\ref{subsubsection:LTAsttutering}. En el marco de la TLA, el operador de cuantificación existencial $\exists x$ adquiere una interpretación única y ampliada en comparación con su uso en la lógica clásica. Mientras que en la lógica estándar, $\exists x$ denota la existencia de un único valor que puede ser sustituido por $x$, en esta lógica, este operador se extiende para afirmar la existencia no solo de un valor singular, sino de una secuencia infinita de valores para $x$. Esta peculiaridad refleja la naturaleza dinámica y temporal de las variables en la lógica, donde $x$ puede representar diferentes valores en distintos momentos o estados a lo largo del tiempo. A pesar de esta ampliación en su significado, el operador $\exists x$ en TLA sigue obedeciendo las leyes ordinarias de la cuantificación existencial. Esta continuidad garantiza que, aunque se aplique a un conjunto más rico de situaciones, el operador mantiene su coherencia lógica y su integridad.

\subsection{TLA: Sintaxis}\label{subsection:TLASyntax}
En esta sección, se procede a consolidar y resumir la sintaxis de la Lógica Temporal de Acciones (TLA), integrando y sintetizando los elementos ya introducidos a lo largo de la discusión previa. Este apartado sirve como una revisión estructurada y formal de los aspectos sintácticos de TLA, proporcionando una visión cohesiva y comprensiva de su marco lingüístico.

La finalidad de esta recapitulación es asegurar una comprensión clara y completa de la estructura sintáctica de TLA, enfatizando cómo los diferentes componentes se entrelazan para formar un lenguaje lógico coherente y funcional. Se revisarán los símbolos básicos, las reglas de formación y la organización general de la sintaxis, reafirmando su papel en la formulación de expresiones lógicas dentro del contexto de TLA.

\subsubsection{Operadores primarios de la lógica}
\noindent
Los operadores primarios utilizados para la TLA son los siguientes:
\vspace{0.2cm}
\newline
\begin{tabular}{ll}
  $\neg$ : negación & $\land$ : conjunción \\
  $\exists$ : cuantificador existe & $\Box$ : siempre\\
\end{tabular}

\subsubsection{Operadores derivados de la lógica}
\noindent
Los operadores derivados son aquellos que se pueden expresar en términos de los primarios, y son los siguientes:
\vspace{0.2cm}
\newline
\begin{tabular}{ll}
  $\lor$ : disyunción & $\implies$ : implicación \\
  $\Diamond$ : eventualmente & $\leadsto$ : conduce a\\
\end{tabular}

\subsubsection{Conjuntos}
\noindent
Algunos de los conjuntos importantes definidos son:
\vspace{0.2cm}
\newline
\begin{tabular}{ll}
  $Val$ : valores & $Var$ : variables \\
  $St$ : estados & $St^\infty$ : comportamientos \\
  $Act$ : acciones & $Form$ : fórmulas \\
\end{tabular}

\subsubsection{Convención de nombres}
\noindent
Para hacer una notación más integral, aquí se define una convención de nombres para cada objeto sintáctico:
\vspace{0.2cm}
\newline
\begin{tabular}{ll}
  $\langle \text{función de estado} \rangle$ : $f$, $g$, $h$, etc. & $\langle \text{estados} \rangle$ : $s$,$t$,$r$,$s_0$,$s_1$,$s_2$, etc. \\
  $\langle \text{comportamiento} \rangle$ : $\sigma$, $\tau$, $\rho$, etc. & $\langle \text{variable rígida} \rangle$ : $c$, $d$, etc.\\
  $\langle \text{acción} \rangle$ : $\mathcal{A}$, $\mathcal{B}$, $\mathcal{M}$, $\mathcal{N}$, etc. & $\langle \text{fórmula} \rangle$ : $F$, $G$, $H_c$, $H_d$ etc.\\
  $\langle \text{predicado} \rangle$ : $P$, $Q$, $I$, etc. \\
\end{tabular}

\subsubsection{Notación adicional}
\noindent
Alguna notación adicional para ciertos objetos:
\vspace{0.2cm}
\newline
\begin{tabular}{ll}
  $f' \triangleq f(\forall 'v' : v' / v)$ & $[\mathcal{A}]_f \triangleq \mathcal{A} \lor (f' = f)$ \\ 
  $\langle A \rangle_f \triangleq \mathcal{A} \land (f' \neq f)$ & \textit{Unchanged} $f \triangleq f' = f$\\
  $F \lor G \triangleq \neg(\neg F \land G)$ & $F \implies G \triangleq \neg(F \land \neg G)$\\
  $\Diamond F \triangleq \neg \Box \neg F$ & $F \leadsto G \triangleq \Box(F \implies \Diamond G)$\\
  $\text{WF}_f(\mathcal{A}) \triangleq (\Box\Diamond \langle \mathcal{A}\rangle_f) \lor (\Box\Diamond \neg \textit{Enabled} \hspace{0.1cm} \langle \mathcal{A}\rangle_f)$ & $\text{SF}_f(\mathcal{A}) \triangleq (\Box\Diamond \langle \mathcal{A}\rangle_f) \lor (\Diamond\Box \neg \textit{Enabled} \hspace{0.1cm} \langle \mathcal{A}\rangle_f)$  \\
\end{tabular}

\subsubsection{Sintaxis}
\noindent
A continuación, se escriben las reglas para la construcción de los objetos lógicos:

\begin{align*}
    \langle \text{fórmula} \rangle &\triangleq \langle \text{predicado} \rangle \hspace{0.2cm} | \hspace{0.2cm} \langle \Box [\langle \text{acción} \rangle]_{\langle \text{función de estado} \rangle} \rangle \\
    &| \hspace{0.2cm} \neg \langle \text{fórmula} \rangle \hspace{0.2cm} | \hspace{0.2cm} \langle \text{fórmula} \rangle \land \langle \text{fórmula} \rangle \\
    &| \hspace{0.2cm} \Box \langle \text{fórmula} \rangle \hspace{0.2cm} | \hspace{0.2cm} \exists \langle \text{var. rígida} \rangle : \langle \text{fórmula} \rangle
\end{align*}

\begin{align*}
    \langle \text{acción} \rangle &\triangleq \text{expresión con constantes, variables, y variables con prima}
\end{align*}

\begin{align*}
    \langle \text{predicado} \rangle &\triangleq \text{función de estado} \hspace{0.2cm} | \hspace{0.2cm} \textit{Enabled} \hspace{0.2cm} \langle \text{acción} \rangle
\end{align*}

\begin{align*}
    \langle \text{función de estado} \rangle &\triangleq \text{expresión con constantes y variables}
\end{align*}

\subsection{TLA: Semántica}\label{subsection:TLASemantic}
En la presente sección, se aborda de forma resumida la semántica de la Lógica Temporal de Acciones (TLA), integrando los principios y conceptos previamente explorados para ofrecer una visión general del significado y la interpretación en TLA.

\subsubsection{Interpretaciones semánticas}
\begin{tabular}{ll}
    $s[[f]] \triangleq f(\forall 'v' : s[[v]] / v)$ & $s[[\mathcal{A}]]t \triangleq \mathcal{A}(\forall 'v' : s[[v]] / v, t[[v]] / v')$\\
    $\sigma[[F \land G]] \triangleq \sigma[[F]] \land \sigma[[G]]$ & $\sigma[[\neg F]] = \neg \sigma [[F]]$\\
    $\sigma[[F \implies G]] \triangleq \sigma[[F]] \implies \sigma[[G]]$ & $\sigma[[F \lor G]] \triangleq \sigma[[F]] \lor \sigma[[G]]$\\
    $\sigma[[\exists c : F]] : \exists c \in Val : \sigma[[F]]$ &\\  
\end{tabular}

\vspace{0.2cm}
\begin{tabular}{ll}
    $\models \mathcal{A} \triangleq \forall s,t \in St : \hspace{0.2cm} \models s[[\mathcal{A}]]t$ & $\models F \triangleq \forall \sigma \in St^\infty : \hspace{0.2cm} \models \sigma[[F]]$\\
\end{tabular}

\vspace{0.2cm}
\begin{tabular}{l}
    $s[[\textit{Enabled} \hspace{0.2cm} \mathcal{A}]] \triangleq \exists t \in St : s[[\mathcal{A}]]t$ \\
    $\langle s_0, s_1, \ldots\rangle[[\Box F]] \triangleq \forall n \in Nat : \langle s_n, s_{n+1}, \ldots\rangle[[F]]$ \\
    $\langle s_0, s_1, \ldots\rangle[[\mathcal{A}]] \triangleq s_0[[\mathcal{A}]]s_1$\\
\end{tabular}

\subsection{TLA: Reglas de prueba}
Las reglas de la Lógica Temporal de Acciones son usadas para derivar tautologías temporales La regla STL1-STL6, la Regla de Celosía (Lattice Rule) y las reglas básicas TLA1 y TLA2 constituyen un sistema axiomático independiente para hacer razonamientos, sólido en términos de su semántica. Un sistema es \textit{sólido} si todas las conclusiones que se pueden derivar de él son verdaderas en todos los modelos o interpretaciones que satisfacen sus axiomas. En otras palabras, la solidez garantiza que el sistema no conduce a conclusiones falsas si se parte de premisas verdaderas.

\subsubsection{La primera regla: STL1}
La primera regla, STL1, significa ue $F$ es una tautología proposicional o es derivable por las leyes de la lógica proposicional a partir de fórmulas probables.

\subsubsection{Regla de Celosía (Lattice Rule)}
Esta regla se aplica a un conjunto $S$, que puede ser infinito, y a una función que asocia cada elemento $c$ de $S$ con una fórmula específica $H_c$. Se define un orden parcial $\succ$ sobre $S$ que se considera bien establecido si existe, y solo si hay una secuencia finita descendente tal que $c_0 \succ c_1 \succ \ldots$, con cada $c_i$ perteneciente a $S$ para todo $i \in Nat$. El propósito de la Regla de Celosía es proporcionar una base formal para los argumentos de cuenta regresiva, que son comúnmente empleados en la demostración de la terminación de programas secuenciales.

\subsubsection{Axiomas}
\begin{tabular}{ll}
    \textbf{STL1} & 
    \begin{minipage}{0.5\linewidth}
    \begin{equation*}
        \frac{\text{F provable mediante l.prop.}}{\Box F}
    \end{equation*}
    \end{minipage} \\[10pt]
    \textbf{STL2} & 
    \begin{minipage}{0.5\linewidth}
    \begin{equation*}
        \vdash \Box F \implies F
    \end{equation*}
    \end{minipage} \\[10pt]
    \textbf{STL3} & 
    \begin{minipage}{0.5\linewidth}
    \begin{equation*}
        \vdash \Box\Box F \equiv \Box F
    \end{equation*}
    \end{minipage} \\[10pt]
    \textbf{STL4} & 
    \begin{minipage}{0.5\linewidth}
    \begin{equation*}
        \frac{F \implies G}{\Box F \implies \Box G}
    \end{equation*}
    \end{minipage} \\[15pt]
    \textbf{STL5} & 
    \begin{minipage}{0.5\linewidth}
    \begin{equation*}
        \vdash \Box(F \land G) \equiv (\Box F) \land (\Box G)
    \end{equation*}
    \end{minipage} \\[5pt]
    \textbf{STL6} & 
    \begin{minipage}{0.5\linewidth}
    \begin{equation*}
        \vdash (\Diamond\Box F) \land (\Diamond\Box G) \equiv \Diamond\Box(F \land G)
    \end{equation*}
    \end{minipage}
\end{tabular}

\vspace{0.5cm}

\begin{tabular}{ll}
    \textbf{Celosía.}
    &  $\succ$ orden parcial bien definido sobre $S \neq \emptyset$ \\
    & 
    \begin{minipage}{0.5\linewidth}
    \begin{equation*}
        \frac{F \land (c \in S) \implies (H_c \leadsto (G \lor \exists d \in S : (c \succ d) \land H_d))}{F \implies ((\exists c \in S : H_c) \leadsto G)}
    \end{equation*}
    \end{minipage} \\
\end{tabular}

\subsubsection{Reglas de cuantificación}
\begin{tabular}{ll}
    \textbf{F1} & 
    \begin{minipage}{0.5\linewidth}
    \begin{equation*}
        \vdash F(e/c) \implies \exists c : F
    \end{equation*}
    \end{minipage} \\[10pt]
    \textbf{F2} & 
    \begin{minipage}{0.5\linewidth}
    \begin{equation*}
    \begin{array}{l}
        F \implies G \\
        \text{c no ocurre libre en} \hspace{0.2cm} G \\
        \hline
        (\exists c : F) \implies G
    \end{array}
    \end{equation*}
    \end{minipage}
\end{tabular}

\subsubsection{Reglas básicas de TLA}
\begin{tabular}{ll}
    \textbf{TLA1} & 
    \begin{minipage}{0.5\linewidth}
    \begin{equation*}
        \vdash \Box P \equiv P \land \Box[P \implies P']_P
    \end{equation*}
    \end{minipage} \\[10pt]
    \textbf{TLA2} & 
    \begin{minipage}{0.5\linewidth}
    \begin{equation*}
    \begin{array}{c}
        P \land [\mathcal{A}]_f \implies Q \land \mathcal{B}_g \\
        \hline
        \Box P \land \Box [\mathcal{A}]_f \implies \Box Q \land \Box[\mathcal{B}]_g
    \end{array}
    \end{equation*}
    \end{minipage}
\end{tabular}

\subsubsection{Reglas adicionales}
\begin{tabular}{ll}
    \textbf{INV1} & 
    \begin{minipage}{0.5\linewidth}
    \begin{equation*}
    \begin{array}{c}
        I \land [\mathcal{N}]_f \implies I' \\
        \hline
        I \land \Box [\mathcal{N}]_f \implies \Box I
    \end{array}
    \end{equation*}
    \end{minipage} \\[10pt]
    \textbf{INV2} & 
    \begin{minipage}{0.5\linewidth}
    \begin{equation*}
    \begin{array}{c}
        \vdash \Box I \implies (\Box[\mathcal{N}]_f \equiv \Box [\mathcal{N} \land I \land I']_f)
    \end{array}
    \end{equation*}
    \end{minipage}
\end{tabular}

\vspace{0.25cm}
\begin{tabular}{ll}
    \textbf{WF1} & 
    \begin{minipage}{0.5\linewidth}
    \begin{equation*}
    \begin{array}{l}
        P \land [\mathcal{N}]_f \implies (P' \lor Q') \\
        P \land \langle \mathcal{N} \land \mathcal{A} \rangle_f \implies Q' \\
        P \implies \textit{Enabled} \hspace{0.2cm} \langle A \rangle_f \\
        \hline
        \Box[\mathcal{N}]_f \land WF_f(\mathcal{A}) \implies (P \leadsto Q)
    \end{array}
    \end{equation*}
    \end{minipage} \\[10pt]
    \textbf{WF2} & 
    \begin{minipage}{0.5\linewidth}
    \begin{equation*}
    \begin{array}{l}
        \langle \mathcal{N} \land \mathcal{B} \rangle_f \implies \langle \mathcal{M} \rangle_g \\
        P \land P' \land \langle \mathcal{N} \land \mathcal{A} \rangle_f \implies \mathcal{B} \\
        P \land \textit{Enabled} \hspace{0.2cm} \langle \mathcal{M} \rangle_g \implies \textit{Enabled} \hspace{0.2cm} \langle \mathcal{A} \rangle_f \\
        \Box [\mathcal{N} \land \mathcal{B}]_f \land WF_f(\mathcal{A}) \land \Box F \implies \Diamond \Box P \\
        \hline
        \Box[\mathcal{N}]_f \land WF_f(\mathcal{A}) \land \Box F \implies WF_g(\mathcal{M})
    \end{array}
    \end{equation*}
    \end{minipage}
\end{tabular}

\vspace{0.25cm}
\begin{tabular}{ll}
    \textbf{SF1} & 
    \begin{minipage}{0.5\linewidth}
    \begin{equation*}
    \begin{array}{l}
        P \land [\mathcal{N}]_f \implies (P' \lor Q') \\
        P \land \langle \mathcal{N} \land \mathcal{A} \rangle_f \implies Q' \\
        \Box P \land \Box [\mathcal{N}]_f \land \Box F \implies \Diamond \hspace{0.2cm} \textit{Enabled} \hspace{0.2cm} \langle \mathcal{A} \rangle_f \\
        \hline
        \Box[\mathcal{N}]_f \land SF_f(\mathcal{A}) \land \Box F \implies (P \leadsto Q)
    \end{array}
    \end{equation*}
    \end{minipage} \\[10pt]
    \textbf{SF2} & 
    \begin{minipage}{0.5\linewidth}
    \begin{equation*}
    \begin{array}{l}
        \langle \mathcal{N} \land \mathcal{B} \rangle_f \implies \langle \mathcal{M} \rangle_g \\
        P \land P' \land \langle \mathcal{N} \land \mathcal{A} \rangle_f \implies \mathcal{B} \\
        P \land \textit{Enabled} \hspace{0.2cm} \langle \mathcal{M} \rangle_g \implies \textit{Enabled} \hspace{0.2cm} \langle \mathcal{A} \rangle_f \\
        \Box [\mathcal{N} \land \mathcal{B}]_f \land SF_f(\mathcal{A}) \land \Box F \implies \Diamond \Box P \\
        \hline
        \Box[\mathcal{N}]_f \land SF_f(\mathcal{A}) \land \Box F \implies SF_g(\mathcal{M})
    \end{array}
    \end{equation*}
    \end{minipage}
\end{tabular}

\newpage
El principio de inducción establecido por la regla TLA1 facilita la demostración de la fórmula $\Box P$. Esta regla subraya la idea elemental de que un predicado $P$ permanece cierto en todo momento si inicialmente es verdadero y si en cada etapa subsiguiente, partiendo de una situación donde $P$ es verdadero, este continúa siendo verdadero. Mientras tanto, la validez de TLA2 es inmediata a partir de STL4 y STL5.

La regla INV1 se aplica para confirmar que un programa cumple con una propiedad invariante denotada como $\Box I$. La premisa de esta regla establece que un paso dado por $[\mathcal{N}]_f$ no puede invalidar a $I$. De esto se deduce que si $I$ es verdadero al inicio y cada acción subsiguiente es un paso $[\mathcal{N}]_f$, entonces $I$ se mantiene verdadero de manera constante.

La regla WF1 se utiliza para deducir la propiedad $P \leadsto Q$ a partir de una condición de justicia débil $WF_f(\mathcal{A})$. Por otra parte, la regla WF2 es utilizada pra deducir una condición de justicia débil a partir de otra. Las reglas SF1 y SF2 son análogas a WF1 y WF2, pero para justicia fuerte.

\section{Un ejemplo sencillo de verificación usando TLA}\label{section:TLAexample}
Tras haber explorado los fundamentos teóricos y las características distintivas de la Lógica Temporal de Acciones (TLA), esta sección se dedica a ilustrar cómo se aplica TLA en un contexto práctico. A través de un ejemplo sencillo de verificación, se demostrará el proceso paso a paso para verificar las propiedades de un programa o sistema concurrente utilizando las herramientas y conceptos de TLA.

El siguiente ejemplo denota un simple programa que cuando se ejecuta, permanece incrementando las dos únicas variables $x,y$ inicializadas a 0, eligiendo de forma no determinística cuál incrementar.

\begin{figure}[ht]
\centering
\begin{lstlisting}[basicstyle=\ttfamily\small, frame=single]
    var natural x,y = 0;
    do 
       < true -> x := x + 1 >
       [ ]
       < true -> y := y + 1 >
    od
\end{lstlisting}
\caption{Ejemplo de programa para especificación: Incremento.}
\label{fig:TLAincrement}
\end{figure}

Se procede a ir especificando poco a poco el programa de incremento. Se denotara por $\Phi$ a la fórmula que especifica el programa por completo. En primer lugar, se puede ver que las variables $x,y \in Var$, están inicializadas a $0$, por lo que en el estado inicial del programa se tendrá que $x = 0$, $y = 0$. Esto se puede representar por el predicado $Init_{\Phi}$, que por definición será:

\begin{align*}
    Init_{\Phi} \triangleq (x = 0) \land (y = 0)
\end{align*}

A continuación, como se ha mencionado en la descripción del programa, se elige de forma no determinista cuál de las dos variables incrementar. Esto se puede representar considerando dos acciones, llamémosles $\mathcal{A}_1$ y $\mathcal{A}_2$ que por definición indican:

\begin{align*}
    \mathcal{A}_1 &\triangleq (x' = x + 1) \land (y' = y) \\
    \mathcal{A}_2 &\triangleq (x' = x) \land (y' = y + 1)
\end{align*}

Estos son los únicos dos posibles pasos, que se pueden formular de forma conjunta considerando $\mathcal{A}$ como:

\begin{align*}
    \mathcal{A} \triangleq \mathcal{A}_1 \lor \mathcal{A}_2
\end{align*}

Con esta información, y teniendo en cuenta la notación definida en (5.21), ya se puede especificar el programa con una fórmula TLA:

\begin{align}
    \Phi \triangleq Init_{\Phi} \land \Box [\mathcal{A}]_{\langle x,y \rangle}
\end{align}

Está claro que para que $\Phi$ sea cierta tiene que darse que primero, el estado de inicialización se cumpla (siempre es verdad), y segundo, $\Box [\mathcal{A}]_{\langle x,y \rangle}$, que asegura que todo paso es un $\mathcal{A}$ paso o un paso que deja a las variables sin modificar. Aunque viendo el programa la condición de cambio de una de las dos variables \textit{parece} claro que se hará, el motivo de definirlo así es para ajustarse a las reglas para escribir fórmulas TLA sintácticamente correctas como se indicó en el apartado de sintaxis en ~\ref{subsection:TLASyntax}.

De hecho, la fórmula $\Phi$ tal y como está escrita ahora mismo, es una \textit{propiedad de seguridad}, pues nunca nada malo va a suceder. Lo que queda por hacer es añadir la propiedad de vivacidad para asegurar que el programa se mantiene ejecutándose. Volviendo a tener en cuenta la notación introducida, se considera la forma dual $\langle \mathcal{A} \rangle_{\langle x,y \rangle}$:

\begin{align*}
    \langle \mathcal{A} \rangle_{\langle x,y \rangle} &\triangleq \langle \mathcal{A}_1 \rangle_{\langle x,y \rangle} \lor \langle \mathcal{A}_2 \rangle_{\langle x,y \rangle} \\
    &\triangleq \mathcal{A}_1 \land ((x' \neq x) \land (y' \neq y)) \lor \mathcal{A}_2 \land ((x' \neq x) \land (y' \neq y))
\end{align*}

\noindent
y teniendo en cuenta que $\neg \mathcal{A} \equiv \neg \mathcal{A}_1 \land \neg \mathcal{A}_2$ y que $\neg \Box[\neg \mathcal{A}]_f \equiv \Diamond \langle \mathcal{A} \rangle_f$, $\Phi$ queda:

\begin{align}
    \Phi &\triangleq Init_{\Phi} \land \Box [\mathcal{A}]_{\langle x,y \rangle} \land \Box(\Diamond \langle \mathcal{A}_1 \rangle_{\langle x,y \rangle} \land \Diamond \langle \mathcal{A}_2 \rangle_{\langle x,y \rangle} )\\ \nonumber
    &\triangleq Init_{\Phi} \land \Box [\mathcal{A}]_{\langle x,y \rangle} \land \Box\Diamond \langle \mathcal{A}_1 \rangle_{\langle x,y \rangle} \land \Box\Diamond \langle \mathcal{A}_2 \rangle_{\langle x,y \rangle}
\end{align}

Para aumentar la precisión de la especificación, es más conveniente expresar los términos de vivacidad en términos de justicia (recordando que la justicia es un tipo particular de propiedad de vivacidad). Se reescribirá la última adición realizada en 5.26 para especificar la propiedad de vivacidad que se había comentado que era la de probar que el programa nunca termina. Por ello, se considera ahora el predicado \textit{Enabled} $\langle \mathcal{A}_1 \rangle_{\langle x,y \rangle}$, que es cierta para cualquier tipo de comportamiento $\sigma \in St$ considerado, porque  siempre existe un paso que incrementa en uno la variable $x$ dejando $y$ sin cambios. Como $\Box\neg \top \equiv \bot$, y recordando las definiciones de justicia débil y fuerte, $WF_{\langle x,y \rangle}(\mathcal{A}_1) \equiv \Box\Diamond\langle \mathcal{A}_1\rangle_{\langle x,y\rangle}$ al igual que $SF_{\langle x,y\rangle}(\mathcal{A}_1) \equiv \Box\Diamond\langle \mathcal{A}_1\rangle_{\langle x,y\rangle}$. Un razonamiento análogo se puede realizar para $\Box\Diamond\langle \mathcal{A}_2 \rangle_{\langle x,y \rangle}$.

Como se quería probar que el programa no terminaba, y eso es una condición débil de vivacidad, el programa queda totalmente especificado finalmente por la fórmula:

\begin{align}
    \Phi \triangleq Init_{\Phi} \land \Box [\mathcal{A}]_{\langle x,y \rangle} \land WF_{\langle x,y \rangle}(\mathcal{A}_1) \land WF_{\langle x,y \rangle}(\mathcal{A}_2)
\end{align}

Hay otras propiedades que se pueden demostrar utilizando la Lógica Temporal de Acciones, como por ejemplo, la invarianza del predicado $P$ que se enuncia como sigue:

\begin{align}
    P &\triangleq \text{``x e y son números naturales''} \\
    &\triangleq (x \in Nat) \land (y \in Nat)
\end{align}

\noindent
Esta corrección sobre el programa se puede expresar formalmente como:

\begin{align}
    \Phi \implies \Box P
\end{align}. 

\noindent
La regla INV1 dice que hay que probar:
\begin{align}
    Init_{\Phi} &\implies P \\
    P \land [\mathcal{A}]_{\langle x,y \rangle} &\implies P'
\end{align}

\noindent
La prueba de (5.31) es trivial pues en efecto:

\begin{align*}
    Init_{\Phi} &\triangleq (x = 0) \land (y = 0) \\
    &\implies (x \in Nat) \land (y \in Nat) \\
    &\implies P
\end{align*}
 
La prueba de (5.32) también es sencilla, realizando una descomposición en primer lugar de $[\mathcal{A}]_{\langle x,y \rangle}$:
\begin{equation*}
\begin{array}{rlr}
    [\mathcal{A}]_{\langle x,y \rangle} &\equiv \mathcal{A} \lor (\langle x,y \rangle' = \langle x,y \rangle) & \quad \text{por (5.21)}\\
    &\equiv \mathcal{A}_1 \lor \mathcal{A}_2 \lor (\langle x,y \rangle' = \langle x,y \rangle) & \quad \text{por definición de $ \mathcal{A}$}
\end{array}
\end{equation*}

Ahora faltaría probar cada caso por separado, al ser una disyunción de casos queda:
\begin{align}
    P \land \mathcal{A}_1 \implies P'\\
    P \land \mathcal{A}_2 \implies P'\\
    P \land (\langle x,y \rangle' = \langle x,y \rangle) \implies P'
\end{align}

Se opta por probar uno de los casos, siendo los otros dos totalmente equivalentes. Se decide demostrar (5.34). Primero, se entiende por $P'$ como:

\begin{align}
    P' \equiv (x' \in Nat) \land (y' \in Nat)
\end{align}

\noindent
Por lo que probar (5.34) resulta en probar la veracidad de:
\begin{align}
    P' \land \mathcal{A}_2 \implies x' \in Nat \\
    P' \land \mathcal{A}_2 \implies y' \in Nat
\end{align}

\noindent
Y en el caso de (5.38) (análogo con (5.37)):
\begin{equation*}
\begin{array}{rlr}
    P' \land \mathcal{A}_2 &\implies (y \in Nat) \land (y' = y + 1) & \quad \text{por definición de $P$ y $\mathcal{A}_2$}\\
    &\implies (y' \in Nat) & \quad \text{propiedades de $\mathbb{N}$}
\end{array}
\end{equation*}

Finalmente, tras haber demostrado las condiciones requeridas, ahora se puede deducir (5.30):
\begin{equation*}
\begin{array}{rlr}
    \Phi &\implies Init_{\Phi} \land \Box[\mathcal{A}]_{\langle x,y \rangle} & \quad \text{por definición de $\Phi$}\\
    &\implies P \land \Box[\mathcal{A}]_{\langle x,y \rangle} & \quad \text{por (5.31)}\\
    &\implies \Box P & \quad \text{por (5.32) y INV1}
\end{array}
\end{equation*}

\noindent
En conclusión, queda demostrado que $\Phi$ siempre satisface la propiedad $P$.

        % Análisis del problema (Lenguajes y compiladores)
        \chapter{\textbf{Fundamentos de Lenguajes y Modelos de Computación}}
Este capítulo profundiza en los fundamentos teóricos de los lenguajes y gramáticas, cruciales para el análisis y construcción de lenguajes de programación, los cuales constituyen la piedra angular de este proyecto. Se explorarán también conceptos básicos de modelos de computación, que proporcionan un marco teórico para entender el procesamiento de información y la ejecución de tareas en sistemas computacionales. Estos modelos son esenciales para comprender la implementación e interpretación de lenguajes en entornos computacionales, incluyendo el papel de intérpretes y compiladores.

Finalmente, se dedica una sección a los intérpretes, destacando su relevancia en la ejecución de programas y en la interpretación de lenguajes en un contexto más amplio. Se examinará cómo estos sistemas procesan y ejecutan lenguajes definidos por gramáticas de variados niveles de complejidad.

\section{Alfabetos y palabras}\label{section:MCAlphabet}
El alfabeto, constituido por un conjunto de símbolos básicos, es el punto de inicio para la formación de cualquier lenguaje. La combinación de estos símbolos da lugar a palabras o cadenas, que son los pilares para estructuras lingüísticas más complejas. Las operaciones sobre estas cadenas, como la concatenación y la sustitución, son clave para comprender la construcción y manipulación de los lenguajes.

Formalmente, un \textbf{alfabeto} es un conjunto finito $A$ cuyos elementos se denominan \textbf{símbolos} o \textbf{letras}. En lo que respecta a la notación, se optará por utilizar las primeras letras del abecedario para sendos conceptos, empleando las mayúsculas para los alfabetos y las minúsculas para los símbolos, además de considerar notación subíndice si fuese necesario. 

Por otro lado, una \textbf{palabra} o \textbf{cadena} sobre el alfabeto $A$ es una sucesión finita de elementos de $A$, esto es:

\begin{align*}
    u = a_1 \ldots a_n
\end{align*}

donde $a_i \in A, \forall i \in \mathbb{N}$. Por ejemplo, considerando el alfabeto $A = \left\lbrace a,b,0,1 \right\rbrace$, entonces $u = 001abb1$ es una palabra sobre ese alfabeto. Al conjunto de todas las palabras sobre un alfabeto $A$ se denomina $A^*$. Para las palabras, se convendrá utilizar las últimas letras del abecedario, a saber, $u,v,w,x,y,z$, o letras griegas minúsculas.

Si $u \in A^*$, se entiende por \textbf{longitud} de la palabra $u$ al número de símbolos de $A$ que contiene. La notación empleada para indicar la longitud será $|u|$. Si $u = a_1\ldots a_n$, entonces $|u| = n$. Existe un tipo especial de palabra y es la denominada como \textbf{palabra vacía}, una palabra cuya longitud es 0, y se nombra utilizando el símbolo $\epsilon$. Al conjunto de palabras sobre el alfabeto $A$ donde se elimina la palabra vacía $\epsilon$, se denota por $A^+$.

Habiendo introducido los elementos más básicos, lo que sigue es definir operaciones para su manipulación. Si $u,v \in A^*, u=a_1 \ldots v=b_1\ldots b_m$, se entiende por \textbf{concatenación} de $u,v$ a la cadena $uv$ dada por:

\begin{align*}
    uv = a_1 \ldots a_n b_1 \ldots b_m
\end{align*}

\noindent
Esta operación verifica las siguientes propiedades:
\begin{enumerate}
    \item $|uv| = |u| + |v|, \forall u,v \in A^*$
    \item  Asociatividad: $u(vw) = (uv)w, \forall u,v,w \in A^*$
    \item Elemento neutro: $u\epsilon = \epsilon u = u, \forall u \in A^*$
\end{enumerate}

Finalmente, la construcción de palabras se puede realizar mediante un proceso iterativo. La \textbf{iteración} $n$-ésima de una cadena $u = a_1 \ldots a_n$, referida como $u^n$, indica la concatenación de la palabra $u$ consigo misma $n$ veces. Por otro lado, se denomina como \textbf{cadena inversa} a la palabra $u^{-1}$, resultante de invertir la escritura de sus símbolos, esto es: $u^{-1} = a_n \ldots a_1$.

\section{Lenguajes}\label{section:languages}
Manejar con un conjunto tan grande como puede ser el de todas las palabras sobre un determinado alfabeto $A$ es poco menos que realista, hablando en el sentido computacional \footnote{En particular, $A^*$ es \textbf{siempre} numerable.}. Un \textbf{lenguaje} sobre el alfabeto $A$ es un subconjunto del conjunto de las cadenas sobre $A$, es decir: $L \subseteq A^*$. Se utilizarán las letras $L,M,N$ para denotar lenguajes, además de subíndices si es oportuno. En otras palabras, los lenguajes incluyen reglas específicas en la construcción de palabras concretas sobre un alfabeto $A$. Aquí se presentan algunos ejemplos:

\begin{itemize}
    \item $L_1 = \left\lbrace 0^i 1^i : i = 0,1,2,\ldots \right\rbrace$. Este lenguaje denota una sucesión de ceros seguida de una sucesión de 1 de la misma longitud.

    \item $L_2 = \left\lbrace uu^{-1} : u \in A^* \right\rbrace $. Este lenguaje denota la concatenación de cualquier palabra sobre el alfabeto $A$ concatenada con su inverso, lo que se conoce como \textit{palíndromo}.
\end{itemize}

Sobre los lenguajes también se pueden realizar operaciones, como las de \textbf{intersección} y \textbf{unión} clásicas de los conjuntos. Además, también se puede realizar la \textbf{concatenación} de una forma similar a la que se ha definido para las palabras, así como el lenguaje \textbf{inverso}, o la \textbf{iteración}. Sobre esta última operación se define otra nueva, denominada \textbf{clausura de Kleene} y definida como:

\begin{align*}
    L^* &= \bigcup_{i \geq 0} L^i \\
    L^+ &= \bigcup_{i \geq 1} L^i
\end{align*}

\noindent
considerando que $L$ es un lenguaje sobre el alfabeto $A$.

\section{Gramática}\label{section:gramatica}
\noindent
Una \textbf{gramática generativa} o simplemente \textbf{gramática} es una cuádrupla:

\begin{align*}
    \mathcal{G} = (V,T,P,S)
\end{align*}

\noindent
en la que:

\begin{itemize}
    \item $V$ es un alfabeto de \textbf{variables} o \textbf{símbolos no terminales}. Sus elementos convendrá representarlos con letras mayúsculas.
    \item $T$ es un alfabeto de \textbf{símbolos terminales}. Sus elementos convendrá representarlos con letras minúsculas.
    \item $P$ es un conjunto finito de pares $(\alpha,\beta)$, denomindaos \textbf{reglas de producción}, donde $\alpha,\beta \in (V \cup T)^*$ y $\alpha$ contiene al menos un símbolo de $V$. Al par anteriormente mencionado convendrá representarlos por $\alpha \rightarrow \beta$.
    \item $S$ es un elemento de $V$, llamado \textbf{símbolo de partida}.
\end{itemize}

La idea es que una gramática sirve para determinar un lenguaje. Las palabras son las de $T^*$ que se obtienen a partir del símbolo inicial $S$ efectuando \textit{pasos de derivación}. Sea $\mathcal{G} = (V,T,P,S)$ y $(\alpha,\beta)\in (V \cup T)^*$. Se dice que $\beta$ es \textbf{derivable} a partir de $\alpha$ \textbf{en un paso} (escrito como $\alpha \implies \beta$) si y sólo si existe una producción $\gamma \rightarrow \phi$ tal que:

\begin{enumerate}
    \item $\alpha$ contiene a $\gamma$ como subcadena.
    \item $\beta$ se obtiene sustituyendo $\gamma$ por $\phi$ en $\alpha$.
\end{enumerate}

La definición anterior comprende un \textbf{paso de derivación}. No obstante, se puede decir que $\beta$ es \textbf{derivable} de $\alpha$ (escrito como $\alpha \overset{*}{\implies} \beta$) si y sólo si existe una sucesión de palabras $\gamma_1,\ldots,\gamma_n, (n \geq 1)$ tales que:

\begin{align*}
    \alpha = \gamma_1 \implies \gamma_2 \implies \ldots \implies \gamma_n = \beta
\end{align*}

\subsection{Lenguaje generado}\label{subsection:gramaticalanguage}
Se dice que $L$ es el \textbf{lenguaje generado} por una gramática $\mathcal{G} = (V,T,P,S)$ al conjunto de palabras formadas por símbolos terminales que son derivables partiendo del símbolo inicial $S$. Formalmente:

\begin{align*}
    L(\mathcal{G}) = \lbrace u \in T^* : S \overset{*}{\implies} u \rbrace
\end{align*}

\section{Jerarquía de Chomsky}\label{section:chomsky}
La Jerarquía de Chomsky, propuesta por Noam Chomsky en 1956, es un marco teórico que clasifica los lenguajes formales en distintos niveles según su complejidad gramatical. Esta jerarquía se divide en cuatro categorías: gramáticas regulares, libres de contexto, sensibles al contexto y recursivamente enumerables. Cada nivel representa un grado de complejidad en la generación y el procesamiento de lenguajes, siendo fundamental para entender la teoría de la computación y el desarrollo de lenguajes de programación. La jerarquía de Chomsky queda determinada como sigue:

\begin{itemize}
    \item \textbf{Tipo 0}: Cualquier gramática sin restricciones. Da lugar a \textbf{lenguajes recursivamente enumerables}.
    \item \textbf{Tipo 1}: Todas las producciones tienen la forma:
    \begin{align*}
        \alpha_1 A \alpha_2 \rightarrow \alpha_1 \beta \alpha_2
    \end{align*}
    donde $\alpha_1,\alpha_2,\beta \in (V \cup T)^*, A \in V, \beta \neq \epsilon$, exceptuando la regla $S \rightarrow \epsilon$, en cuyo caso $S$ no aparece a la derecha de las reglas. Esto da lugar a \textbf{lenguajes dependientes del contexto}.
    \item \textbf{Tipo 2}: Cualquier producción tiene la forma:
    \begin{align*}
        A \rightarrow \alpha
    \end{align*}
    donde $A \in V, \alpha \in (V \cup T)^*$. Esto da lugar a \textbf{lenguajes independientes del contexto}.
    \item \textbf{Tipo 3}: Toda regla tiene la forma:
    \begin{align*}
        A \rightarrow uB \hspace{0.2cm} \text{ó} \hspace{0.2cm} A \rightarrow u
    \end{align*}
    donde $u \in T^*; A,B \in V$. Da lugar a \textbf{lenguajes regulares}.
\end{itemize}

La clase o familia de lenguajes de los tipos anteriormente mencionados se denota por $\mathcal{L}_i, i = 0,1,2,3$. Además, se verifica la siguiente cadena de inclusiones:

\begin{align*}
    \mathcal{L}_3 \subseteq \mathcal{L}_2 \subseteq \mathcal{L}_1 \subseteq \mathcal{L}_0
\end{align*}

Para este proyecto es esencial considerar los lenguajes independientes de contexto, que son los que constituyen lenguajes de progamación. También los lenguajes regulares, útiles para el reconocimiento de cadenas.

\section{Expresiones regulares}\label{section:expr}
Las expresiones regulares son una herramienta teórica fundamental en el estudio de los lenguajes formales, específicamente los lenguajes regulares. Se utilizan para describir de manera sintética y precisa los patrones y estructuras que conforman estos lenguajes. A través de una serie de símbolos y operadores, las expresiones regulares permiten representar conjuntos infinitos de cadenas y facilitan el análisis y clasificación de estos lenguajes en la teoría de la computación. Su estudio es esencial para comprender cómo se pueden definir y reconocer los lenguajes regulares, que son la base de modelos más complejos en la informática y la lingüística teórica.

Si $A$ es un alfabeto, una \textbf{expresión regular} sobre ese alfabeto se define de la siguiente forma:

\begin{itemize}
    \item $\emptyset$ es una expresión regular que denota el lenguaje vacío.
    \item $\epsilon$ es una expresión regular que denota el lenguaje $\lbrace \epsilon \rbrace$.
    \item Si $a \in A$, \textbf{a} es una expresión regular que denota el lenguaje $\lbrace a \rbrace$.
    \item Si \textbf{r,s} son expresiones regulares que denotan los lenguajes $R,S$ respectivamente, se definen las operaciones:
    \begin{itemize}
        \item \textbf{Unión}: \textbf{(r + s)} es una expresión regular que denota el lenguaje $R \cup S$.
        \item \textbf{Concatenación}: \textbf{rs} es una expresión regular que denota el lenguaje $RS$.
        \item \textbf{Clausura}: \textbf{r$^*$} es una expresión regular que denota el lenguaje $R^*$.
    \end{itemize}
\end{itemize}

Sean $r,r_1,r_2$ expresiones regulares. Algunas de las propiedades más importantes de las expresiones regulares son las siguientes:
\vspace{0.5cm}
\newline
\begin{tabular}{ll}
    $r_1 + r_2 = r_2 + r_1$ & $r_1(r_2+r_3) = r_1r_2 + r_1r_3$ \\
    $r_1 + (r_2 + r_3) = (r_1 + r_2) + r_3$ & $(r_1+r_2)r_3 = r_1r_3 + r_2r_3$ \\
    $r_1(r_2r_3) = (r_1r_2)r_3$ & $r^+ + \epsilon = r^*$ \\
    $r\epsilon = r$ & $r^* + \epsilon = r^*$ \\
    $r\emptyset = \emptyset$ & $(r+\epsilon)^* = r^*$ \\
    $r+\emptyset = r$ & $(r+\epsilon)^+ = r^*$ \\
    $\epsilon^* = \epsilon$ & $(r_1^*+r_2^*)^* = (r_1+r_2)^*$ \\
\end{tabular}
\vspace{0.5cm}

\noindent
Algunos ejemplos de expresiones regulares son los siguientes:
\begin{itemize}
    \item $(0+1)^*$: Representa una secuencia de ceros y unos en cualquier combinación, incluyendo la palabra vacía, denotada como $\epsilon$. Ejemplos de palabras aceptadas por esta expresión incluyen $011101$, $1$, $000$, y $\epsilon$.
    \item $a(bb)^*c+d$: Esta expresión puede interpretarse de dos maneras principales. Primero, como una secuencia que comienza con un símbolo $a$, seguido de cualquier número (incluido cero) de pares de $b$, y terminando con un $c$. Alternativamente, puede ser simplemente la letra $d$. Ejemplos de palabras aceptadas incluirían $abbc$, $ac$, $abbbbc$, y $d$.
\end{itemize}



\section{Computación de lenguajes. Autómatas}\label{section:automat}
Esta sección explora el papel fundamental de los autómatas en la computación de lenguajes. Los autómatas finitos, que son esenciales en el reconocimiento de lenguajes regulares, actúan como modelos simplificados de computación. Estos se representan mediante grafos conocidos como diagramas de transición de estados, que ilustran cómo un sistema cambia de un estado a otro en respuesta a entradas específicas.

Más allá de los autómatas finitos, se introducen también los autómatas con pila, que son cruciales para el reconocimiento de lenguajes independientes del contexto. Estos autómatas se caracterizan por su capacidad de almacenar una cantidad de información adicional gracias a su ``pila'', lo que les permite procesar estructuras más complejas que las que pueden manejar los autómatas finitos. La habilidad de los autómatas con pila para manejar gramáticas libres de contexto los hace particularmente importantes en el análisis sintáctico de lenguajes de programación y en la comprensión de estructuras más complejas en la informática teórica.

\subsection{Autómatas finitos}\label{subsection:AF}
La función de los autómatas finitos es la de \textit{reconocer} un determinado patrón a partir de una palabra de entrada. Pueden ser de dos tipos:

\begin{itemize}
    \item \textbf{Autómatas Finitos no Deterministas (AFND)}. Estos autómatas no presentan limitaciones en las etiquetas de sus transiciones. Es posible que un mismo símbolo etiquete múltiples transiciones originadas en un estado idéntico, y se permite el uso de $\epsilon$ como etiqueta.
    \item \textbf{Autómatas Finitos Deterministas (AFD)}. Cada estado en estos autómatas, para cada símbolo del alfabeto de entrada, cuenta con una única transición correspondiente a ese símbolo que se origina en dicho estado.
\end{itemize}

\noindent
Formalmente, se define un autómata se define como una quíntupla:

\begin{align*}
    M = (Q,A,\delta,q_0,F)
\end{align*}

\noindent
donde:
\begin{itemize}
    \item $Q$ es un conjunto finito llamado \textbf{conjunto de estados}.
    \item $A$ es un alfabeto llamado \textbf{alfabeto de entrada}.
    \item $\delta$ es una aplicación llamada \textbf{función de transición}.
    \item $q_0$ es un elemento de $Q$ llamado \textbf{estado inicial}.
    \item $F$ es un subconjunto de $Q$, llamado \textbf{conjunto de estados finales}.
\end{itemize}

La diferencia entre un AFND con transiciones nulas y un AFD es en la definición de la función de transición.

Una propiedad interesante en el estudio de los autómatas y los lenguajes formales es la equivalencia entre autómatas finitos no deterministas con transiciones nulas (AFND) y autómatas finitos deterministas (AFD). Específicamente, un lenguaje $L$ puede ser aceptado por un AFND con transiciones nulas si y sólo si existe un AFD que también acepta $L$. El lenguaje aceptado por un autómata $M$ se denota como $L(M)$. Profundizando en esta relación, una característica notable es que un lenguaje $L$ es aceptado por un AFD si y sólo si puede ser descrito mediante una expresión regular. Finalmente, cabe destacar que todo AFND con transiciones nulas puede ser convertido en un AFD equivalente, lo que demuestra una correspondencia fundamental entre estos dos tipos de autómatas en la teoría de lenguajes formales. Hay varios algoritmos de conversión, algunos de ellos mencionados en \cite{aho1990compiladores} (sección 3.7).

\subsubsection{Generación de analizadores léxicos}\label{subsubsection:}
La generación de analizadores léxicos es un imprescindible en el análisis de lenguajes de programación, donde las expresiones regulares y los autómatas juegan un papel importantísimo. Estos analizadores, diseñados para reconocer patrones léxicos en el código fuente, son a menudo construidos utilizando herramientas como Flex, que convierte expresiones regulares en autómatas finitos. Flex simplifica este proceso al permitir la definición de patrones léxicos a través de expresiones regulares, que luego son automáticamente transformadas en un AFND y posteriormente en un AFD para el análisis eficiente del texto de entrada. Este enfoque aprovecha la teoría de autómatas y las propiedades de las expresiones regulares para crear sistemas capaces de descomponer y entender la estructura léxica de los lenguajes de programación.

\subsection{Autómatas con pila}\label{subsection:automatPila}
En la sección ~\ref{section:chomsky} se ha hablado acerca de las gramáticas independientes del contexto. Estas gramáticas, que generan lenguajes más complejos que los lenguajes regulares, requieren un mecanismo de análisis más avanzado para su procesamiento. Aquí es donde entran en juego los autómatas con pila, una forma extendida de autómatas que son capaces de manejar esta complejidad adicional.

Los autómatas con pila son una clase de autómatas que, a diferencia de los autómatas finitos, cuentan con una memoria adicional en forma de una pila. Esta característica les permite no solo procesar la entrada actual, sino también almacenar y recuperar información, lo que es esencial para manejar dependencias a largo plazo y estructuras anidadas típicas de los lenguajes independientes del contexto. Por lo tanto, son herramientas fundamentales para el análisis sintáctico en compiladores y para el entendimiento profundo de la computación y procesamiento de lenguajes más complejos.



\noindent
Formalmente, un \textbf{Autómata con Pila} es una séptupla:

\begin{align*}
    M = (Q,A,B,\delta,q_0,Z_0,F)
\end{align*}

\noindent
donde:
\begin{itemize}
    \item $Q$ es un conjunto finito llamado \textbf{conjunto de estados}.
    \item $A$ es un alfabeto llamado \textbf{alfabeto de entrada}.
    \item $B$ es un alfabeto llamado \textbf{alfabeto de pila}.
    \item $\delta$ es una aplicación llamada \textbf{función de transición}.
    \item $q_0$ es un elemento de $Q$ llamado \textbf{estado inicial}.
    \item $Z_0 \in B$ es el \textbf{símbolo inicial de pila}
    \item $F$ es un subconjunto de $Q$, llamado \textbf{conjunto de estados finales}.
\end{itemize}

También hay versiones deterministas y no deterministas, donde la diferencia reside en las condiciones a imponer sobre la función de transición. Un lenguaje independiente del contexto se dice que es \textbf{determinista} si y sólo si es aceptado por un autómata con pila determinista por el \textit{criterio de estados finales}. Este criterio implica que un autómata con pila determinista acepta una cadena de entrada si, y solo si, al finalizar el procesamiento de la cadena, el autómata llega a un estado final y la pila está vacía. En otras palabras, la aceptación de una cadena no solo depende de alcanzar un estado final, sino también de que la pila se haya vaciado completamente al final del procesamiento. Esto asegura que el autómata no solo reconoce la secuencia de símbolos de la cadena de entrada, sino que también cumple con las restricciones adicionales impuestas por la estructura de la pila.

\subsubsection{Generación de analizadores sintácticos}\label{subsubsection:analizadoresyntax}
La utilidad de los autómatas con pila se extiende más allá de la teoría y encuentra una aplicación práctica significativa en la generación de analizadores sintácticos. El proceso de análisis sintáctico implica reconocer patrones y estructuras sintácticas más complejas, una tarea para la cual los autómatas con pila están especialmente equipados. Gracias a su capacidad de almacenar y manejar información contextual en la pila, pueden eficientemente analizar gramáticas independientes del contexto, que son típicas en la mayoría de los lenguajes de programación modernos.

Herramientas como Yacc (Yet Another Compiler-Compiler) y su sucesor Bison, utilizan la teoría de los autómatas con pila para generar analizadores sintácticos. Estas herramientas toman una especificación de gramática y producen un analizador que puede descomponer y analizar estructuras lingüísticas complejas, siguiendo las reglas definidas en la gramática. Al igual que con los analizadores léxicos, el uso de autómatas con pila simplifica enormemente el proceso de desarrollo de software, permitiendo a los programadores y desarrolladores de lenguajes centrarse en la definición de reglas gramaticales, mientras la herramienta maneja los detalles del análisis sintáctico.

\section{Herramientas de procesamiento de lenguajes. Compiladores e intérpretes}\label{section:compiladores}
Lo acontecido en secciones anteriores en este capítulo ofrecen un marco teórico más que suficiente para el desarrollo de los siguientes capítulos que se centran en la parte práctica del proyecto. En esta última sección se hablará de herramientas prácticas para el procesamiento de lenguajes: los \textit{compiladores} e \textit{intérpretes}. Estas herramientas tienen como objetivo convertir el código fuente escrito en un lenguaje de programación a un formático que la máquina pueda ejecutar o interpretar directamente.

Un \textbf{compilador} es un programa que traduce código fuente escrito en lenguaje de alto nivel a lenguaje máquina o a un código intermedio. El proceso de compilación consta de varias fases:

\begin{enumerate}
    \item \textbf{Análisis léxico}: 
        \begin{itemize}
            \item El compilador lee el código fuente y lo descompone en tokens.
            \item Los tokens pueden ser identificadores, palabras clave, constantes, operadores, etc.
            \item Se simplifica y estructura el código fuente para las fases posteriores.
        \end{itemize}

    \item \textbf{Análisis sintáctico}:
        \begin{itemize}
            \item Organiza los tokens en un árbol sintáctico que representa la estructura gramatical.
            \item Verifica que la secuencia de tokens siga las reglas gramaticales del lenguaje.
            \item Identifica errores de sintaxis como paréntesis faltantes o errores en construcciones de bucles.
        \end{itemize}

    \item \textbf{Análisis semántico}:
        \begin{itemize}
            \item Verifica la corrección semántica del código, asegurando que los elementos del programa tengan sentido en su contexto.
            \item Incluye la verificación de tipos de datos y la coherencia en el uso de variables y funciones.
            \item Detecta errores como asignaciones de tipos de datos incorrectos o llamadas incorrectas a funciones.
        \end{itemize}

    \item \textbf{Generación de código intermedio}:
        \begin{itemize}
            \item Transforma el árbol sintáctico en una representación intermedia.
            \item Esta representación es independiente del lenguaje de programación y del hardware.
            \item Facilita la optimización del código y prepara la generación del código máquina.
        \end{itemize}

    \item \textbf{Optimización de código}:
        \begin{itemize}
            \item Mejora la representación intermedia para aumentar la eficiencia del programa.
            \item Incluye la eliminación de código inaccesible y la optimización de bucles.
            \item Esencial para mejorar el rendimiento y la eficiencia del código compilado.
        \end{itemize}

    \item \textbf{Generación de código máquina}:
        \begin{itemize}
            \item Convierte la representación intermedia en código de máquina ejecutable por el procesador.
            \item El código generado está optimizado para el hardware específico de destino.
            \item Completa el proceso de traducción resultando en un programa ejecutable o archivo objeto.
        \end{itemize}
\end{enumerate}


A diferencia de los compiladores, un \textbf{intérprete} no genera un archivo de salida ejecutable. En su lugar, leen y ejecutan el código fuente directamente, traduciendo el programa a medida que se ejecuta. Esto permite una mayor flexibilidad y una iteración más rápida durante el desarrollo, aunque puede tener un rendimiento más lento en comparación con los programas compilados. Los intérpretes también utilizan técnicas de análisis léxico y sintáctico para entender el código fuente, pero ejecutan las instrucciones inmediatamente después de su análisis.

También hay una variante que combina ambos enfoques, que es lo que se denomina como \textbf{compilador híbrido}. En la práctica en este proyecto, aunque se dispone de una fase de compilación que traduce código del lenguaje a código intermedio, su funcionamiento es más semejante al de un \textit{intérprete}, al traducir inmediatamente dichas instrucciones intermedias.

        % Desarrollo bajo sprints: 
	\chapter{\textbf{Estudio y diseño del lenguaje Lamport}}
El lenguaje \textit{Lamport}, nombrado en honor al renombrado informático Leslie Lamport, emerge como una herramienta didáctica y a la vez completamente funcional. Su propósito es modelar, diseñar y simular sistemas concurrentes y distribuidos.



En este capítulo, se llevará a cabo un estudio detallado sobre el pseudocódigo propuesto en la asignatura \textit{Sistemas Concurrentes y Distribuidos}, que sustentará la base del lenguaje destino \textit{Lamport}. Se discutirán sus objetivos de diseño, principales características y cómo estas contribuyen a abordar los desafíos de modelar y simular sistemas concurrentes. Se profundizará en aspectos como la sintaxis, semántica y otras características esenciales. Finalmente, se presentará una definición formal de la gramática del lenguaje Lamport, orientada a ser un recurso valioso para quienes aspiren a aprender o instruir sobre sistemas concurrentes, e incluso utilizarlo como un lenguaje de programación más.

\section{Análisis del pseudocódigo base}\label{sec:pseudoAnalisis}
El pseudocódigo, una representación abstracta y simplificada de la lógica de programación, es un elemento crucial en la fase de diseño de sistemas informáticos. En la asignatura \textit{Sistemas Concurrentes y Distribuidos}, se propone un pseudocódigo específico que sienta las bases para la concepción del lenguaje \textit{Lamport}. Analizar este pseudocódigo nos permite entender su estructura, características y potenciales áreas de mejora o adaptación.



A continuación se presentarán una serie de ejemplos representativos del pseudocódigo, ilustrando sus principales componentes. Posteriormente, se desglosarán sus elementos fundamentales y se considerará su relación con lenguajes de programación más convencionales, permitiendo así contextualizar su diseño y utilidad.

\newpage

\subsection{Ejemplos de pseudocódigo}\label{subsec:pseudoAnalisisEjemplos}

Los ejemplos que se presentan a continuación ofrecen una vista panorámica sobre la utilización del pseudocódigo en diferentes contextos. Cada uno ilustra un escenario particular en la modelación y simulación de sistemas concurrentes y distribuidos, haciendo uso de las estructuras, declaraciones y procesos propios del pseudocódigo. Estos ejemplos servirán como base para el análisis detallado de los componentes sintácticos y semánticos que se llevará a cabo en las subsecciones posteriores.

\subsubsection{Ejemplo 1: Paradigma del Productor-Consumidor}\label{subsubsec:pseudoAnalisisEjemplo1}
Este ejemplo muestra la interacción entre dos procesos cooperantes en los cuales uno de ellos \textit{productor} genera una secuencia de valores (como por ejemplo, enteros) y el otro \textit{consumidor} utiliza cada uno de estos valores.

\begin{figure}[h]
\begin{lstlisting}[style=lamportStyle]
var x : integer; {contiene cada valor producido}

{Proceso productor: calcula 'x'}
process Productor;
var a : integer; {no compartida}
begin
  while true do begin
    {calcular un valor}
    a := ProducirValor();
    {escribir en mem. compartida}
    x := a; {sentencia E}
  end
end

{Proceso consumidor: lee 'x'}
process Consumidor;
var b : integer;
begin
  while true do begin
    {leer de mem. compartida}
    b := x; {sentencia L}
    {utilizar el valor leido}
    UsarValor(b);
  end
end
\end{lstlisting}
\caption{Ejemplo del problema del productor-consumidor en pseudocódigo.}
\label{fig:ejemplo1}
\end{figure}

\newpage


A partir de este ejemplo, es posible identificar componentes y patrones que serán importantes para el diseño del código \textit{Lamport}. A continuación, se enumeran y detallan algunas de las características más relevantes:

\begin{itemize}
    \item \textbf{Variables}: Estos elementos, presentes en las líneas 1, 5 y 17 y con nombres \code{x}, \code{a} y \code{b} respectivamente, son fundamentales en la mayoría de los lenguajes de programación. Las variables permiten almacenar y manipular datos de diferentes tipos.
    
    \item \textbf{Ámbitos de las Variables (Scopes)}: En este pseudocódigo, es posible distinguir entre variables globales y locales. La variable \code{x} es global y, por lo tanto, es accesible desde cualquier proceso del programa. Por otro lado, las variables \code{a} y \code{b} son locales a sus respectivos procesos, lo que significa que solo son conocidas y manipulables dentro del proceso en el que se declaran.

    \item \textbf{Tipos de dato de Variables}: Se refiere a la categoría de dato que una variable puede almacenar. En el pseudocódigo, vemos que se hace uso del tipo de dato \code{integer} para representar números enteros. Esto se aprecia en las declaraciones de las variables en las líneas 1, 5 y 17. Definir el tipo de dato es crucial, ya que delimita las operaciones que se pueden realizar con la variable y la cantidad de memoria que se reserva para ella. Así, cuando se asigna un valor a una variable, el sistema sabe cómo interpretar y manipular ese dato en función de su tipo.

    \item \textbf{Procesos}: Son unidades fundamentales de ejecución en este pseudocódigo. Cada proceso encapsula un conjunto de instrucciones que se ejecutan de manera secuencial o concurrente. En el ejemplo anterior, se identifican dos procesos: el \code{Productor} (definido a partir de la línea 4) y el \code{Consumidor} (definido a partir de la línea 16). Estos procesos tienen sus propias variables locales y pueden interactuar con variables globales. La palabra clave \code{process} indica el comienzo de la definición de un proceso, y todo lo que sigue, hasta el \code{end} correspondiente, pertenece a ese proceso. Estos procesos pueden ejecutarse de manera concurrente o en paralelo, dependiendo del contexto del programa.

    \item \textbf{Bucles While}: El bucle \code{while} es una estructura de control que permite repetir un conjunto de instrucciones mientras una condición sea verdadera. En el pseudocódigo presentado, observamos bucles \code{while} en los procesos \code{Productor} y \code{Consumidor} (a partir de las líneas 7 y 19, respectivamente). Aquí, el conjunto de instrucciones dentro del bucle se ejecuta indefinidamente debido a la condición \code{true}. Este tipo de bucles es importante para modelar operaciones que deben continuar hasta que se cumpla una condición específica o, como en este caso, para operaciones que deben continuar en un ciclo perpetuo.

    \item \textbf{Llamadas a funciones y procedimientos}: Estas estructuras son vitales para organizar el código, permitiendo la reutilización y modularidad. Una llamada a función o procedimiento invoca un conjunto específico de instrucciones predefinidas. En el pseudocódigo presentado, encontramos ejemplos de llamadas a funciones con \code{ProducirValor()} (línea 9) y \code{UsarValor(b)} (línea 23). La diferencia principal entre funciones y procedimientos es que las funciones generalmente devuelven un valor, mientras que los procedimientos realizan una acción sin necesariamente devolver algo. Estas llamadas sirven para simplificar el código y evitar redundancias, permitiendo una estructuración más clara y organizada del programa.

    \item \textbf{Operaciones de Asignación}: Una de las operaciones fundamentales en cualquier lenguaje de programación es la capacidad de asignar valores a variables. Las asignaciones permiten que los programas capturen y modifiquen estados dinámicamente durante su ejecución. En el pseudocódigo presentado, observamos asignaciones en como \code{a := ProducirValor();} (línea 9) y \code{x := a;} (línea 21). El operador \code{:=} es utilizado para denotar la acción de asignar el valor del lado derecho al identificador del lado izquierdo. La correcta gestión de asignaciones es esencial para garantizar que el programa funcione de manera adecuada y predecible, especialmente en contextos concurrentes donde las operaciones de lectura y escritura deben ser cuidadosamente coordinadas.

    \item \textbf{Comentarios}: Los comentarios son fragmentos de texto que se incluyen en el código con el propósito de proporcionar explicaciones adicionales o aclaraciones sobre la lógica o el propósito del código, pero que no se ejecutan como parte del programa. En el pseudocódigo presentado, los comentarios están demarcados por llaves \code{\{\}}. Por ejemplo, en la línea 1, \code{\{contiene cada valor producido\}} es un comentario que ofrece información sobre la variable \code{x}. Los comentarios son útiles para garantizar la legibilidad y comprensión del código, especialmente cuando se comparte con otros desarrolladores o para referencia futura.
\end{itemize}

\subsubsection{Ejemplo 2: Ordenación de Arrays utilizando procedimientos}\label{subsubsec:pseudoAnalisisEjemplo2}
El siguiente ejemplo ilustra cómo se pueden ordenar y copiar arrays utilizando procedimientos en el pseudocódigo. A través de funciones como \code{Sort} y \code{Copiar}, se demuestra cómo la modularidad y la reutilización de código son posibles, lo que mejora la legibilidad y la estructura del programa.

\newpage

\begin{figure}[h]
\begin{lstlisting}[style=lamportStyle]
var a,b : array[1..2*n] integer; {n es una constante predefinida}

procedure Sort( s,t : integer );
  var i,j : integer;
begin
  for i := s to t do
    for j := s+1 to t do
      if a[i] < a[j] then
        swap(a[i], b[j]);
end

procedure Copiar( o,s,t : integer );
  var d : integer;
begin
  for d := 0 to t-s do
    b[o+d] := a[s+d];
end

procedure Secuencial();
   var i : integer;
begin
   Sort(1, 2*n); {ordena a}
   Copiar(1, 2*n); {copia a en b}
end
\end{lstlisting}
\caption{Ejemplo de ordenación y copia de arrays con procedimientos en pseudocódigo.}
\label{fig:ejemplo2}
\end{figure}

A partir de este ejemplo también podemos encontrar una serie de características importantes para la definición del lenguaje \textit{Lamport}:

\begin{itemize}
    \item \textbf{Definición de Arrays}: Los arrays \code{a} y \code{b} se definen con tamaños basados en una constante predefinida \code{n}. Estos arrays almacenan valores enteros y sirven como ejemplos principales para las operaciones de ordenación y copia.

    \item \textbf{Definición de Procedimientos}: Los procedimientos como \code{Sort} (línea 3), \code{Copiar} (línea 12) y \code{Secuencial}, (línea 19) demuestran cómo se pueden encapsular tareas específicas y reutilizarlas a lo largo del código.

    \item \textbf{Bucles For}: Los bucles son una herramienta esencial para iterar sobre colecciones de datos y realizar operaciones repetitivas. El uso de bucles \code{for} en este ejemplo muestra cómo se pueden manejar conjuntos de datos de manera eficiente.

    \item \textbf{Estructuras Condicionales (If)}: Las estructuras condicionales, como el \code{if} (línea 8), son prácticamente imprescindibles en la programación. Permiten que un programa evalúe condiciones y tome decisiones basadas en esas evaluaciones. Esta capacidad es fundamental para controlar el flujo de ejecución y adaptar el comportamiento del programa según diferentes situaciones o entradas.
\end{itemize}

\subsubsection{Ejemplo 3: Concurrencia entre instrucciones de un proceso}\label{subsubsec:pseudoAnalisisEjemplo3}
En este ejemplo, exploramos una prueba sencilla de una forma de representación de la concurrencia en el pseudocódigo. Se introduce el concepto de \code{cobegin} y \code{coend}, que marcan el inicio y el fin de un bloque de código que se ejecuta concurrentemente. El código dentro de este bloque puede considerarse como una serie de operaciones que tienen la capacidad de ejecutarse simultáneamente, lo que implica que no hay un orden garantizado de ejecución entre las sentencias que se encuentran en dicho bloque.

\begin{figure}[h]
\begin{lstlisting}[style=lamportStyle]
process P;
var x : integer := 0;
cobegin
  x := x+1 ; x := x+2;
coend
\end{lstlisting}
\caption{Ejemplo simple de programa concurrente.}
\label{fig:ejemplo3}
\end{figure}

También tendremos en cuenta este ejemplo para identificar algunos aspectos del lenguaje \textit{Lamport}, detallándolos a continuación:

\begin{itemize}
    \item \textbf{Instrucciones Concurrentes}: El bloque que comienza con \code{cobegin} y termina con \code{coend} (desde línea 3 hasta línea 5) se utiliza para indicar la ejecución concurrente de instrucciones. Dentro de este bloque, las instrucciones se ejecutan de manera concurrente, lo que significa que no hay un orden garantizado entre ellas.
\end{itemize}

\subsubsection{Ejemplo 4: Instrucciones compuestas atómicas}\label{subsubsec:pseudoAnalisisEjemplo4}

En este ejemplo, se ilustra la ejecución de instrucciones compuestas atómicas. En el contexto del pseudocódigo y especialmente en programación concurrente, la atomicidad asegura que las operaciones se ejecutan completamente o no se ejecutan en absoluto, sin posibilidad de interrupción. Es una característica crucial para evitar condiciones de carrera y garantizar la coherencia de los datos.

\begin{figure}[h]
\begin{lstlisting}[style=lamportStyle]
begin
  x := 0;
  cobegin
    < x := x+1 >
    < x := x-1 >
  coend
end
\end{lstlisting}
\caption{Ejemplo de programa con instrucciones atómicas.}
\label{fig:ejemplo4}
\end{figure}

\newpage 

Se tendrán en cuenta las siguientes características a partir de este ejemplo:

\begin{itemize}
    \item \textbf{Instrucciones Atómicas}: Las instrucciones rodeadas por \textless \textgreater \hspace{0.1cm} representan operaciones atómicas. Estas instrucciones no pueden ser interrumpidas y se ejecutan completamente sin interferencia externa. En este caso, \code{x := x+1} (linea 4) y \code{x := x-1} (linea 5) se ejecutarán de manera atómica.
\end{itemize}

\subsubsection{Ejemplo 5: Creación de procesos no estructurada. Fork-Join.}\label{subsubsec:pseudoAnalisisEjemplo5}
En este cuarto ejemplo, exploramos una característica interesante del pseudocódigo: la creación dinámica de procesos y su sincronización. En muchos sistemas, especialmente en entornos paralelos o distribuidos, la habilidad de crear procesos en tiempo de ejecución y sincronizarlos correctamente es esencial. Veamos cómo \textit{Lamport} aborda esto a través de los comandos \code{fork} y \code{join}.

\begin{figure}[h]
\begin{lstlisting}[style=lamportStyle]
procedure P1;
begin
  A;
  fork P2;
  B;
  join P2;
  C;
end

procedure P2;
begin
  D;
end;
\end{lstlisting}
\caption{Ejemplo de creación dinámica de procesos y sincronización en pseudocódigo.}
\label{fig:ejemplo5}
\end{figure}

Las características más importantes a considerar en este ejemplo son las siguientes:

\begin{itemize}
    \item \textbf{Creación Dinámica de Procesos}: En la línea 3, se utiliza la sentencia \code{fork}, que indica la creación de un nuevo proceso en tiempo de ejecución. En este caso, se lanza el proceso \code{P2}.
    
    \item \textbf{Sincronización de Procesos}: Tras la ejecución de ciertas instrucciones, hay que comprobar que los procesos se sincronicen para garantizar la coherencia. En la línea 5, la sentencia \code{join} asegura que el proceso \code{P1} espere a que \code{P2} termine antes de continuar.
    
    \item \textbf{Orden de Ejecución}: Aunque esto no es una característica explícita de la sintaxis del lenguaje, es importante para comprender el funcionamiento de las sentencias anteriores. Observando el flujo de ejecución, primero se ejecuta la instrucción \code{A} en \code{P1}, luego se inicia \code{P2}, y después de eso, \code{P1} ejecuta la instrucción \code{B}. Sin embargo, antes de que \code{P1} pueda continuar con \code{C}, debe esperar a que \code{P2} haya completado su ejecución.
\end{itemize}

\subsubsection{Ejemplo 6: Definición estática de vectores de procesos.}\label{subsubsec:pseudoAnalisisEjemplo6}
En este ejemplo se muestra una característica importante en la definición y manejo de procesos: la definición estática de vectores de procesos. Esto permite tener un conjunto de procesos con características similares, pero con un identificador específico (usualmente un índice) que permite individualizar su comportamiento. Esta estructura es especialmente útil en entornos donde se requiere un comportamiento similar, pero ligeramente diferente, para un conjunto de tareas o procesos.

\begin{figure}[h]
\begin{lstlisting}[style=lamportStyle]
var ... {variables compartidas}

process NomP[ind : a..b];
var ... {variables locales}
begin
  ... {codigo}
  ... {ind vale a, a+b, ..., b}
end
\end{lstlisting}
\caption{Ejemplo de definición estática de procesos en pseudocódigo.}
\label{fig:ejemplo6}
\end{figure}

Aquí, las características más importantes a considerar son las siguientes:

\begin{itemize}
    \item \textbf{Vectores de Procesos}: La estructura \code{process NomP[ind : a..b]} permite definir un conjunto o vector de procesos, que van desde el índice \code{a} hasta el índice \code{b}. Cada proceso individual puede ser referenciado por su índice específico, que es \code{ind}.

\end{itemize}

\newpage

Tras revisar diversos ejemplos del pseudocódigo \textit{Lamport}, es evidente la flexibilidad y adaptabilidad que este tipo de representación ofrece. No obstante, el pseudocódigo, por su naturaleza intrínseca, carece de una consistencia estricta. Aunque es valioso para transmitir ideas y lógica de programación de forma intuitiva, su falta de rigidez puede llevar a ambigüedades o interpretaciones múltiples. Es aquí donde radica la importancia de definir un lenguaje convencional con una estructura y gramática claramente establecidas. En la siguiente sección, abordaremos el desafío de formalizar el lenguaje \textit{Lamport} a través de una gramática bien definida, sentando así las bases para un entendimiento unificado y coherente de su estructura y funcionamiento.

\section{La gramática del lenguaje Lamport}\label{sec:gramaticaLamport}
En esta sección, se desglosará la estructura fundamental del lenguaje \textit{Lamport}. Se comenzará por definir los tokens, que son los símbolos terminales que forman las unidades básicas del lenguaje. Posteriormente, para la descripción detallada de la gramática, se recurrirá a la notación BNF (Backus-Naur Form), una herramienta valorada por su claridad y precisión en la definición de gramáticas de lenguajes de programación. Más allá de la estructura sintáctica, también se abordará una descripción semántica del lenguaje, garantizando un entendimiento profundo de cómo se traducen las instrucciones en comportamientos específicos. Así, se proporcionará una visión completa de las capacidades y limitaciones de \textit{Lamport}, estableciendo una base sólida para el desarrollo de aplicaciones y sistemas concurrentes.

\subsection{Componentes del lenguaje}\label{subsec:componentesLamport}

Antes de describir formalmente la gramática de Lamport, es menester indicar que el lenguaje está compuesto por una serie de componentes clave que definen su estructura y funcionalidad. A continuación, se describen estos componentes, dejando este convenio de nombres para cuando se necesite en sucesivos capítulos o secciones posteriores:

\begin{itemize}
    \item \textbf{Declaraciones:} Son instrucciones que definen variables, constantes u otros elementos identificables, reservando memoria o estableciendo valores iniciales. Por ejemplo, la declaración de una variable puede ser \code{int x;} donde se está reservando espacio para un entero llamado x.

    \item \textbf{Expresiones:} Son combinaciones de constantes, variables, operadores y funciones que son evaluadas para producir un valor. Ejemplo: \code{x + y - 3}.

    \item \textbf{Sentencias:} Son instrucciones que realizan una acción específica. Puede ser una asignación, una llamada a una función, un bucle for o while, entre otros. Por ejemplo: \code{x = x + 1;}.

    \item \textbf{Procesos:} Se refieren a conjuntos de sentencias que juntas llevan a cabo una tarea específica dentro del programa. 

    \item \textbf{Subprogramas:} Este término puede referirse tanto a funciones como a procedimientos. Son segmentos de código que tienen un propósito definido y pueden ser llamados con ciertos parámetros. Permiten modularizar el código y reutilizar funcionalidades. Las funciones siempre devuelven un valor, mientras que los procedimientos no. 

    Ejemplo:
    \begin{verbatim}
    procedure mostrarMensaje();
    begin
        print("¡Hola Mundo!");
    end

    function sumaNumeros(a : integer, b : integer) : integer;
    begin
        return a+b;
    end
    \end{verbatim}

    \item \textbf{Parámetro de Subprograma:} Un parámetro es una variable que se utiliza para pasar información entre subprogramas. Los parámetros permiten que los subprogramas sean más flexibles y reutilizables, ya que pueden operar con diferentes datos cada vez que se llaman. Los parámetros se especifican en la definición del subprograma y se pasan valores cuando se llama al subprograma. Por ejemplo, en la función \code{sumaNumeros} anterior, \code{a} y \code{b} son parámetros.


\end{itemize}

\subsection{Tokens del lenguaje Lamport}\label{subsec:tokensLamport}
En esta subsección, se abordará una de las partes importantes en la definición de un lenguaje de programación: los tokens. Los tokens, también conocidos como símbolos terminales, representan las unidades léxicas mínimas que componen las sentencias y expresiones del lenguaje. Su identificación y clasificación permiten una interpretación estructurada del código fuente y facilitan la comprensión de la sintaxis del lenguaje. A continuación, se presentarán y describirán los tokens específicos que conforman el lenguaje \textit{Lamport}, proporcionando una base sólida para entender su estructura y funcionamiento. En la siguiente tabla se recogen todos los tokens del lenguaje, especificando el nombre de token, una breve descripción, un patrón de reconocimiento y un ejemplo si procede:

\newpage

\renewcommand{\arraystretch}{1.5}
\begin{longtable}{|c|M{4cm}|M{4cm}|M{2.5cm}|}
\caption{Tokens del lenguaje Lamport.} \label{tab:tokensLamport} \\
\hline
\textbf{TOKEN} & \textbf{DESCRIPCIÓN INFORMAL} & \textbf{PATRÓN} & \textbf{EJEMPLO} \\
\hline
\endfirsthead

\multicolumn{4}{c}%
{{\bfseries \tablename\ \thetable{} -- continuación de la página anterior}} \\
\hline
\textbf{TOKEN} & \textbf{DESCRIPCIÓN INFORMAL} & \textbf{PATRÓN} & \textbf{EJEMPLO} \\
\hline
\endhead

\hline \multicolumn{4}{|r|}{{Continúa en la siguiente página}} \\
\hline
\endfoot

\hline
\endlastfoot

\code{S_PROGRAM} & Palabra reservada para indicar el inicio del programa. & ``\code{program}'' & program \\
\hline
\code{S_VAR} & Palabra reservada para declarar variables. & ``\code{var}'' & var \\
\hline
\code{T_INTEGER} & Palabra reservada para el tipo entero. & ``\code{integer}'' & integer \\
\hline
\code{T_BOOLEAN} & Palabra reservada para el tipo boolean. & ``\code{boolean}'' & boolean \\
\hline
\code{T_CHAR} & Palabra reservada para el tipo carácter. & ``\code{char}'' & char \\
\hline
\code{T_STRING} & Palabra reservada para el tipo cadena de caracteres. & \code{string}'' & string \\
\hline
\code{T_REAL} & Palabra reservada para el tipo real (flotante). & ``\code{real}'' & real \\
\hline
\code{T_ARRAY} & Palabra reservada para declarar un vector. & ``\code{array}'' & array \\
\hline
\code{T_SEMAPHORE} & Palabra reservada para el tipo semáforo. & ``\code{semaphore}'' & semaphore \\
\hline
\code{T_DPROCESS} & Palabra reservada para declarar un proceso dinámico & ``\code{dprocess}'' & dprocess \\
\hline
\code{S_PROCESS} & Palabra reservada para indicar el inicio de un proceso & ``\code{process}'' & process \\
\hline
\code{S_PROCEDURE} & Palabra reservada para indicar el inicio de un procedimiento. & ``\code{procedure}'' & procedure \\
\hline
\code{S_FUNCTION} & Palabra reservada para indicar el inicio de una función & ``\code{function}'' & function \\
\hline
\code{RETURN} & Palabra reservada para indicar el retorno de una función & ``\code{return}'' & return \\
\hline
\code{B_BEGIN} & Palabra reservada para indicar el inicio de un bloque. & ``\code{begin}'' & begin \\
\hline
\code{B_END} & Palabra reservada para indicar el fin de un bloque. & ``\code{end}'' & end \\
\hline
\code{B_COBEGIN} & Palabra reservada para indicar el inicio de un bloque paralelo. & ``\code{cobegin}'' & cobegin \\
\hline
\code{B_COEND} & Palabra reservada para indicar el fin de un bloque paralelo. & ``\code{coend}'' & coend \\
\hline
\code{S_FORK} & Palabra reservada para indicar inicio del fork. & ``\code{fork}'' & fork \\
\hline
\code{JOIN} & Palabra reservada para sincronización de dprocess & ``\code{join}'' & join \\
\hline
\code{SLEEP} & Palabra reservada para dormir proceso & ``\code{sleep}'' & sleep \\
\hline
\code{IF} & Palabra reservada para estructura de control if. & ``\code{if}'' & if \\
\hline
\code{THEN} & Palabra reservada para condición then de if. & ``\code{then}'' & then \\
\hline
\code{ELSE} & Palabra reservada para condición else de if. & ``\code{else}'' & else \\
\hline
\code{WHILE} & Palabra reservada para estructura de control while. & ``\code{while}'' & while \\
\hline
\code{DO} & Palabra reservada para el bucle do de while. & ``\code{do}'' & do \\
\hline
\code{FOR} & Palabra reservada para estructura de control for. & ``\code{for}'' & for \\
\hline
\code{TO} & Palabra reservada para indicar límite bucle for. & ``\code{to}'' & to \\
\hline
\code{IDENT} & Identificador de variables, funciones, procedimientos. & ``\code{[a-zA-Z] ([a-zA-Z] \| [0-9])*}'' & x, aux, sum, proc1B \\
\hline
\code{LITERAL} & Secuencia de caracteres entre comillas dobles. & \code{\"^.*?\"\$} & "Hola Mundo!" \\
\hline
\code{L_INTEGER} & Literal entero. & ``\code{^-?[0-9]+\$}'' & 2, -31, 0 \\
\hline
\code{L_REAL} & Literal flotante. & ``\code{^-?[0-9]+ (\.[0-9]+)?\$}'' & 2.71, -6, -4.1314 \\
\hline
\code{L_BOOLEAN_TRUE} & Literal booleano (verdadero) & ``\code{true}'' & true \\
\hline
\code{L_BOOLEAN_FALSE} & Literal booleano (falso) & ``\code{false}'' & false \\
\hline
\code{L_CHAR} & Carácter entre comillas simples. & ``\code{^'.'\$}'' & 'A', '4', '?' \\
\hline
\code{OP_ASSIGN} & Operador de asignación. & ``\code{:=}`` & := \\
\hline
\code{OP_REL_LT} & Operador de comparación (menor que). & ``$<$'' & $<$ \\
\hline
\code{OP_REL_GT} & Operador de comparación (mayor que). & ``$>$'' & $>$ \\
\hline
\code{OP_REL_LTE} & Operador de comparación (menor o igual que). & ``$<=$`` & $<=$ \\
\hline
\code{OP_REL_GTE} & Operador de comparación (mayor o igual que). & ``$>=$`` & $>=$ \\
\hline
\code{OP_REL_EQ} & Operador de comparación (igual que). & ``\code{==}`` & == \\
\hline
\code{OP_REL_NEQ} & Operador de comparación (distinto que) & ``\code{\!=}`` & != \\
\hline
\code{OP_NOT} & Operador negación lógica. & ``\code{not}'' & not \\
\hline
\code{OP_AND} & Operador conjunción lógica. & ``\code{and}'' & and \\
\hline
\code{OP_OR} & Operación disyunción lógica. & ``\code{or}'' & or \\
\hline
\code{OP_SUM} & Operador suma & ``\code{+}'' & + \\
\hline
\code{OP_MINUS} & Operador resta & ``\code{-}'' & - \\
\hline
\code{OP_MULT} & Operador multiplicación & ``\code{*}'' & * \\
\hline
\code{OP_DIV} & Operador división & ``\code{/}'' & / \\
\hline
\code{OP_MOD} & Operador módulo & ``\code{\%}'' & \% \\
\hline
\code{PAR_IZDO} & Paréntesis izquierdo & ``\code{(}'' & ( \\
\hline
\code{PAR_DCHO} & Paréntesis derecho & ``\code{)}'' & ) \\
\hline
\code{CORCH_IZDO} & Corchete izquierdo & ``\code{[}'' & [ \\
\hline
\code{CORCH_DCHO} & Corchete derecho & ``\code{]}'' & ] \\
\hline
\code{DELIM_C} & Delimitador (coma) & ``\code{,}'' & , \\
\hline
\code{DELIM_PC} & Delimitador (punto y coma) & ``\code{;}'' & ; \\
\hline
\code{DELIM_2P} & Delimitador (dos puntos) & ``\code{:}'' & : \\
\hline
\code{DELIM_P} & Delimitador (punto) & ``\code{.}'' & . \\
\hline
\code{DELIM_ARR} & Delimitador de size de array & ``\code{..}'' & .. \\
\hline
\code{ATOM_INI} & Delimitador inicio sección atómica & ``$<$$<$'' & $<$$<$ \\
\hline
\code{ATOM_FIN} & Delimitador fin sección atómica & ``$>$$>$'' & $>$$>$ \\
\hline
\code{SEM_WAIT} & Indica operación wait sobre semáforo & ``\code{WAIT}'' & WAIT \\
\hline
\code{SEM_SIGNAL} & Indica operación signal sobre semáforo & ``\code{SIGNAL}'' & SIGNAL \\
\hline

\end{longtable}
\renewcommand{\arraystretch}{1.0}

Para la designación de las clases de los tokens, se ha decidido utilizar un convenio en el prefijo donde:

\begin{itemize}
    \item \textit{S\_} delimita aquellos símbolos que indican el inicio de un evento (programa, procedimiento, 
    función...).
    \item \textit{T\_} delimita aquellos símbolos relacionados con el tipo de dato (integer, real, array,...).
    \item \textit{B\_} delimita aquellos símbolos relacionados con un bloque (inicio, fin).
    \item \textit{L\_} delimita aquellos símbolos relacionados con literales (literal entero, flotante, booleano, char,...).
    \item \textit{OP\_} delimita aquellos símbolos que son operadores (binarios, de comparación,...).
\end{itemize}

\subsection{Descripción de la Gramática del lenguaje Lamport utilizando la notación Backus-Naur Form (BNF)}\label{subsec:sintaxisLamport}

La notación Backus-Naur Form (BNF), inventada por John Backus y Peter Naur, proporciona una forma concisa y comprensible de definir la gramática de un lenguaje, permitiendo a los desarrolladores y analistas de lenguajes comprender rápidamente su estructura. Al describir la gramática de Lamport mediante BNF, se ofrece una visión clara y detallada de cómo se estructura este lenguaje y cómo pueden formarse instrucciones válidas dentro de él.



A continuación, se presentan una serie de reglas y tokens definidos con la notación BNF para el lenguaje Lamport, con base en los ejemplos mencionados en la sección anterior. Estas reglas servirán como una guía para aquellos interesados en profundizar en la programación o análisis de este lenguaje, facilitando su aprendizaje y evitando ambigüedades en su interpretación.

\newpage

\begin{BNFCode}
# Definicion de reglas de sintaxis de creacion de programas
<program> ::= "program" <identifier> [";"] [<declarations>] 
    (<subprogram>)* <process>+

# Definion de reglas de sintaxis de creacion de declaraciones de variable
# Las declaraciones de variables se pueden encontrar:
#  --- Al principio de un programa: variables globales
#  --- Al principio de un subprograma: variables locales
#  --- Al principio de un proceso: variables locales
<declarations> ::= ("var" <identifier> ":" <type> 
    [":=" <expression>] ";")+

# Definicion de reglas de sintaxis de creacion de tipos
# Los tipos pueden ser:
#  --- Atomicos (basicos) : INTEGER, REAL, CHAR, STRING, BOOLEAN
#  --- Compuestos : ARRAY [size].
#  --- Especiales : SEMAPHORE, DPROCESS
<type> ::= <basic-type-or-array>
    | <special-type>

<basic-type-or-array> ::= <basic-type>
    | "array" "[" <expression> "]" <basic-type>

<special-type> ::= "semaphore"
    | "dprocess"

<basic-type> ::= "integer"
    | "real"
    | "char"
    | "string"
    | "boolean"
 
# Definicion de reglas de sintaxis de creacion de subprogramas
# Los subprogramas pueden ser:
#  --- Funciones : retornan tipo de dato (basicos)
#  --- Procedimientos : no retornan datos (funciones void)
<subprogram> ::= <procedure-definition>
	| <function-definition>


<procedure-definition> ::= "procedure" <identifier> 
    "(" <parameters> ")" ";" [<declarations>] 
    <block-statements-begin-end>
	
<procedure-function> ::= "function" <identifier> 
    "(" <parameters> ")" ":" <basic-type>
    ";" [<declarations>] <block-statements-function>
    
# Definicion de reglas de sintaxis de creacion de parametros
# Los parametros son listas de identificadores seguidos de un tipo de dato
<parameters> ::= [<parameter> ["," <parameters>]]
<parameter> ::= <identifier> ":" <basic-type>

# Definicion de reglas de sintaxis de creacion de procesos
# Los procesos pueden ser:
#   --- Normales (Single).
#   --- Vectorizados. Definen una serie de procesos con base en un indexador
<process> ::= "process" <identifier> [ "[" <identifier> : 
    <expression> ".." <expression> "]" ] ";" 
    [<declarations>] <block-statements-begin-end>


# Definicion de reglas de sintaxis de sentencias
# Las diferentes sentencias disponibles para el lenguaje son:
#   --- Bloque de sentencias BEGIN/END
#   --- Bloque de sentencias COBEGIN/COEND
#   --- Bloque de sentencias ATOMICAS
#   --- Asignacion
#   --- Bucle for
#   --- Bucle while
#   --- If/else
#   --- Fork
#   --- Join
#   --- Sleep
#   --- Llamada a procedimiento
#   --- Impresion de contenido (print)
#   --- Operacion wait sobre semaforo
#   --- Operacion signal sobre semaforo
<block-statements-begin-end> ::= "begin" (<statement>)+ "end"
<block-statements-cobegin-coend> ::= "cobegin" 
    (<statement>)+ "coend"
<block-statements-atomic> ::= "<<" (statement)+ ">>"
<block-statements-function> ::= "begin" (<statement>)+ 
    <return-statement> "end"
	
<statement> ::= <assignment-statement>
    | <while-statement>
    | <for-statement>
    | <if-statement>
    | <procedure-call-statement>
    | <block-statements-atomic>
    | <block-statements-cobegin-coend>
    | <fork-statement>
    | <join-statement>
    | <sleep-statement>
    | <print-statement>
    | <sem-wait-statement>
    | <sem-signal-statement>
	
<assignment-statement> ::= <identifier> ["[" <expression> "]"] 
    ":=" <expression> ";"
    
<while-statement> ::= "while" <expression> "do" 
    <block-statements-begin-end>
    
<for-statement> ::= "for" <identifier> ":=" <expression> "to" 
    <expression> "do" <block-statements-begin-end>
    
<if-statement> ::= "if" <expression> "then" 
    <block-statements-begin-end> 
    ["else" <block-statements-begin-end>]
    
<procedure-call-statement> ::= <identifier> "(" <arguments> ")" ";"

<fork-statement> ::= "fork" <identifier> ";"

<join-statement> ::= "join" ";"

<sleep-statement> ::= "sleep" <expression> ";"

<print-statement> ::= "print" "(" <print-list> ")" ";"
<print-list> ::= <expression> ("," <expression>)*

<return-statement> ::= "return" <expression> ";"

<sem-wait-statement> ::= "WAIT" <identifier> ";"
<sem-signal-statement> ::= "SIGNAL" <identifier> ";"

# Definicion de reglas de sintaxis de expresiones
# Las diferentes expresiones disponibles para el lenguaje son:
#   --- binarias (<expression> simbolo <expresion>)
#   --- unarias  (simbolo <expression>)
#   --- literales (INTEGER, REAL, CHAR, STRING, BOOLEAN)
#   --- identificador (variable de tipo basico/ARRAY)
#   --- llamada a funcion
#   --- expresion entre parentesis
<expression> ::= <expression> <binary-operator> <expression>
	| <unary-operator> <expression>
	| <term>

<term> ::= <identifier> ["[" <expression> "]"]
	| <literal-expression>
	| <function-call-expression>
	| "(" <expression> ")"
	
<literal-expression> ::= <integer-literal>
	| <real-literal>
	| <string-literal>
	| <char-literal>
	| <boolean-literal>
	
<integer-literal> ::= <digit>+
<boolean-literal> ::= "true" | "false"
<string-literal> ::= ``cualquier secuencia de caracteres del juego de caracteres en uso, con comillas dobles al inicio y al final''
<char-literal> ::= ``cualquier caracter del juego de caracteres en uso, con comilla simple al inicio y al final''
<real-literal> ::= <integer-literal> "." <integer-literal>
<identifier> ::= <letter> (<leter-or-digit>)*
<letter-or-digit> ::= <letter> | <digit>
<letter> ::= [a-zA-Z]
<digit> ::= [0-9]

<function-call-expression> ::= <identifier> "(" <arguments> ")"

<binary-operator> ::= "*" | "/" | "+" | "-" | "%" |
 	| ">" | "<" | "<=" | ">=" | "==" | "!="
	| "and" | "or"
<unary-operator> ::= "-" | "not"

# Definicion de reglas de sintaxis de generacion de argumentos
<arguments> ::= [<expression> ["," <arguments>]]
\end{BNFCode}

\noindent
Sobre las características del metalenguaje de BNF hay que destacar que:
\begin{itemize}
    \item El metasímbolo [ ... ] denota una opción: el contenido dentro de los corchetes puede estar presente o no. Por ejemplo, \code{[<declarations>]} indica que se puede una lista de declaraciones.
    \item El metasímbolo ( ... )* indica cero o más repeticiones del contenido dentro de los paréntesis. Por ejemplo, \code{(a|b)*} puede representar cadenas como '', 'a', 'b', 'ab', 'ba', 'aa', 'bb', y así sucesivamente.
    \item El metasímbolo ( ... )+ indica una o más repeticiones del contenido dentro de los paréntesis. Difiere del anterior en que debe haber al menos una ocurrencia. Así, \code{(a|b)+} podría representar 'a', 'b', 'ab', 'ba', 'aa', 'bb', pero no la cadena vacía.
\end{itemize}

\noindent
Y sobre la descrición del lenguaje Lamport, es importante notar que:
\begin{itemize}
    \item Los tokens, representados entre comillas (p.ej. \code{''program''}), denotan palabras clave y símbolos específicos del lenguaje \footnote{Aunque justo en la sección anterior se ha introducido ya la tabla de tokens, para el lector es más agradable visualizar las reglas de esta forma, dejando el patrón de reconocimiento de los tokens en vez de su identificación.}.
    \item Los símbolos <...> \hspace{0.05cm} representan categorías gramaticales, que pueden estar compuestas por otras categorías gramaticales o tokens.
    \item Las reglas son recursivas, permitiendo la construcción de sentencias y expresiones de diversa complejidad y longitud.
    \item Los comentarios, precedidos por el símbolo \#, proporcionan aclaraciones y contextos adicionales sobre las reglas.
\end{itemize}

\subsubsection{Ambigüedades en la gramática}\label{subsubsec:gramaticaAmbiguaLamport}
Al definir la gramática del lenguaje Lamport utilizando la notación BNF, hay que identificar y resolver cualquier ambigüedad que pueda surgir. Una gramática ambigua es aquella en la que una cadena puede tener más de un árbol de derivación, lo que lleva a diferentes interpretaciones de la cadena. En el contexto de los lenguajes de programación, esto puede ser problemático ya que diferentes interpretaciones pueden llevar a diferentes comportamientos del programa. Es por ello por lo que la pregunta que cabe hacerse ahora es: \textit{¿Presenta esta gramática ambigüedades?}



La respuesta es \textbf{sí}, y a continuación, se discuten algunas áreas en la gramática presentada donde podrían surgir ambigüedades y se proporcionan soluciones o aclaraciones para las mismas.

\noindent
\textbf{Expresiones Binarias y Precedencia de Operadores:}

\vspace{0.3cm}

\noindent
La regla:
\begin{BNFCode}
<expression> ::= <expression> <binary-operator> <expression>
| <unary-operator> <expression>
| <term>
\end{BNFCode}



Puede dar lugar a ambigüedades en expresiones como ``\code{a + b * c}''. Sin una clarificación sobre la precedencia de los operadores, no está claro si se debe interpretar como ``\code{(a + b) * c}'' o como ``\code{a + (b * c)}''.



\textbf{Solución:} Establecer reglas separadas para cada nivel de precedencia o modificar la gramática para que refleje explícitamente la precedencia.


\subsection{Descripción Semántica del lenguaje Lamport}\label{subsec:semanticaLamport}

La semántica, en el contexto de los lenguajes de programación, se refiere al significado de los programas. No solo se considera la estructura de un programa (su sintaxis), sino también lo que realiza al ejecutarse (su semántica). En esta sección, se profundizará en la semántica del lenguaje Lamport, desvelando el comportamiento y operación de sus distintos constructos y componentes.

\subsubsection{Descripción semántica general del lenguaje Lamport}
El lenguaje Lamport se puede describir como \textit{inseguro}, \textit{estático} y \textit{explícito}. Esto quiere decir que:

\begin{itemize}
    \item \textbf{Inseguro}: En un lenguaje de programación no seguro, se permiten escribir programas válidos que pueden tener un comportamiento indefinido que viola la estructura básica del programa. Esto por ejemplo se puede ver en los accesos a un ARRAY, donde si tiene tamaño 100 y se accede a una posición fuera de ese rango, compilará, pero puede que su comportamiento será indecible. En este lenguaje en particular si esto sucede se lanzará una excepción.
    \item \textbf{Estático}: La comprobación de tipos se realiza en tiempo de compilación. Cuando llegue el momento de traducir el código a binario (o en el caso de este lenguaje, la interpretación de las instrucciones generadas), no será necesario mantener información del tipo de dato de cada variable, subprograma o parámetro, porque todas las operaciones se habrán verificado y determinado como seguras.
    \item \textbf{Explícito}: El programador es responsable de indicar los tipos de variables o de otros elementos de código explícitamente.
\end{itemize}

\subsubsection{Tipos de datos del lenguaje Lamport}
Con respecto a los tipos de dato del lenguaje Lamport, se tienen los siguientes:
\begin{itemize}
    \item \textbf{Tipos básicos (atómicos)}:
    \begin{itemize}
        \item \code{integer}: entero con signo de 32 bits.
        \item \code{real}: número flotante con signo de 32 bits.
        \item \code{char}: caracteres ASCII.
        \item \code{string}: cadena de caracteres ASCII, terminados en ``\textbackslash 0''.
        \item \code{boolean}: símbolos \code{true} y \code{false}.
    \end{itemize}
    \item \textbf{Tipos compuestos}:
    \begin{itemize}
        \item \code{array [size] <basic-type>}: Array de tamaño \textit{size} y de tipo atómico \textit{<basic-type>}.
    \end{itemize}
\end{itemize}


\subsubsection{Restricciones semánticas}\label{subsubsec:restriccionesSemanticas}
Finalmente, con respecto a las reglas de validación de operaciones:
\begin{itemize}
    \item Sólo se puede asignar un único valor a una variable al mismo tiempo. Definir dos asignaciones diferentes, efectivamente, son dos operaciones distintas.
    \item Un parámetro de función sólo acepta un un valor al mismo tiempo.
    \item En un subprograma función, el tipo de dato que devuelve debe coincidir con el tipo de dato de su sentencia de retorno $<$ \code{return-statement} $>$. Además, esta sentencia \code{return} debe ser \textit{la última que aparezca en el cuerpo de una función}.
    \item Los operadores de comparación de igualdad \code{\!=} y \code{==} pueden aceptar cualquier tipo de dato \textit{básico}.
    \item Los operadores de comparación $<$, $>$, $<=$, $>=$ sólo aceptan o números enteros \code{integer} o reales \code{real} en ambos miembros de la comparación, y además, deben coincidir en tipo.
    \item Los operadores aritméticos \code{+}, \code{-}, \code{*}, \code{/}, sólo aceptan números enteros \code{integer} o reales \code{real} en ambos operandos, y además, deben coincidir en tipo. Es decir, no se pueden aplicar a \code{integer} y \code{real} simultáneamente.
    \item El operador aritmético \code{\%} sólo acepta números enteros \code{integer}.
\end{itemize}



Es necesario indicar al final de esta lista de restricciones semánticas que hay un par de restricciones adicionales que no son motivo intrínseco del lenguaje en sí, sino por cuestiones de implementación, y son:

\begin{itemize}
    \item En una declaración de tipo \code{array} su tamaño debe ser indicado exactamente con un literal entero puro. No se permiten otro tipo de expresiones.
    \item Igual que en el caso anterior, en la definición estática de procesos vectorizados, los límites de principio y fin que delimitan al índice sólo pueden ser indicados exactamente con un literal entero puro.
\end{itemize}

La cuestión de hacer esto es que así se está realizando una declaración de vector (o procesos) estáticos. En otro caso, se debería determinar en tiempo de ejecución el tamaño que hay que dedicar en la memoria de la Máquina Virtual, y reservarla. Esto se verá más detalladamente en el siguiente capítulo de implementación.

\subsubsection{Precedencia y Asociatividad de operadores}\label{subsec:precOperadoresLamport}

La correcta definición de la precedencia y la asociatividad de los operadores es esencial en cualquier lenguaje de programación. Estos dos conceptos dictan el orden en el que se evalúan las operaciones y cómo se agrupan los operandos en presencia de varios operadores. La especificación clara de estas reglas garantiza que los programas se ejecuten de manera coherente y predecible, evitando ambigüedades que podrían llevar a comportamientos inesperados o a errores difíciles de diagnosticar. A continuación, se presenta una tabla que detalla el orden de prioridades y la asociatividad de los operadores en el lenguaje.

\newpage

\renewcommand{\arraystretch}{1.5}
\begin{longtable}{|c|M{3.5cm}|M{3cm}|c|}
\caption{Precedencia y asociatividad de operadores de Lamport.} \label{tab:precedencyOperatorsLamport} \\
\hline
\textbf{PRECEDENCIA} & \textbf{OPERADOR(ES)} & \textbf{DESCRIPCIÓN} & \textbf{ASOCIATIVIDAD} \\
\hline
\endfirsthead

\multicolumn{4}{c}%
{{\bfseries \tablename\ \thetable{} -- continuación de la página anterior}} \\
\hline
\textbf{PRECEDENCIA} & \textbf{OPERADOR(ES)} & \textbf{DESCRIPCIÓN} & \textbf{ASOCIATIVIDAD} \\
\hline
\endhead

\hline \multicolumn{4}{|r|}{{Continúa en la siguiente página}} \\
\hline
\endfoot

\hline
\endlastfoot

1 & $-$ & Operador menos (unario) & Derecha a Izquierda \\
\hline
2 & $*$ \hspace{0.1cm} $/$ \hspace{0.1cm} \% & Multiplicación, división y módulo (binario) & Izquierda a Derecha \\
\hline
3 & $+$ \hspace{0.1cm} $-$ & Suma y resta (binario) & Izquierda a Derecha \\
\hline
4 & $-$ $<$ $<=$ $>$ $>=$ $==$ \code{\!=} & Operadores de comparación (binario) & Izquierda a Derecha \\
\hline
5 & not & Negación lógica (unario) & Derecha a Izquierda \\
\hline
6 & and & Conjunción lógica (binario) & Izquierda a Derecha \\
\hline
7 & or & Disyunción lógica (binario) & Izquierda a Derecha \\

\end{longtable}
\renewcommand{\arraystretch}{1.0}

La enumeración en la primera columna sirve para indicar el orden de análisis de los operadores, en el que un número más pequeño implica una mayor prioridad o importancia. Esto significa que, en una expresión que involucra múltiples operadores, los operadores con menor número se evaluarán antes que aquellos con números mayores. Por ejemplo, en una expresión como \code{2 + 3 * 4}, debido a que la multiplicación tiene una mayor prioridad (2) que la suma (3), se evaluará primero \code{3 * 4} y luego se sumará \code{2} al resultado.



La columna de asociatividad informa sobre cómo se evalúan los operadores cuando aparecen varios del mismo tipo en una secuencia sin paréntesis claros para determinar el orden. La asociatividad ``\textit{Izquierda a Derecha}'' indica que, en caso de una secuencia de operadores con la misma precedencia, se comenzará a evaluar desde el operador más a la izquierda y se procederá hacia la derecha. Por ejemplo, en la expresión \code{5 - 3 - 2}, se evaluará primero \code{5-3} debido a esta asociatividad, y luego se restará \code{2} al resultado. Por otro lado, la asociatividad ``\textit{Derecha a Izquierda} implica que la evaluación comienza con el operador más a la derecha y procede hacia la izquierda. Este tipo de asociatividad es menos común y generalmente se encuentra en operadores unarios, como el operador negativo unario mostrado en la tabla.

\section{Conclusiones del estudio}
A lo largo de este capítulo, se ha llevado a cabo un estudio detallado y sistemático del lenguaje Lamport, desde su concepción mediante el análisis del pseudocódigo base propuesto en la asignatura \textit{Sistemas Concurrentes y Distribuidos} hasta su posterior diseño. Se ha descrito la estructura gramatical, que es el núcleo de cualquier lenguaje, proporcionando una visión clara y concisa de cómo se construyen y analizan las sentencias en Lamport.


\noindent
Ahora, podemos definir formalmente la gramática de Lamport como la cuádrupla:

$$
G = (V,T,P,S)
$$

\noindent
donde:

\begin{itemize}
    \item $V$ es el conjunto de símbolos no terminales, definidos en la parte izquierda de las reglas de sintaxis definidas en la subsección ~\ref{subsec:sintaxisLamport}.
    \item $T$ son el conjunto de símbolos terminales que han sido representados mediante tokens, y recogidos en una tabla junto con su patrón de reconocimiento en la subsección ~\ref{subsec:tokensLamport}.
    \item $P$ es un conjunto finito de producciones, detallado en la subsección ~\ref{subsec:sintaxisLamport}.
    \item $S$ es un elemento de $V$ que corresponde al \textit{símbolo de partida}, y es el símbolo \hspace{0.5cm} $<$ \code{program} $>$.
\end{itemize}

Además, se ha discutido acerca de la ambigüedad de la gramática en la subsección ~\ref{subsubsec:gramaticaAmbiguaLamport}, donde se han proporcionado algunas soluciones para la resolución de dichas ambigüedades, como en el caso de evaluación de expresiones utilizando una tabla de precedencia de operadores, definida en la subsección ~\ref{subsec:precOperadoresLamport}.



Para finalizar, se puede concluir que la gramática definida siguiendo la Jerarquía de Chomsky es de \textbf{tipo 2} (~\ref{section:chomsky} ), o en otras palabras, es una \textbf{gramática independiente del contexto}, y genera el lenguaje Lamport, denotado como $\mathscr{L}$.



En el siguiente capítulo, comprobaremos que aplicando algunas modificaciones a las reglas y eliminando las ambigüedades, la gramática es \code{LALR(1)} \cite{aho1990compiladores} (sección 4.7.4), puesto que el generador de analizadores sintácticos Bison, es capaz de procesar la gramática sin notificar ningún de conflicto de tipo reducción o desplazamiento.



	% Desarrollo bajo sprints: 
	\input{secciones/08_implementacion}

	% Conclusiones
	\input{secciones/09_programas}

	% Conclusiones
	\chapter{\textbf{Conclusiones y trabajos futuros}}

\section{Conclusiones}
Este proyecto ha abarcado dos áreas fundamentales relacionadas con los sistemas concurrentes y distribuidos. Primero, se llevó a cabo un estudio exhaustivo de los sistemas concurrentes, centrando la atención en una especificación formal como la propuesta por Leslie Lamport en su Lógica Temporal de Acciones (TLA). Este análisis detallado no solo proporcionó una base teórica sólida para entender la complejidad inherente a estos sistemas, sino que también destacó la importancia de una especificación rigurosa para su correcto diseño y verificación, a través del formalismo matemático.

En segundo lugar, se adoptó un enfoque práctico mediante el desarrollo de un intérprete para un lenguaje de programación diseñado específicamente para simular sistemas concurrentes y distribuidos. Este lenguaje se basó en el pseudocódigo utilizado en la asignatura ``Sistemas Concurrentes y Distribuidos'', permitiendo así una mayor accesibilidad y comprensión para aquellos familiarizados con el curso. La implementación de este intérprete se realizó siguiendo una metodología de desarrollo ágil, lo que facilitó un proceso de desarrollo iterativo y adaptable.

Durante este proceso, se utilizaron herramientas avanzadas de programación, incluyendo analizadores léxicos y sintácticos, y se aprovecharon las capacidades de los lenguajes C y C++ para asegurar un rendimiento óptimo y una integración efectiva. Esta combinación de teoría y práctica no solo ha enriquecido la comprensión de los sistemas concurrentes y distribuidos, sino que también ha proporcionado una herramienta valiosa para su estudio y simulación.

Al concluir este proyecto, se ha logrado un balance entre la teoría formal y la aplicación práctica, proporcionando una perspectiva integral de los sistemas concurrentes y distribuidos. Este trabajo no solo sirve como un recurso educativo para aquellos que buscan profundizar en este campo, sino que también sienta las bases para futuras investigaciones y desarrollos en esta área tan dinámica y desafiante de la informática.

\section{Trabajos futuros}
Desarrollar un intérprete completo para dar vida a un nuevo lenguaje de programación es cuanto menos desafiante, además teniendo en cuenta los cortos plazos de tiempo en los que se han desarrollado. Puesto que el software siempre está en continua evolución y en continua obsolescencia, hay algunos aspectos que se pueden mejorar y nuevas que desarrollar, citadas a continuación:

\begin{itemize}
    \item \textbf{SAAS (Software As A Service)}: Con el intérprete ya dockerizado, una prometedora dirección futura para este proyecto es su desarrollo y lanzamiento como Software as a Service (SaaS). Esta transición a una plataforma basada en la nube no solo facilitaría un acceso más amplio y flexible al intérprete, sino que también permitiría una gestión más eficiente y la implementación rápida de actualizaciones y mejoras. Al ofrecer el intérprete como un servicio en la nube, podríamos expandir significativamente su alcance y utilidad, proporcionando una herramienta valiosa y accesible para una audiencia global interesada en la simulación y el estudio de sistemas concurrentes y distribuidos.
    \item \textbf{Nuevos mecanismos de sincronización y ampliación de la gramática}: De forma nativa el intérprete permite definir semáforos como mecanismo de sincronización de hebras, por lo que quizá estaría bien considerar otras como ``Monitores''. También, puede considerarse ampliar la gramática de Lamport permitiendo nuevos constructos que faciliten el uso por parte del usuario.
    \item \textbf{Reimplementación de intérprete en C++}: En este proyecto se utilizó C y C++ como lenguajes de desarrollo del intérprete, dejando C para la parte más cercana a la fase de análisis de código (léxico, sintáctico y semántico). Aunque C es un lenguaje muy versátil a día de hoy, su sucesor C++ implementa características más seguras, en lo que concierne a gestión de memoria dinámica vía punteros. Por otra parte, su sintaxis orientada a objetos y sus múltiples bibliotecas estándar hace que la definición y/o uso de estructuras de datos sea más directa que en C, y como Flex y Bison permiten generar analizadores para este lenguaje, quizá sería conveniente reimplementar todos esos módulos con este lenguaje, garantizando más seguridad y claridad.
    \item \textbf{Gestión de errores sintácticos uniforme}: Se podría mejorar la gestión actual de los errores sintácticos por parte de Bison, definiendo unas reglas de producción más granulares y específicas, incluso teniendo en cuenta patrones incorrectos debido a tokens que no debían estar ahí.
    \item \textbf{Habilitar expresiones en definición de arrays}: Gestionar la memoria en tiempo de ejecución es una tarea ardua si las implementaciones deben hacerse desde cero, y es por eso por lo que la gramática actual, aunque contemple cualquier tipo de expresión para la declaración de arrays, se considera semánticamente incorrecta la declaración si dicha expresión no contiene exactamente un literal entero. Es por ello por lo que dentro de la memoria de la máquina virtual se podría considerar un heap que permita obtener el tamaño de los arrays en tiempo de ejecución.
    \item \textbf{Simulación realista de la asignación de los registros}: Con la estrategia actual la simulación que hace la máquina virtual es menos realista que lo que se hace en una máquina convencional. Se podría considerar cambiar esta mecánica de decisión de registros a la hora de generar las instrucciones de la representación intermedia, comprobando la vivacidad de los registros.
\end{itemize}

Para concluir, este proyecto no solo representa un paso significativo en la simulación y estudio de sistemas concurrentes y distribuidos, sino que también establece una sólida base para futuras innovaciones y mejoras. Con estas vías futuras, el proyecto está bien posicionado para adaptarse y evolucionar, manteniéndose relevante y útil en un campo que está en constante cambio y crecimiento.

	% Trabajos futuros


	
	\newpage
        \nocite{*}
	\bibliography{bibliografia}
	\bibliographystyle{plain}
	
\end{document}

