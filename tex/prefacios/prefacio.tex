\thispagestyle{empty}

\begin{center}
{\large\bfseries Simulador de Sistemas Concurrentes y Distribuidos. Lógica Temporal de Acciones }\\
\end{center}
\begin{center}
Daniel Pérez Ruiz\\
\end{center}

%\vspace{0.7cm}

\vspace{0.5cm}
\noindent\textbf{Palabras clave}: \textit{ágil, concurrencia, sistema, programa, lógica, instrucción, lenguaje, compilador, máquina virtual}
\vspace{0.7cm}

\noindent\textbf{Resumen}\\
Este trabajo se centra en el desarrollo de un compilador para un lenguaje de programación, nombrado Lamport en honor a Leslie Lamport y su influencia en el ámbito de los sistemas concurrentes y distribuidos, especialmente a través de su contribución con la Lógica Temporal de Acciones (TLA). Además este proyecto presenta un estudio detallado de TLA y un ejemplo práctico de su aplicación en la especificación de sistemas, proporcionando una perspectiva matemática y formal para la especificación y verificación de sistemas concurrentes en general. El objetivo principal, sin embargo, es la creación de una herramienta práctica que concretice los conceptos de concurrencia en una aplicación tangible y accesible para los usuarios.

El compilador de Lamport ha sido diseñado específicamente para simular sistemas concurrentes y distribuidos, permitiendo así a los usuarios interactuar directamente con las complejidades y retos inherentes a la programación en este campo. Desarrollado bajo una metodología ágil y utilizando las capacidades de los lenguajes C y C++, el compilador profundiza en la comprensión de aspectos clave como la sincronización de procesos o el manejo de bloqueos, a parte de ser un lenguaje totalmente funcional para programas secuenciales.

El trabajo también incluye una exploración de los principios fundamentales de los sistemas concurrentes y distribuidos, proporcionando el marco teórico esencial para comprender y utilizar eficazmente el compilador Lamport. Se examinan posibles ampliaciones y mejoras del compilador, incluyendo su eventual evolución hacia un modelo de Software as a Service (SaaS), la incorporación de mecanismos de sincronización más sofisticados y la expansión de la gramática del lenguaje.

\cleardoublepage

\begin{center}
	{\large\bfseries Concurrent and Distributed Systems Simulator. Temporal Logic of Actions}\\
\end{center}
\begin{center}
	Daniel Pérez Ruiz\\
\end{center}
\vspace{0.5cm}
\noindent\textbf{Keywords}: \textit{agile, concurrency, system, program, logic, instruction, language, compiler, virtual machine}
\vspace{0.7cm}

\noindent\textbf{Abstract}\\
This work focuses on the development of an compiler for a programming language, named Lamport in honor of Leslie Lamport and his influence in the field of concurrent and distributed systems, particularly through his contribution to Temporal Logic of Actions (TLA). Additionally, this project presents a detailed study of TLA and a practical example of its application in system specification, offering a mathematical and formal perspective for the specification and verification of concurrent systems in general. The main goal, however, is to create a practical tool that materializes the concepts of concurrency into a tangible and accessible application for users.

The Lamport compiler has been specifically designed to simulate concurrent and distributed systems, thus allowing users to directly interact with the complexities and challenges inherent in programming in this field. Developed under an agile methodology and utilizing the capabilities of C and C++ languages, the compiler deepens the understanding of key aspects such as process synchronization or deadlock handling, apart from being a fully functional language for sequential programs.

The work also includes an exploration of the fundamental principles of concurrent and distributed systems, providing the essential theoretical framework to effectively understand and use the Lamport compiler. Potential extensions and improvements of the compiler are examined, including its eventual evolution into a Software as a Service (SaaS) model, the incorporation of more sophisticated synchronization mechanisms, and the expansion of the language grammar.



\cleardoublepage

\thispagestyle{empty}

\noindent\rule[-1ex]{\textwidth}{2pt}\\[4.5ex]

D. \textbf{Tutora/e(s)}, Profesor(a) del ...

\vspace{0.5cm}

\textbf{Informo:}

\vspace{0.5cm}

Que el presente trabajo, titulado \textit{\textbf{Simulador de Sistemas Concurrentes y Distribuidos. Lógica Temporal de Acciones}},
ha sido realizado bajo mi supervisión por \textbf{Daniel Pérez Ruiz}, y autorizo la defensa de dicho trabajo ante el tribunal
que corresponda.

\vspace{0.5cm}

Y para que conste, expiden y firman el presente informe en Granada a Noviembre de 2023.

\vspace{1cm}

\textbf{El/la director(a)/es: }

\vspace{5cm}

\noindent \textbf{(nombre completo tutor/a/es)}

\chapter*{Agradecimientos}

Desde que era muy pequeño mi madre siempre me ha dicho que, si ves una mariposa blanca, eso es señal de suerte. Aunque no me caracterizo por ser una persona supersticiosa, siempre he convivido con esa inocencia cada vez que tenía ante mis ojos una de ellas, aleteando, al son de la libertad, iluminando en mí una sonrisa.

\vspace{0.2cm}

A lo largo de mi vida he tenido muchas ocasiones donde sentía que me habían abandonado. Y la realidad, es que siempre me han acompañado, aunque muchas veces no las viera. Ahí se encontraban, danzando al compás de todas las melodías y letras, de todas aquellas canciones que en tiempos pasados resonaron en mi cabeza.

\vspace{0.2cm} 

A mi madre, a mi hermana Irene, a mi padre. A mis amigos de toda una vida Adrián, Alberto, Elías, Juanmi, Juan, Rafa Aguilar y Rafa Barrales. A mis amigos de esta nueva vida Lucía, Martín, Mery, Jaime, Jose, Joseja y Pablo. No hay suficientes palabras en este mundo para describir lo inmensos que son. A mi amiga Candela, por haber por fin coincidido en este camino tras muchas historias y canciones, y que prometo recorrerlo bailando una vez más. A mi amiga Elena, por haber sido la mayor voz que me guió y me dio luz cuando no la había. A Coral, por haber venido casi al final con su deslumbrante forma de ser, brindándome un nuevo y prometedor comienzo. A mis dos grandes mentores: mi tutor, Carlos, por ser mi fuente de inspiración desde que coincidí con él; y JJ, la persona que me enseñó a cómo encontrarme si me perdía en el camino.

\vspace{0.2cm}
En este trabajo dejo atrás una muy buena parte de mi vida, donde cumplí el sueño de la infancia que me llevó hasta este mismo momento, el de descubrir y aprender todo lo \textit{immenso} que es el mundo de la informática, así como también lo es la vida misma. Esta parte de mí dejará de existir para siempre, pero ahora es mi turno de volar, al igual que todas las mariposas blancas que siempre estuvieron para mí.

\newpage
\vspace*{\fill}

\large
\noindent
$P := \textit{``Estás dispuesto a arriesgar.''}$
\newline
$Q := \textit{``Puedes crecer.''}$
\newline
$R := \textit{``Puedes dar lo mejor de ti.''}$
\newline
$S := \textit{``Puedes ser feliz.''}$
\newline
\newline

\noindent
\textit{Si} $\neg P \implies \neg Q$;
\newline
\textit{Si} $\neg Q \implies \neg R$;
\newline
\textit{Si} $\neg R \implies \neg S$;
\newline
\textit{Y si} $\neg S$, ¿qué te queda?
\newline



\vspace*{\fill}