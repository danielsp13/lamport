\thispagestyle{empty}

\begin{center}
{\large\bfseries Simulador de Sistemas Concurrentes y Distribuidos. Lógica Temporal de Acciones }\\
\end{center}
\begin{center}
Daniel Pérez Ruiz\\
\end{center}

%\vspace{0.7cm}

\vspace{0.5cm}
\noindent\textbf{Palabras clave}: \textit{software libre, ágil, concurrencia, sistema, programa, lógica, tiempo, instrucción, lenguaje, compilador, máquina virtual}
\vspace{0.7cm}

\noindent\textbf{Resumen}\\
Este trabajo se centra en el desarrollo de un lenguaje de programación y un compilador en tiempo de ejecución para el mismo, con el objetivo de facilitar el aprendizaje y la enseñanza de los conceptos y el diseño de sistemas concurrentes y distribuidos. Utilizando metodologías de desarrollo ágil y poniendo el foco siempre en los posibles usuarios del proyecto, se consigue una mejora continua y una respuesta eficaz a los desafíos emergentes durante el desarrollo. El lenguaje, creado con una sintaxis simplificada y clara, evita la necesidad de bibliotecas externas o dependencias adicionales, diferenciándolo de lenguajes de programación de alto nivel convencionales como C++.

En el desarrollo de este software, se ha hecho uso de avanzadas herramientas de análisis sintáctico y léxico, lo que ha permitido una implementación más precisa y eficiente del compilador. Estas herramientas han sido fundamentales para construir un sistema robusto que proporciona al usuario información detallada como el Árbol de Sintaxis Abstracta (AST), la Representación Intermedia (IR), y la traza de ejecución de instrucciones dentro de la Máquina Virtual implementada.

Una parte fundamental de este trabajo es la exploración de los fundamentos matemáticos en la verificación de sistemas concurrentes, particularmente a través de la Lógica Temporal de Acciones, definida por Leslie Lamport. Esta exploración no es meramente un complemento, sino un elemento central que fortalece el desarrollo del lenguaje y del compilador. La comprensión profunda de la verificación matemática es vital para asegurar la fiabilidad y efectividad de los sistemas desarrollados, ofreciendo a los usuarios una formación integral en la teoría y práctica de estos sistemas.

La creciente complejidad de los sistemas informáticos modernos y la omnipresencia de aplicaciones y servicios basados en la nube subrayan la importancia del entendimiento de los sistemas concurrentes y distribuidos. En un mundo donde la eficiencia y la capacidad de manejar múltiples tareas simultáneamente son cruciales, la programación concurrente se convierte en una habilidad indispensable. Sin embargo, estos sistemas a menudo presentan desafíos significativos debido a su naturaleza compleja. Proporcionar un medio para una comprensión clara y práctica de estos sistemas, como lo hace la herramienta desarrollada, es vital para la formación de futuros profesionales y el impulso de soluciones tecnológicas avanzadas.

\cleardoublepage

\begin{center}
	{\large\bfseries Concurrent and Distributed Systems Simulator. Temporal Logic of Actions}\\
\end{center}
\begin{center}
	Daniel Pérez Ruiz\\
\end{center}
\vspace{0.5cm}
\noindent\textbf{Keywords}: \textit{free software, agile, concurrency, system, program, logic, time, instruction, language, compiler, virtual machine}
\vspace{0.7cm}

\noindent\textbf{Abstract}\\
This work focuses on the development of a programming language and a runtime compiler for it, with the aim of facilitating the learning and teaching of concepts and design of concurrent and distributed systems. By using agile development methodologies and always focusing on potential users of the project, we achieve continuous improvement and an effective response to emerging challenges during development. The language, created with a simplified and clear syntax, avoids the need for external libraries or additional dependencies, differentiating it from conventional high-level programming languages like C++.

In the development of this software, we have used advanced syntactic and lexical analysis tools, allowing for a more precise and efficient implementation of the compiler. These tools have been fundamental in building a robust system that provides the user with detailed information such as the Abstract Syntax Tree (AST), Intermediate Representation (IR), and the trace of instruction execution within the implemented Virtual Machine.

A fundamental part of this work is the exploration of the mathematical foundations in the verification of concurrent systems, particularly through the Temporal Logic of Actions, defined by Leslie Lamport. This exploration is not merely an addition, but a central element that strengthens the development of the language and the compiler. A deep understanding of mathematical verification is vital to ensure the reliability and effectiveness of the systems developed, offering users comprehensive training in the theory and practice of these systems.

The increasing complexity of modern computer systems and the ubiquity of cloud-based applications and services underscore the importance of understanding concurrent and distributed systems. In a world where efficiency and the ability to handle multiple tasks simultaneously are crucial, concurrent programming becomes an indispensable skill. However, these systems often present significant challenges due to their complex nature. Providing a means for a clear and practical understanding of these systems, as the developed tool does, is vital for training future professionals and driving advanced technological solutions.



\cleardoublepage

\thispagestyle{empty}

\noindent\rule[-1ex]{\textwidth}{2pt}\\[4.5ex]

Yo, \textbf{Daniel Pérez Ruiz}, alumno de la titulación \textit{Doble Grado en Ingeniería Informática y Matemáticas} de la \textbf{Escuela Técnica Superior de Ingenierías Informática y Telecomunicación} y \textbf{Facultad de Ciencias}, con DNI ***0415**, autorizo la ubicación de la siguiente copia de mi Trabajo de Fin de Grado en las bibliotecas de los centros para que pueda ser consultada por personas que lo deseen.

\vspace{0.5cm}

\noindent
Y para que conste, firmo el presente informe en Granada a Diciembre de 2023.

\vspace{1cm}

\noindent
\textbf{El alumno: Daniel Pérez Ruiz}

\chapter*{Agradecimientos}

Desde que era muy pequeño mi madre siempre me ha dicho que, si ves una mariposa blanca, eso es señal de suerte. Aunque no me caracterizo por ser una persona supersticiosa, siempre he convivido con esa inocencia cada vez que tenía ante mis ojos una de ellas, aleteando, al son de la libertad, iluminando en mí una sonrisa.

\vspace{0.2cm}

A lo largo de mi vida he tenido muchas ocasiones donde sentía que me habían abandonado. Y la realidad, es que siempre me han acompañado, aunque muchas veces no las viera. Ahí se encontraban, danzando al compás de todas las melodías y letras, de todas aquellas canciones que en tiempos pasados resonaron en mi cabeza.

\vspace{0.2cm} 

A mi madre, a mi hermana Irene, a mi padre. A mis amigos de toda una vida Adrián, Alberto, Elías, Juanmi, Juan, Rafa Aguilar y Rafa Barrales. A mis amigos de esta nueva vida Lucía, Martín, Mery, Jaime, Jose, Joseja y Pablo. No hay suficientes palabras en este mundo para describir lo inmensos que son. A mi amiga Candela, por haber por fin coincidido en este camino tras muchas historias y canciones, y que prometo recorrerlo bailando una vez más. A mi amiga Elena, por haber sido una de las voces que me guió y me dio luz cuando no la había. A mi amiga Coral, por haber venido casi al final con su deslumbrante forma de ser, brindándome un nuevo y prometedor comienzo. A mis dos grandes mentores: mi tutor, Carlos, por ser mi fuente de inspiración desde que coincidí con él; y JJ, la persona que me enseñó a cómo encontrarme si me perdía en el camino.

\vspace{0.2cm}
En este trabajo dejo atrás una muy buena parte de mi vida, donde cumplí el sueño de la infancia que me llevó hasta este mismo momento, el de descubrir y aprender todo lo \textit{immenso} que es el mundo de la informática, así como también lo es la vida misma. Esta parte de mí dejará de existir para siempre, pero ahora es mi turno de volar, al igual que todas las mariposas blancas que siempre estuvieron para mí.

\newpage
\vspace*{\fill}

\large
\noindent
$P := \textit{``Estás dispuesto a arriesgar.''}$
\newline
$Q := \textit{``Puedes crecer.''}$
\newline
$R := \textit{``Puedes dar lo mejor de ti.''}$
\newline
$S := \textit{``Puedes ser feliz.''}$
\newline
\newline

\noindent
\textit{Si} $\neg P \implies \neg Q$;
\newline
\textit{Si} $\neg Q \implies \neg R$;
\newline
\textit{Si} $\neg R \implies \neg S$;
\newline
\textit{Y si} $\neg S$, ¿qué te queda?
\newline



\vspace*{\fill}